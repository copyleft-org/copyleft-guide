% comprehensive-gpl-guide.tex                                    -*- LaTeX -*-
%
% Toplevel file to build the entire book.
\documentclass[10pt, letterpaper, openany, oneside]{book}
% I'm somewhat convinced that this book would be better formatted using
%  the memoir class :
%    http://www.ctan.org/pkg/memoir
%   http://mirror.unl.edu/ctan/macros/latex/contrib/memoir/memman.pdf

% For the moment, I've thrown in fancychap because I don't have time to
% research memoir.


% FIXME: Some overall formatting hacks that would really help:

%   * I have started using  \hyperref[LABEL]{text} extensively, which seems
%     to work great in the PDF and HTML versions, but in the Postscript
%     version, the link lost entirely.  I think we need an additional command
%     to replace \hyperref which takes an optional third argument that will
%     insert additional text only when generating print versions, such as:
%      \newhyperref[GPLv2s3]{the requirements for binary distribution under
%      GPLv2}{(see section~\ref*{GPLv2s3} for more information)}
%
%     This is a careful balance, because it'd be all too easy to over-pepper
%     the printed version with back/forward references, but there are
%     probably times when this is useful.

%   * Similar issue: \href{} is well known not to carry the URLs in the print
%     versions.  Adding a footnote with the URL for the print version is
%     probably right.  (or maybe a References page?)

%   * The text is extremely inconsistent regarding formatting of code and
%     commands.  The following varied different methods have been used:
%         + the \verb%..% inline form
%         + verbatim environment (i.e., \begin{verbatim}
%         + {\tt }
%         + \texttt{}
%         + the lstlisting environment (i.e., \begin{lstlisting}
%     These should be made consistent, using only two forms: one for line and
%     one for a long quoted section.



% FIXME: s/GPL enforcers/COGEOs/g

%        (the term coined later but not used throughout) This can't be done
%        by rote, since it may not be appropriate everywhere and shouldn't be
%        used *before* it's coined in the early portions of
%        compliance-guide.tex (and it's probably difficult to coin it earlier
%        anyway).  BTW, I admit COGEOs isn't the best acronym, but I started
%        with ``Community Enforcement Organizations'', which makes CEO, which
%        is worse. :)  My other opting was   COEO, which seemed too close to
%        CEO.  Suggestions welcome.

\usepackage{hyperref}
\usepackage{listings}
\usepackage{enumerate}
\usepackage{enumitem}
\usepackage[Conny]{fncychap}
\usepackage[dvips]{graphicx}
\usepackage[verbose, twoside, dvips,
              paperwidth=8.5in, paperheight=11in,
              left=1in, right=1in, top=1.25in, bottom=.75in,
           ]{geometry}

\newcommand{\tutorialpartsplit}[2]{#2}

%% BEGIN CODE TO FORCE NO PAGE NUMBER ON ToC
\usepackage{tocloft}
\addtocontents{toc}{\cftpagenumbersoff{part}} %% Similarly for subsection, figure... as you wish
\addtocontents{toc}{\cftpagenumbersoff{section}} %% Similarly for subsection, figure... as you wish
\addtocontents{toc}{\cftpagenumbersoff{chapter}}
\addtocontents{toc}{\cftpagenumbersoff{section}} %% Similarly for subsection, figure... as you wish
\addtocontents{toc}{\cftpagenumbersoff{subsection}} %% Similarly for subsection, figure... as you wish
\renewcommand{\cftdot}{} %empty {} for no dots. you can have any symbol inside. For example put {\ensuremath{\ast}} and see what happens.
% END  CODE TO FORCE NO PAGE NUMBER ON ToC

\providecommand{\hrefnofollow}[2]{\href{#1}{#2}}

\hypersetup{pdfinfo={Title={Copyleft and the GNU General Public License: A Comprehensive Tutorial and Guide}}}

    \begin{document}

\pagestyle{plain}
\pagenumbering{roman}

\frontmatter

\begin{titlepage}

\begin{center}

{\Huge
{\sc Copyleft and the  \\

GNU General Public License:

\vspace{.25in}

A Comprehensive Tutorial \\

\vspace{.1in}

and Guide
}}
\vfill

{\parindent 0in
\begin{tabbing}
Copyright \= \copyright{} 2003--2007, 2014 \hspace{1.mm} \=  \kill
Copyright \> \copyright{} 2014 \>  Bradley M. Kuhn. \\
Copyright \> \copyright{} 2014 \>  Anthony K. Sebro, Jr. \\
Copyright \= \copyright{} 2014 \> Denver Gingerich \\
Copyright \= \copyright{} 2003--2007, 2014 \> \hspace{.2in} Free Software Foundation, Inc. \\
Copyright \> \copyright{} 2008, 2014 \>  Software Freedom Law Center. \\
\end{tabbing}

\vspace{.3in}

The copyright holders hereby grant the freedom to copy, modify, convey,
Adapt, and/or redistribute this work under the terms of the Creative Commons
Attribution Share Alike 4.0 International License.  A copy of that license is
available at \url{https://creativecommons.org/licenses/by-sa/4.0/legalcode}.

Each part of this book, except the appendix, is separately under this same
license, but copyrighted by different entities at different times.  Each part
therefore also contains its own copyright and licensing notice.  The notice
above is for the entire work, and includes the full copyright and licensing
details, except for the appendix.

The appendix includes copies of the texts of various licenses published
by the FSF, and they are all licensed under the license, ``Everyone is permitted
to copy and distribute verbatim copies of this license document, but changing
it is not allowed.''.  However, those who seek to make modified versions of
those licenses should note the
\href{https://www.gnu.org/licenses/gpl-faq.html#ModifyGPL}{explanation given in the GPL FAQ}.

\vfill

This material is regularly updated by a community of contributors and is
available online at all times at \url{https://copyleft.org/guide/}.  Patches
are indeed welcome to this material.  Sources can be found in the Git
repository at: \url{https://gitorious.org/copyleft-org/tutorial/}
}
\end{center}

\end{titlepage}

\tableofcontents

\chapter{Preface}

This tutorial is the culmination of nearly a decade of studying and writing
about software freedom licensing and the GPL\@.  Each part of this tutorial
is a course unto itself, educating the reader on a myriad of topics from the
deep details of the GPLv2 and GPLv3, common business models in the copyleft
licensing area (both the friendly and unfriendly kind), best practices for
compliance with the GPL, for engineers, managers, and lawyers, as well as
real-world case studies of GPL enforcement matters.

It is unlikely that all the information herein is necessary to learn all at
once, and therefore this tutorial likely serves best as a reference book.
The material herein has been used as the basis for numerous live tutorials
and discussion groups since 2002, and the materials have been periodically
updated.   They likely stand on their own as excellent reference material.

However, if you are reading these course materials without attending a live
tutorial session, please note that this material is merely a summary of the
highlights of the various CLE and other tutorial courses based on this
material.  Please be aware that during the actual courses, class discussion
and presentation supplements this printed curriculum.  Simply reading this
material is \textbf{not equivalent} to attending a course.

\mainmatter

% gpl-lgpl.tex                                                  -*- LaTeX -*-
%      Tutorial Text for the Detailed Study and Analysis of GPL and LGPL course
%
% Copyright (C) 2003, 2004, 2005 Free Software Foundation, Inc.
% Copyright (C) 2014             Bradley M. Kuhn

% License: CC-By-SA-4.0

% The copyright holders hereby grant the freedom to copy, modify, convey,
% Adapt, and/or redistribute this work under the terms of the Creative
% Commons Attribution Share Alike 4.0 International License.

% This text is distributed in the hope that it will be useful, but
% WITHOUT ANY WARRANTY; without even the implied warranty of
% MERCHANTABILITY or FITNESS FOR A PARTICULAR PURPOSE.

% You should have received a copy of the license with this document in
% a file called 'CC-By-SA-4.0.txt'.  If not, please visit
% https://creativecommons.org/licenses/by-sa/4.0/legalcode to receive
% the license text.

\newcommand{\defn}[1]{\emph{#1}}

\part{Detailed Analysis of the GNU GPL and Related Licenses}

\begin{center}

{\parindent 0in
This part is: \\
\begin{tabbing}
Copyright \= \copyright{} 2003, 2004, 2005 \= \hspace{.2in} Free Software Foundation, Inc. \\
Copyright \= \copyright{} 2014 \= \hspace{.2in} Bradley M. Kuhn \\
\end{tabbing}

Authors of this Part Are: \\

Bradley M. Kuhn \\
David ``Novalis'' Turner \\
Daniel B. Ravicher \\
John Sullivan

\vspace{.3in}

The copyright holders of this part hereby grant the freedom to copy, modify,
convey, Adapt, and/or redistribute this work under the terms of the Creative
Commons Attribution Share Alike 4.0 International License.  A copy of that
license is available at
\verb=https://creativecommons.org/licenses/by-sa/4.0/legalcode=.  }

\end{center}

\bigskip

This part of the tutorial gives a comprehensive explanation of the most
popular Free Software copyright license, the GNU General Public License
(``GNU GPL'', or sometimes just ``GPL'') -- both version 2 (``GPLv2'') and
version 3 (``GPLv3'') -- and teaches lawyers, software developers, managers
and business people how to use the GPL (and GPL'd software) successfully both
as a community-building ``Constitution'' for a software project, or to
incorporate copylefted software into a new Free Software business and in
existing, successful enterprises.

To successfully benefit of from this part of the tutorial, readers should
have a general familiarity with software development processes.  A vague
understanding of how copyright law applies to software is also helpful.  The
tutorial is of most interest to lawyers, software developers and managers who
run or advise software businesses that modify and/or redistribute software
under terms of the GNU GPL (or who wish to do so in the future), and those
who wish to make use of existing GPL'd software in their enterprise.

Upon completion of this part of the tutorial, successful students can expect
to have learned the following:

\begin{itemize}

  \item The freedom-defending purpose of each term of the GNU GPLv2 and GPLv3.

  \item The differences between GPLv2 and GPLv3.

  \item The redistribution options under the GPLv2 and GPLv3.

  \item The obligations when modifying GPLv2'd or GPLv3'd software.

  \item How to build a plan for proper and successful compliance with the GPL.

  \item The business advantages that the GPL provides.

  \item The most common business models used in conjunction with the GPL.

  \item How existing GPL'd software can be used in existing enterprises.

  \item The basics of the Lesser GPLv2.1 and Lesser GPLv3, and how they
    differs from the GPLv2 and GPLv3, respectively.

  \item The basics to begin understanding the complexities regarding
    derivative and combined works of software.
\end{itemize}

%%%%%%%%%%%%%%%%%%%%%%%%%%%%%%%%%%%%%%%%%%%%%%%%%%%%%%%%%%%%%%%%%%%%%%%%%%%%%%%
% END OF ABSTRACTS SECTION
%%%%%%%%%%%%%%%%%%%%%%%%%%%%%%%%%%%%%%%%%%%%%%%%%%%%%%%%%%%%%%%%%%%%%%%%%%%%%%%
% START OF DAY ONE COURSE
%%%%%%%%%%%%%%%%%%%%%%%%%%%%%%%%%%%%%%%%%%%%%%%%%%%%%%%%%%%%%%%%%%%%%%%%%%%%%%%

\chapter{What Is Software Freedom?}

Study of the GNU General Public License (herein, abbreviated as \defn{GNU
  GPL} or just \defn{GPL}) must begin by first considering the broader world
of software freedom. The GPL was not created from a void, rather, it was
created to embody and defend a set of principles that were set forth at the
founding of the GNU project and the Free Software Foundation (FSF) -- the
organization that upholds, defends and promotes the philosophy of software
freedom. A prerequisite for understanding both of the popular versions of GPL
(GPLv2 and GPLv3) and their terms and conditions is a basic understanding of
the principles behind it.  The GPL family of licenses are unlike almost all
other software licenses in that they are designed to defend and uphold these
principles.

\section{The Free Software Definition}
\label{Free Software Definition}

The Free Software Definition is set forth in full on FSF's website at
\verb0http://fsf.org/0 \verb0philosophy/free-sw.html0. This section presents
an abbreviated version that will focus on the parts that are most pertinent
to the GPL\@.

A particular program grants software freedom to a particular user if that
user is granted the following freedoms:

\begin{itemize}


\teim The freedom to run the program, for any purpose.

\item The freedom to study how the program works, and modify it

\item The freedom to redistribute copies.

\item The freedom to distribute copies of  modified versions to others.

\end{itemize}

The focus on ``a particular user'' is particularly pertinent here.  It is not
uncommon for the same version of a specific program to grant these freedoms
to some subset of its user base, while others have none or only some of these
freedoms.  Section~\ref{Proprietary Relicensing} talks in detail about how
this can unfortunately happen even if a program is released under the GPL\@.

Many people refer to software that gives these freedoms as ``Open Source.''
Besides having a different political focus than those who call it Free
Software,\footnote{The political differences between the Free Software
  Movement and the Open Source Movement are documented on FSF's Web site at
  {\tt http://www.fsf.org/licensing/essays/free-software-for-freedom.html}.}
those who call the software ``Open Source'' are often focused on a side
issue.  Specifically, user access to the source code of a program is a
prerequisite to make use of the freedom to modify.  However, the important
issue is what freedoms are granted in the license of that source code.

Software freedom is only complete when no restrictions are imposed on how
these freedoms are exercised.  Specifically, users and programmers can
exercise these freedoms noncommercially or commercially.  Licenses that grant
these freedoms for noncommercial activities but prohibit them for commercial
activities are considered non-free.  Even the Open Source Initiative
(\defn{OSI}) (the arbiter of what is considered ``Open Source'') also rules
such licenses not in fitting with their ``Open Source Definition''.

In general, software for which most or all of these freedoms are
restricted in any way is called ``non-Free Software.''  Typically, the
term ``proprietary software'' is used more or less interchangeably with
``non-Free Software.''  Personally, I tend to use the term ``non-Free
Software'' to refer to noncommercial software that restricts freedom
(such as ``shareware'') and ``proprietary software'' to refer to
commercial software that restricts freedom (such as nearly all of
Microsoft's and Oracle's offerings).

Keep in mind that the none of the terms ``software freedom'', ``open source''
and ``free software'' are not known to be trademarked by any organization in
any jurisdiction.  As such, it's quite common that these terms are abused and
misused by parties who wish to bank on the popularity of software freedom.
When one considers using, modifying or redistributing a software package that
purports to be Open Source or Free Software, one \textbf{must} verify that
the license grants software freedom

Furthermore, throughout this text, we generally prefer the term ``software
freedom'', as this is the least ambiguous term available to describe software
that meets the Free Software Definition.  For example, it is well known and
often discussed that the adjective ``free'' has two unrelated meanings in
English: ``free as in freedom'' and ``free as in price''.  Meanwhile, the
term ``open source'' is even more confusing, because it refers only to the
``freedom to study'', which is merely a subset of one of the four freedoms.

The remainder of this section considers each of each component of software
freedom in detail.

\subsection{The Freedom to Run}

The first tenant of software freedom is the user's fully unfettered right to
run the program.  The software's license must permit any conceivable use of
the software.  Perhaps, for example, the user has discovered an innovative
use for a particular program, one that the programmer never could have
predicted.  Such a use must not be restricted.

It was once rare that this freedom was restricted by even proprietary
software; but such is quite common today. Most End User Licensing Agreements
(EULAs) that cover most proprietary software typically restrict some types of
uses.  Such restrictions of any kind are an unacceptable restriction on
software freedom.

\subsection{The Freedom to Change and Modify}

Perhaps the most useful right of software freedom is the users' right to
change, modify and adapt the software to suit their needs.  Access to the
source code and related build and installation scripts are an essential part
of this freedom.  Without the source code, and the ability to build and
install the binary applications from that source, users cannot effectively
exercise this freedom.

Programmers take direct benefit from this freedom.  However, this freedom
remains important to users who are not programmers.  While it may seem
counterintuitive at first, non-programmer users often exercise this freedom
indirectly in both commercial and noncommercial settings.  For example, users
often seek noncommercial help with the software on email lists and in users
groups.  To make use of such help they must either have the freedom to
recruit programmers who might altruistically assist them to modify their
software, or to at least follow rote instructions to make basic modifications
themselves.

More commonly, users also exercise this freedom commercially.  Each user, or
group of users, may hire anyone they wish in a competitive free market to
modify and change the software.  This means that companies have a right to
hire anyone they wish to modify their Free Software.  Additionally, such
companies may contract with other companies to commission software
modification.

\subsection{The Freedom to Copy and Share}

Users may share Free Software in a variety of ways. Free Software
advocates work to eliminate a fundamental ethical dilemma of the software
age: choosing between obeying a software license, and friendship (by
giving away a copy of a program to your friend who likes the software you are
using). Free Software licenses, therefore, must permit this sort of
altruistic sharing of software among friends.

The commercial environment must also have the benefits of this freedom.
Commercial sharing typically takes the form of selling copies of Free
Software. Free Software can be sold at any price to anyone. Those who
redistribute Free Software commercially have the freedom to selectively
distribute (you can pick your customers) and to set prices at any level
the redistributor sees fit.

It is true that many people get copies of Free Software very cheaply (and
sometimes without charge). The competitive free market of Free Software
tends to keep prices low and reasonable. However, if someone is willing
to pay a billion dollars for one copy of the GNU Compiler Collection, such
a sale is completely permitted.

Another common instance of commercial sharing is service-oriented
distribution. For example, a distribution vendor may provide immediate
security and upgrade distribution via a special network service. Such
distribution is completely permitted for Free Software.

(Section~\ref{Business Models} of this tutorial talks in detail about
various Free Software business models that take advantage of the freedom
to share commercially.)

\subsection{The Freedom to Share Improvements}

The freedom to modify and improve is somewhat empty without the freedom to
share those improvements. The Free Software community is built on the
pillar of altruistic sharing of improved Free Software. Inevitably, a
Free Software project sprouts a mailing list where improvements are shared
freely among members of the development community. Such noncommercial
sharing must be permitted for Free Software to thrive.

Commercial sharing of modified Free Software is equally important.
For commercial support to exist in a competitive free market, all
developers --- from single-person contractors to large software
companies --- must have the freedom to market their services as
improvers of Free Software. All forms of such service marketing must
be equally available to all.

For example, selling support services for Free Software is fully
permitted. Companies and individuals can offer themselves as ``the place
to call'' when software fails or does not function properly. For such a
service to be meaningful, the entity offering that service must have the
right to modify and improve the software for the customer to correct any
problems that are beyond mere user error.

Entities must also be permitted to make available modified versions of
Free Software. Most Free Software programs have a ``standard version''
that is made available from the primary developers of the software.
However, all who have the software have the ``freedom to fork'' --- that
is, make available nontrivial modified versions of the software on a
permanent or semi-permanent basis. Such freedom is central to vibrant
developer and user interaction.

Companies and individuals have the right to make true value-added versions
of Free Software. They may use freedom to share improvements to
distribute distinct versions of Free Software with different functionality
and features. Furthermore, this freedom can be exercised to serve a
disenfranchised subset of the user community. If the developers of the
standard version refuse to serve the needs of some of the software's
users, other entities have the right to create a long- or short-lived fork
to serve that sub-community.

\section{How Does Software Become Free?}

The last section set forth the freedoms and rights respected by Free
Software. It presupposed, however, that such software exists. This
section discusses how Free Software comes into existence. But first, it
addresses how software can be non-Free in the first place.

Software can be made proprietary only because it is governed by copyright
law.\footnote{This statement is a bit of an oversimplification. Patents
  and trade secrets can cover software and make it effectively non-Free,
  one can contract away their rights and freedoms regarding software, or
  source code can be practically obscured in binary-only distribution
  without reliance on any legal system. However, the primary control
  mechanism for software is copyright.} Copyright law, with respect to
software, governs copying, modifying, and redistributing that
software.\footnote{Copyright law in general also governs ``public
  performance'' of copyrighted works. There is no generally agreed
  definition for public performance of software and both GPLv2 and GPLv3
  do not govern public performance.} By law, the copyright holder (a.k.a.
the author) of the work controls how others may copy, modify and/or
distribute the work. For proprietary software, these controls are used to
prohibit these activities. In addition, proprietary software distributors
further impede modification in a practical sense by distributing only
binary code and keeping the source code of the software secret.

Copyright law is a construction. In the USA, the Constitution permits,
but does not require, the creation of copyright law as federal
legislation. Software, since it is an idea fixed in a tangible medium, is
thus covered by the statues, and is copyrighted by default.

However, this legal construction is not necessarily natural. Software, in
its natural state without copyright, is Free Software. In an imaginary
world with no copyright, the rules would be different. In this
world, when you received a copy of a program's source code, there would be
no default legal system to restrict you from sharing it with others,
making modifications, or redistributing those modified
versions.\footnote{There could still exist legal systems, like our modern
  patent system, which could restrict the software in other ways.}

Software in the real world is copyrighted by default and is
automatically covered by that legal system. However, it is possible
to move software out of the domain of the copyright system. A
copyright holder is always permitted to \defn{disclaim} their
copyright. If copyright is disclaimed, the software is not governed
by copyright law. Software not governed by copyright is in the
``public domain.''

\subsection{Public Domain Software}
% FIXME: this section needs more improvements to make it clear that public
% domain dedication is difficult, if not impossible.
% Karen suggests that talking about USA government software being public
% domain might make sense here.
Theoretically, an author can create public domain software by disclaiming all
copyright interest on the work. In the USA and other countries that have
signed the Berne convention on copyright, software is copyrighted
automatically by the author when she ``fixes the software into a tangible
medium.''  In the software world, this usually means typing the source code
of the software into a file.

Imagine if an author can truly disclaim that default control given to her by the
copyright laws. Once this is done, the software is in the public domain
--- it is no longer covered by copyright. Since it is copyright law that
allows for various controls on software (i.e., prohibition of copying,
modification, and redistribution), removing the software from the
copyright system and placing it into the public domain does yield Free
Software.

Carefully note that software in the public domain is \emph{not} licensed
in any way. It is nonsensical to say software is ``licensed for the
public domain,'' or any phrase that implies the copyright holder gave
expressed permission to take actions governed by copyright law.

By contrast, what the copyright holder has done is renounce her copyright
controls on the work. The law gave her controls over the work, and she
has chosen to waive those controls. Software in the public domain is
absent copyright and absent a license. The software freedoms discussed in
Section~\ref{Free Software Definition} are all granted because there is no
legal system in play to take them away.

\subsection{Why Copyright Free Software?}

If simply disclaiming copyright on software yields Free Software, then it
stands to reason that putting software into the public domain is the
easiest and most straightforward way to produce Free Software. Indeed,
some major Free Software projects have chosen this method for making their
software Free. However, most of the Free Software in existence \emph{is}
copyrighted. In most cases (particularly in those of FSF and the GNU
Project), this was done due to very careful planning.

Software released into the public domain does grant freedom to those users
who receive the standard versions on which the original author disclaimed
copyright. However, since the work is not copyrighted, any nontrivial
modification made to the work is fully copyrightable.

Free Software released into the public domain initially is Free, and
perhaps some who modify the software choose to place their work into the
public domain as well. However, over time, some entities will choose to
proprietarize their modified versions. The public domain body of software
feeds the proprietary software. The public commons disappears, because
fewer and fewer entities have an incentive to contribute back to the
commons. They know that any of their competitors can proprietarize their
enhancements. Over time, almost no interesting work is left in the public
domain, because nearly all new work is done by proprietarization.

A legal mechanism is needed to redress this problem. FSF was in fact
originally created primarily as a legal entity to defend software freedom,
and that work of defending software freedom is a substantial part of
its work today. Specifically because of this ``embrace, proprietarize and
extend'' cycle, FSF made a conscious choice to copyright its Free Software,
and then license it under ``copyleft'' terms. Many, including the
developers of the kernel named Linux, have chosen to follow this paradigm.

Copyleft is a legal strategy to defend, uphold and propagate software
freedom. The basic technique of copyleft is as follows: copyright the
software, license it under terms that give all the software freedoms, but
use the copyright law controls to ensure that all who receive a copy of
the software have equal rights and freedom. In essence, copyleft grants
freedom, but forbids others to forbid that freedom to anyone else along
the distribution and modification chains.

Copyleft is a general concept. Much like ideas for what a computer might
do must be \emph{implemented} by a program that actually does the job, so
too must copyleft be implemented in some concrete legal structure.
``Share and share alike'' is a phrase that is used often enough to explain the
concept behind copyleft, but to actually make it work in the real world, a
true implementation in legal text must exist. The GPL is the primary
implementation of copyleft in copyright licensing language.

\section{A Community of Equality}

The GPL uses copyright law to defend freedom and equally ensure users'
rights. This ultimately creates an community of equality for both
business and noncommercial users.

\subsection{The Noncommercial Community}

A GPL'd code base becomes a center of a vibrant development and user
community. Traditionally, volunteers, operating noncommercially out of
keen interest or ``scratch an itch'' motivations, produce initial versions
of a GPL'd system. Because of the efficient distribution channels of the
Internet, any useful GPL'd system is adopted quickly by noncommercial
users.

Fundamentally, the early release and quick distribution of the software
gives birth to a thriving noncommercial community. Users and developers
begin sharing bug reports and bug fixes across a shared intellectual
commons. Users can trust the developers, because they know that if the
developers fail to address their needs or abandon the project, the GPL
ensures that someone else has the right to pick up development.
Developers know that the users cannot redistribute their software without
passing along the rights granted by GPL, so they are assured that every
one of their users is treated equally.

Because of the symmetry and fairness inherent in GPL'd distribution,
nearly every GPL'd package in existence has a vibrant noncommercial user
and developer base.

\subsection{The Commercial Community}

By the same token, nearly all established GPL'd software systems have a
vibrant commercial community. Nearly every GPL'd system that has gained
wide adoption from noncommercial users and developers eventually begins
to fuel a commercial system around that software.

For example, consider the Samba file server system that allows Unix-like
systems (including GNU/Linux) to serve files to Microsoft Windows systems.
Two graduate students originally developed Samba in their spare time and
it was deployed noncommercially in academic environments. However, very
soon for-profit companies discovered that the software could work for them
as well, and their system administrators began to use it in place of
Microsoft Windows NT file-servers. This served to lower the cost of
running such servers by orders of magnitude. There was suddenly room in
Windows file-server budgets to hire contractors to improve Samba. Some of
the first people hired to do such work were those same two graduate
students who originally developed the software.

The noncommercial users, however, were not concerned when these two
fellows began collecting paychecks off of their GPL'd work. They knew
that because of the nature of the GPL that improvements that were
distributed in the commercial environment could easily be folded back into
the standard version. Companies are not permitted to proprietarize
Samba, so the noncommercial users, and even other commercial users are
safe in the knowledge that the software freedom ensured by GPL will remain
protected.

Commercial developers also work in concert with noncommercial
developers. Those two now-long-since graduated students continue to
contribute to Samba altruistically, but also get paid work doing it.
Priorities change when a client is in the mix, but all the code is
contributed back to the standard version. Meanwhile, many other
individuals have gotten involved noncommercially as developers,
because they want to ``cut their teeth on Free Software,'' or because
the problems interest them. When they get good at it, perhaps they
will move on to another project, or perhaps they will become
commercial developers of the software themselves.

No party is a threat to another in the GPL software scenario because
everyone is on equal ground. The GPL protects rights of the commercial
and noncommercial contributors and users equally. The GPL creates trust,
because it is a level playing field for all.

\subsection{Law Analogy}

In his introduction to Stallman's \emph{Free Software, Free Society},
Lawrence Lessig draws an interesting analogy between the law and Free
Software. He argues that the laws of a free society must be protected
much like the GPL protects software. So that I might do true justice to
Lessig's argument, I quote it verbatim:

\begin{quotation}

A ``free society'' is regulated by law. But there are limits that any free
society places on this regulation through law: No society that kept its
laws secret could ever be called free. No government that hid its
regulations from the regulated could ever stand in our tradition. Law
controls. But it does so justly only when visibly. And law is visible
only when its terms are knowable and controllable by those it regulates,
or by the agents of those it regulates (lawyers, legislatures).

This condition on law extends beyond the work of a legislature. Think
about the practice of law in American courts. Lawyers are hired by their
clients to advance their clients' interests. Sometimes that interest is
advanced through litigation. In the course of this litigation, lawyers
write briefs. These briefs in turn affect opinions written by judges.
These opinions decide who wins a particular case, or whether a certain law
can stand consistently with a constitution.

All the material in this process is free in the sense that Stallman means.
Legal briefs are open and free for others to use. The arguments are
transparent (which is different from saying they are good), and the
reasoning can be taken without the permission of the original lawyers.
The opinions they produce can be quoted in later briefs. They can be
copied and integrated into another brief or opinion. The ``source code''
for American law is by design, and by principle, open and free for anyone
to take. And take lawyers do---for it is a measure of a great brief that
it achieves its creativity through the reuse of what happened before. The
source is free; creativity and an economy is built upon it.

This economy of free code (and here I mean free legal code) doesn't starve
lawyers. Law firms have enough incentive to produce great briefs even
though the stuff they build can be taken and copied by anyone else. The
lawyer is a craftsman; his or her product is public. Yet the crafting is
not charity. Lawyers get paid; the public doesn't demand such work
without price. Instead this economy flourishes, with later work added to
the earlier.

We could imagine a legal practice that was different---briefs and
arguments that were kept secret; rulings that announced a result but not
the reasoning. Laws that were kept by the police but published to no one
else. Regulation that operated without explaining its rule.

We could imagine this society, but we could not imagine calling it
``free.''  Whether or not the incentives in such a society would be better
or more efficiently allocated, such a society could not be known as free.
The ideals of freedom, of life within a free society, demand more than
efficient application. Instead, openness and transparency are the
constraints within which a legal system gets built, not options to be
added if convenient to the leaders. Life governed by software code should
be no less.

Code writing is not litigation. It is better, richer, more
productive. But the law is an obvious instance of how creativity and
incentives do not depend upon perfect control over the products
created. Like jazz, or novels, or architecture, the law gets built
upon the work that went before. This adding and changing is what
creativity always is. And a free society is one that assures that its
most important resources remain free in just this sense.\footnote{This
quotation is Copyright \copyright{} 2002, Lawrence Lessig. It is
licensed under the terms of
\texttt{http://creativecommons.org/licenses/by/1.0/}{the ``Attribution
License'' version 1.0} or any later version as published by Creative
Commons.}
\end{quotation}

In essence, lawyers are paid to service the shared commons of legal
infrastructure. Few citizens defend themselves in court or write their
own briefs (even though they are legally permitted to do so) because
everyone would prefer to have an expert do that job.

The Free Software economy is a market ripe for experts. It
functions similarly to other well established professional fields like the
law. The GPL, in turn, serves as the legal scaffolding that permits the
creation of this vibrant commercial and noncommercial Free Software
economy.

%%%%%%%%%%%%%%%%%%%%%%%%%%%%%%%%%%%%%%%%%%%%%%%%%%%%%%%%%%%%%%%%%%%%%%%%%%%%%%%
\chapter{A Tale of Two Copyleft Licenses}

\section{Historical Motivations for the General Public License}

\section{Proto-GPLs And Their Impact}

\section{The GNU General Public License, Version 1}

\section{The GNU General Public License, Version 2}

\section{The GNU General Public License, Version 3}

\section{The Innovation of Optional ``Or Any Later'' Version}

\section{Complexities of Two Simultaneously Popular Copylefts}

%%%%%%%%%%%%%%%%%%%%%%%%%%%%%%%%%%%%%%%%%%%%%%%%%%%%%%%%%%%%%%%%%%%%%%%%%%%%%%%
\chapter{GPLv2: Running Software and Verbatim Copying}
\label{run-and-verbatim}


This chapter begins the deep discussion of the details of the terms of
GPLv2\@. In this chapter, we consider the first two sections: GPLv2 \S\S
0--2. These are the straightforward sections of the GPL that define the
simplest rights that the user receives.

\section{GPLv2 \S 0: Freedom to Run}
\label{GPLs0}

\S 0, the opening section of GPLv2, sets forth that the work is governed by
copyright law. It specifically points out that it is the ``copyright
holder'' who decides if a work is licensed under its terms and explains
how the copyright holder might indicate this fact.

A bit more subtly, \S 0 makes an inference that copyright law is the only
system under which it is governed. Specifically, it states:
\begin{quote}
Activities other than copying, distribution and modification are not
covered by this License; they are outside its scope.
\end{quote}
In essence, the license governs \emph{only} those activities, and all other
activities are unrestricted, provided that no other agreements trump GPLv2
(which they cannot; see Sections~\ref{GPLs6} and~\ref{GPLs7}). This is
very important, because the Free Software community heavily supports
users' rights to ``fair use'' and ``unregulated use'' of copyrighted
material. GPLv2 asserts through this clause that it supports users' rights
to fair and unregulated uses.

Fair use of copyrighted material is an established legal doctrine that
permits certain activities. Discussion of the various types of fair
use activity are beyond the scope of this tutorial. However, one
important example of fair use is the right to quote a very few lines
(less than seven or so) and reuse them as you would with or without
licensing restrictions.

Fair use is a doctrine established by the courts or by statute. By
contrast, unregulated uses are those that are not covered by the statue
nor determined by a court to be covered, but are common and enjoyed by
many users. An example of unregulated use is reading a printout of the
program's source code like an instruction book for the purpose of learning
how to be a better programmer.

\medskip

Thus, the GPLv2 protects users fair and unregulated use rights precisely by
not attempting to cover them. Furthermore, the GPLv2 ensures the freedom
to run specifically by stating the following:
\begin{quote}
''The act of running the Program is not restricted.''
\end{quote}
Thus, users are explicitly given the freedom to run by \S 0.

\medskip

The bulk of \S 0 not yet discussed gives definitions for other terms used
throughout. The only one worth discussing in detail is ``work based on
the Program.''  The reason this definition is particularly interesting is
not for the definition itself, which is rather straightforward, but
because it clears up a common misconception about the GPL\@.

The GPL is often mistakenly criticized because it fails to give a
definition of ``derivative work.''  In fact, it would be incorrect and
problematic if the GPL attempted to define this. A copyright license, in
fact, has no control over what may or may not be a derivative work. This
matter is left up to copyright law, not the licenses that utilize it.

It is certainly true that copyright law as a whole does not propose clear
and straightforward guidelines for what is and is not a derivative
software work under copyright law. However, no copyright license --- not
even the GNU GPL --- can be blamed for this. Legislators and court
opinions must give us guidance to decide the border cases.

\section{GPLv2 \S 1: Verbatim Copying}
\label{GPLs1}

GPLv2 \S 1 covers the matter of redistributing the source code of a program
exactly as it was received. This section is quite straightforward.
However, there are a few details worth noting here.

The phrase ``in any medium'' is important. This, for example, gives the
freedom to publish a book that is the printed copy of the program's source
code. It also allows for changes in the medium of distribution. Some
vendors may ship Free Software on a CD, but others may place it right on
the hard drive of a pre-installed computer. Any such redistribution media
is allowed.

Preservation of copyright notice and license notifications are mentioned
specifically in \S 1. These are in some ways the most important part of
the redistribution, which is why they are mentioned by name. The GPL
always strives to make it abundantly clear to anyone who receives the
software what its license is. The goal is to make sure users know their
rights and freedoms under GPL, and to leave no reason that someone would be
surprised the software she got was licensed under GPL\@. Thus
throughout the GPL, there are specific references to the importance of
notifying others down the distribution chain that they have rights under
GPL.

Also mentioned by name is the warranty disclaimer. Most people today do
not believe that software comes with any warranty. Notwithstanding the
proposed state-level UCITA bills (which have never obtained widespread
adoption), there are few or no implied warranties with software.
However, just to be on the safe side, GPL clearly disclaims them, and the
GPL requires redistributors to keep the disclaimer very visible. (See
Sections~\ref{GPLs11} and~\ref{GPLs12} of this tutorial for more on GPL's
warranty disclaimers.)

Note finally that \S 1 begins to set forth the important defense of
commercial freedom. \S 1 clearly states that in the case of verbatim
copies, one may make money. Redistributors are fully permitted to charge
for the redistribution of copies of Free Software. In addition, they may
provide the warranty protection that the GPL disclaims as an additional
service for a fee. (See Section~\ref{Business Models} for more discussion
on making a profit from Free Software redistribution.)

%%%%%%%%%%%%%%%%%%%%%%%%%%%%%%%%%%%%%%%%%%%%%%%%%%%%%%%%%%%%%%%%%%%%%%%%%%%%%%%

\chapter{Derivative Works: Statute and Case Law}

We digress for this chapter from our discussion of GPL's exact text to
consider the matter of derivative works --- a concept that we must
understand fully before considering \S\S 2--3 of GPLv2\@. GPL, and Free
Software licensing in general, relies critically on the concept of
``derivative work'' since software that is ``independent,'' (i.e., not
``derivative'') of Free Software need not abide by the terms of the
applicable Free Software license. As much is required by \S 106 of the
Copyright Act, 17 U.S.C. \S 106 (2002), and admitted by Free Software
licenses, such as the GPL, which (as we have seen) states in \S 0 that ``a
`work based on the Program' means either the Program or any derivative
work under copyright law.'' It is being a derivative work of Free Software
that triggers the necessity to comply with the terms of the Free Software
license under which the original work is distributed. Therefore, one is
left to ask, just what is a ``derivative work''? The answer to that
question differs depending on which court is being asked.

The analysis in this chapter sets forth the differing definitions of
derivative work by the circuit courts. The broadest and most
established definition of derivative work for software is the
abstraction, filtration, and comparison test (``the AFC test'') as
created and developed by the Second Circuit. Some circuits, including
the Ninth Circuit and the First Circuit, have either adopted narrower
versions of the AFC test or have expressly rejected the AFC test in
favor of a narrower standard. Further, several other circuits have yet
to adopt any definition of derivative work for software.

As an introductory matter, it is important to note that literal copying of
a significant portion of source code is not always sufficient to establish
that a second work is a derivative work of an original
program. Conversely, a second work can be a derivative work of an original
program even though absolutely no copying of the literal source code of
the original program has been made. This is the case because copyright
protection does not always extend to all portions of a program's code,
while, at the same time, it can extend beyond the literal code of a
program to its non-literal aspects, such as its architecture, structure,
sequence, organization, operational modules, and computer-user interface.

\section{The Copyright Act}

The copyright act is of little, if any, help in determining the definition
of a derivative work of software. However, the applicable provisions do
provide some, albeit quite cursory, guidance. Section 101 of the Copyright
Act sets forth the following definitions:

\begin{quotation}
A ``computer program'' is a set of statements or instructions to be used
directly or indirectly in a computer in order to bring about a certain
result.

A ``derivative work'' is a work based upon one or more preexisting works,
such as a translation, musical arrangement, dramatization,
fictionalization, motion picture version, sound recording, art
reproduction, abridgment, condensation, or any other form in which a work
may be recast, transformed, or adapted. A work consisting of editorial
revisions, annotations, elaborations, or other modifications which, as a
whole, represent an original work of authorship, is a ``derivative work.''
\end{quotation}

These are the only provisions in the Copyright Act relevant to the
determination of what constitutes a derivative work of a computer
program. Another provision of the Copyright Act that is also relevant to
the definition of derivative work is \S 102(b), which reads as follows:

\begin{quotation}
In no case does copyright protection for an original work of authorship
extend to any idea, procedure, process, system, method of operation,
concept, principle, or discovery, regardless of the form in which it is
described, explained, illustrated, or embodied in such work.
\end{quotation}

Therefore, before a court can ask whether one program is a derivative work
of another program, it must be careful not to extend copyright protection
to any ideas, procedures, processes, systems, methods of operation,
concepts, principles, or discoveries contained in the original program. It
is the implementation of this requirement to ``strip out'' unprotectable
elements that serves as the most frequent issue over which courts
disagree.

\section{Abstraction, Filtration, Comparison Test}

As mentioned above, the AFC test for determining whether a computer
program is a derivative work of an earlier program was created by the
Second Circuit and has since been adopted in the Fifth, Tenth, and
Eleventh Circuits. Computer Associates Intl., Inc. v. Altai, Inc., 982
F.2d 693 (2nd Cir. 1992); Engineering Dynamics, Inc. v. Structural
Software, Inc., 26 F.3d 1335 (5th Cir. 1994); Kepner-Tregoe,
Inc. v. Leadership Software, Inc., 12 F.3d 527 (5th Cir. 1994); Gates
Rubber Co. v. Bando Chem. Indust., Ltd., 9 F.3d 823 (10th Cir. 1993);
Mitel, Inc. v. Iqtel, Inc., 124 F.3d 1366 (10th Cir. 1997); 5 Bateman
v. Mnemonics, Inc., 79 F.3d 1532 (11th Cir. 1996); and, Mitek Holdings,
Inc. v. Arce Engineering Co., Inc., 89 F.3d 1548 (11th Cir. 1996).

Under the AFC test, a court first abstracts from the original program its
constituent structural parts. Then, the court filters from those
structural parts all unprotectable portions, including incorporated ideas,
expression that is necessarily incidental to those ideas, and elements
that are taken from the public domain. Finally, the court compares any and
all remaining kernels of creative expression to the structure of the
second program to determine whether the software programs at issue are
substantially similar so as to warrant a finding that one is the
derivative work of the other.

Often, the courts that apply the AFC test will perform a quick initial
comparison between the entirety of the two programs at issue in order to
help determine whether one is a derivative work of the other. Such a
holistic comparison, although not a substitute for the full application of
the AFC test, sometimes reveals a pattern of copying that is not otherwise
obvious from the application of the AFC test when, as discussed below,
only certain components of the original program are compared to the second
program. If such a pattern is revealed by the quick initial comparison,
the court is more likely to conclude that the second work is indeed a
derivative of the original.

\subsection{Abstraction}

The first step courts perform under the AFC test is separation of the
work's ideas from its expression. In a process akin to reverse
engineering, the courts dissect the original program to isolate each level
of abstraction contained within it. Courts have stated that the
abstractions step is particularly well suited for computer programs
because it breaks down software in a way that mirrors the way it is
typically created. However, the courts have also indicated that this step
of the AFC test requires substantial guidance from experts, because it is
extremely fact and situation specific.

By way of example, one set of abstraction levels is, in descending order
of generality, as follows: the main purpose, system architecture, abstract
data types, algorithms and data structures, source code, and object
code. As this set of abstraction levels shows, during the abstraction step
of the AFC test, the literal elements of the computer program, namely the
source and object code, are defined as particular levels of
abstraction. Further, the source and object code elements of a program are
not the only elements capable of forming the basis for a finding that a
second work is a derivative of the program. In some cases, in order to
avoid a lengthy factual inquiry by the court, the owner of the copyright in
the original work will submit its own list of what it believes to be the
protected elements of the original program. In those situations, the court
will forgo performing its own abstraction, and proceed to the second step of
the AFC test.

\subsection{Filtration}

The most difficult and controversial part of the AFC test is the second
step, which entails the filtration of protectable expression contained in
the original program from any unprotectable elements nestled therein. In
determining which elements of a program are unprotectable, courts employ a
myriad of rules and procedures to sift from a program all the portions
that are not eligible for copyright protection.

First, as set forth in \S 102(b) of the Copyright Act, any and all ideas
embodied in the program are to be denied copyright protection. However,
implementing this rule is not as easy as it first appears. The courts
readily recognize the intrinsic difficulty in distinguishing between ideas
and expression and that, given the varying nature of computer programs,
doing so will be done on an ad hoc basis. The first step of the AFC test,
the abstraction, exists precisely to assist in this endeavor by helping
the court separate out all the individual elements of the program so that
they can be independently analyzed for their expressive nature.

A second rule applied by the courts in performing the filtration step of
the AFC test is the doctrine of merger, which denies copyright protection
to expression necessarily incidental to the idea being expressed. The
reasoning behind this doctrine is that when there is only one way to
express an idea, the idea and the expression merge, meaning that the
expression cannot receive copyright protection due to the bar on copyright
protection extending to ideas. In applying this doctrine, a court will ask
whether the program's use of particular code or structure is necessary for
the efficient implementation of a certain function or process. If so, then
that particular code or structure is not protected by copyright and, as a
result, it is filtered away from the remaining protectable expression.

A third rule applied by the courts in performing the filtration step of
the AFC test is the doctrine of scenes a faire, which denies copyright
protection to elements of a computer program that are dictated by external
factors. Such external factors can include:

\begin{itemize}

  \item The mechanical
specifications of the computer on which a particular program is intended
to operate

  \item Compatibility requirements of other programs with which a
program is designed to operate in conjunction

  \item Computer manufacturers'
design standards

  \item Demands of the industry being serviced, and

widely accepted programming practices within the computer industry

\end{itemize}

Any code or structure of a program that was shaped predominantly in
response to these factors is filtered out and not protected by
copyright. Lastly, elements of a computer program are also to be filtered
out if they were taken from the public domain or fail to have sufficient
originality to merit copyright protection.

Portions of the source or object code of a computer program are rarely
filtered out as unprotectable elements. However, some distinct parts of
source and object code have been found unprotectable. For example,
constant s, the invariable integers comprising part of formulas used to
perform calculations in a program, are unprotectable. Further, although
common errors found in two programs can provide strong evidence of
copying, they are not afforded any copyright protection over and above the
protection given to the expression containing them.

\subsection{Comparison}

The third and final step of the AFC test entails a comparison of the
original program's remaining protectable expression to a second
program. The issue will be whether any of the protected expression is
copied in the second program and, if so, what relative importance the
copied portion has with respect to the original program overall. The
ultimate inquiry is whether there is ``substantial'' similarity between
the protected elements of the original program and the potentially
derivative work. The courts admit that this process is primarily
qualitative rather than quantitative and is performed on a case-by-case
basis. In essence, the comparison is an ad hoc determination of whether
the protectable elements of the original program that are contained in the
second work are significant or important parts of the original program. If
so, then the second work is a derivative work of the first. If, however,
the amount of protectable elements copied in the second work are so small
as to be de minimis, then the second work is not a derivative work of the
original.

\section{Analytic Dissection Test}

The Ninth Circuit has adopted the analytic dissection test to determine
whether one program is a derivative work of another. Apple Computer,
Inc. v. Microsoft Corp., 35 F.3d 1435 (9th Cir. 1994). The analytic
dissection test first considers whether there are substantial similarities
in both the ideas and expressions of the two works at issue. Once the
similar features are identified, analytic dissection is used to determine
whether any of those similar features are protected by copyright. This
step is the same as the filtration step in the AFC test. After identifying
the copyrightable similar features of the works, the court then decides
whether those features are entitled to ``broad'' or ``thin''
protection. ``Thin'' protection is given to non-copyrightable facts or
ideas that are combined in a way that affords copyright protection only
from their alignment and presentation, while ``broad'' protection is given
to copyrightable expression itself. Depending on the degree of protection
afforded, the court then sets the appropriate standard for a subjective
comparison of the works to determine whether, as a whole, they are
sufficiently similar to support a finding that one is a derivative work of
the other. ``Thin'' protection requires the second work be virtually
identical in order to be held a derivative work of an original, while
``broad'' protection requires only a ``substantial similarity.''

\section{No Protection for ``Methods of Operation''}

The First Circuit expressly rejected the AFC test and, instead, takes a
much narrower view of the meaning of derivative work for software. The
First Circuit holds that ``method of operation,'' as used in \S 102(b) of
the Copyright Act, refers to the means by which users operate
computers. Lotus Development Corp. v. Borland Int’l., Inc., 49 F.3d 807
(1st Cir. 1995). More specifically, the court held that a menu command
hierarchy for a computer program was uncopyrightable because it did not
merely explain and present the program’s functional capabilities to the
user, but also served as a method by which the program was operated and
controlled. As a result, under the First Circuit’s test, literal copying
of a menu command hierarchy, or any other ``method of operation,'' cannot
form the basis for a determination that one work is a derivative of
another. It is also reasonable to expect that the First Circuit will read
the unprotectable elements set forth in \S 102(b) broadly, and, as such,
promulgate a definition of derivative work that is much narrower than that
which exists under the AFC test.

\section{No Test Yet Adopted}

Several circuits, most notably the Fourth and Seventh, have yet to
declare their definition of derivative work and whether or not the
AFC, Analytic Dissection, or some other test best fits their
interpretation of copyright law. Therefore, uncertainty exists with
respect to determining the extent to which a software program is a
derivative work of another in those circuits. However, one may presume
that they would give deference to the AFC test since it is by far the
majority rule amongst those circuits that have a standard for defining
a software derivative work.

\section{Cases Applying Software Derivative Work Analysis}

In the preeminent case regarding the definition of a derivative work for
software, Computer Associates v. Altai, the plaintiff alleged that its
program, Adapter, which was used to handle the differences in operating
system calls and services, was infringed by the defendant's competitive
program, Oscar. About 30\% of Oscar was literally the same code as
that in Adapter. After the suit began, the defendant rewrote those
portions of Oscar that contained Adapter code in order to produce a new
version of Oscar that was functionally competitive with Adapter, without
have any literal copies of its code. Feeling slighted still, the
plaintiff alleged that even the second version of Oscar, despite having no
literally copied code, also infringed its copyrights. In addressing that
question, the Second Circuit promulgated the AFC test.

In abstracting the various levels of the program, the court noted a
similarity between the two programs' parameter lists and macros. However,
following the filtration step of the AFC test, only a handful of the lists
and macros were protectable under copyright law because they were either
in the public domain or required by functional demands on the
program. With respect to the handful of parameter lists and macros that
did qualify for copyright protection, after performing the comparison step
of the AFC test, it was reasonable for the district court to conclude that
they did not warrant a finding of infringement given their relatively minor
contribution to the program as a whole. Likewise, the similarity between
the organizational charts of the two programs was not substantial enough
to support a finding of infringement because they were too simple and
obvious to contain any original expression.

Perhaps not surprisingly, there have been few cases involving a highly
detailed software derivative work analysis. Most often, cases involve
clearer basis for decision, including frequent bad faith on the part of
the defendant or overaggressiveness on the part of the plaintiff.
However, no cases involving Free Software licensing have ever gone to
court. As Free Software becomes an ever-increasingly important part of
the economy, it remains to be seen if battle lines will be
drawn over whether particular programs infringe the rights of Free
Software developers or whether the entire community, including industry,
adopts norms avoiding such risk.

%%%%%%%%%%%%%%%%%%%%%%%%%%%%%%%%%%%%%%%%%%%%%%%%%%%%%%%%%%%%%%%%%%%%%%%%%%%%%%%

\chapter{Modified Source and Binary Distribution}
\label{source-and-binary}

In this chapter, we discuss the two core sections that define the rights
and obligations for those who modify, improve, and/or redistribute GPL'd
software. These sections, \S\S 2--3, define the central core rights and
requirements of GPLv2\@.

\section{GPLv2 \S 2: Share and Share Alike}

For many, this is where the ``magic'' happens that defends software
freedom along the distribution chain. \S 2 is the only place in the GPL
that governs the modification controls of copyright law. If someone
modifies a GPL'd program, she is bound in the making those changes by \S
2. The goal here is to ensure that the body of GPL'd software, as it
continues and develops, remains Free as in freedom.

To achieve that goal, \S 2 first sets forth that the rights of
redistribution of modified versions are the same as those for verbatim
copying, as presented in \S 1. Therefore, the details of charging,
keeping copyright notices intact, and other \S 1 provisions are in tact
here as well. However, there are three additional requirements.

The first (\S 2(a)) requires that modified files carry ``prominent
notices'' explaining what changes were made and the date of such
changes. The goal here is not to put forward some specific way of
marking changes nor controlling the process of how changes get made.
Primarily, \S 2(a) seeks to ensure that those receiving modified
versions know the history of changes to the software. For some users,
it is important to know that they are using the standard version of
program, because while there are many advantages to using a fork,
there are a few disadvantages. Users should be informed about the
historical context of the software version they use, so that they can
make proper support choices. Finally, \S 2(a) serves an academic
purpose --- ensuring that future developers can use a diachronic
approach to understand the software.

\medskip

The second requirement (\S 2(b)) contains the four short lines that embody
the legal details of ``share and share alike.''  These 46 words are
considered by some to be the most worthy of careful scrutiny because \S
2(b) can be a source of great confusion when not properly understood.

In considering \S 2(b), first note the qualifier: it only applies to
derivative works that ``you distribute or publish.''  Despite years of
education efforts by FSF on this matter, many still believe that modifiers
of GPL'd software are required by the license to publish or otherwise
share their changes. On the contrary, \S 2(b) {\bf does not apply if} the
changes are never distributed. Indeed, the freedom to make private,
personal, unshared changes to software for personal use only should be
protected and defended.\footnote{FSF does maintain that there is an {\bf
    ethical} obligation to redistribute changes that are generally useful,
  and often encourages companies and individuals to do so. However, there
  is a clear distinction between what one {\bf ought} to do and what one
  {\bf must} do.}

Next, we again encounter the same matter that appears in \S 0, in the
following text:
\begin{quote}
``...that in whole or part contains or is derived from the Program or any part thereof.''
\end{quote}
Again, the GPL relies here on what the copyright law says is a derivative
work. If, under copyright law, the modified version ``contains or is
derived from'' the GPL'd software, then the requirements of \S 2(b)
apply. The GPL invokes its control as a copyright license over the
modification of the work in combination with its control over distribution
of the work.

The final clause of \S 2(b) describes what the licensee must do if she is
distributing or publishing a work that is deemed a derivative work under
copyright law --- namely, the following:
\begin{quote}
[The work must] be licensed as a whole at no charge to all third parties
under the terms of this License.
\end{quote}
That is probably the most tightly-packed phrase in all of the GPL\@.
Consider each subpart carefully.

The work ``as a whole'' is what is to be licensed. This is an important
point that \S 2 spends an entire paragraph explaining; thus this phrase is
worthy of a lengthy discussion here. As a programmer modifies a software
program, she generates new copyrighted material --- fixing expressions of
ideas into the tangible medium of electronic file storage. That
programmer is indeed the copyright holder of those new changes. However,
those changes are part and parcel to the original work distributed to
the programmer under GPL\@. Thus, the license of the original work
affects the license of the new whole derivative work.

% {\cal I}
\newcommand{\gplusi}{$\mathcal{G\!\!+\!\!I}$}
\newcommand{\worki}{$\mathcal{I}$}
\newcommand{\workg}{$\mathcal{G}$}

\label{separate-and-independent}

It is certainly possible to take an existing independent work (called
\worki{}) and combine it with a GPL'd program (called \workg{}). The
license of \worki{}, when it is distributed as a separate and independent
work, remains the prerogative of the copyright holder of \worki{}.
However, when \worki{} is combined with \workg{}, it produces a new work
that is the combination of the two (called \gplusi{}). The copyright of
this combined work, \gplusi{}, is held by the original copyright
holder of each of the two works.

In this case, \S 2 lays out the terms by which \gplusi{} may be
distributed and copied. By default, under copyright law, the copyright
holder of \worki{} would not have been permitted to distribute \gplusi{};
copyright law forbids it without the expressed permission of the copyright
holder of \workg{}. (Imagine, for a moment, if \workg{} were a Microsoft
product --- would they give you permission to create and distribute
\gplusi{} without paying them a hefty sum?)  The license of \workg{}, the
GPL, sets forth ahead of time options for the copyright holder of \worki{}
who may want to create and distribute \gplusi{}. This pregranted
permission to create and distribute derivative works, provided the terms
of GPL are upheld, goes far above and beyond the permissions that one
would get with a typical work not covered by a copyleft license. Thus, to
say that this restriction is any way unreasonable is simply ludicrous.

\medskip

The next phrase of note in \S 2(b) is ``licensed...at no charge.''
This is a source of great confusion to many. Not a month goes by that
FSF does not receive an email that claims to point out ``a
contradiction in GPL'' because \S 2 says that redistributors cannot
charge for modified versions of GPL'd software, but \S 1 says that
they can. The ``at no charge'' does not prohibit redistributors from
charging when performing the acts governed by copyright
law,\footnote{Recall that you could by default charge for any acts not
governed by copyright law, because the license controls are confined
by copyright.} but rather that they cannot charge a fee for the
\emph{license itself}. In other words, redistributors of (modified
and unmodified) GPL'd works may charge any amount they choose for
performing the modifications on contract or the act of transferring
the copy to the customer, but they may not charge a separate licensing
fee for the software.

\S 2(b) further states that the software must ``be licensed...to all
third parties.''  This too has led to some confusions, and feeds the
misconception mentioned earlier --- that all modified versions must made
available to the public at large. However, the text here does not say
that. Instead, it says that the licensing under terms of the GPL must
extend to anyone who might, through the distribution chain, receive a copy
of the software. Distribution to all third parties is not mandated here,
but \S 2(b) does require redistributors to license the derivative works in
a way that extends to all third parties who may ultimately receive a
copy of the software.

In summary, \S 2(b) says what terms under which the third parties must
receive this no-charge license. Namely, they receive it ``under the terms
of this License,'' the GPL. When an entity \emph{chooses} to redistribute
a derivative work of GPL'd software, the license of that whole derivative
work must be GPL and only GPL\@. In this manner, \S 2(b) dovetails nicely
with \S 6 (as discussed in Section~\ref{GPLs6} of this tutorial).

\medskip

The final paragraph of \S 2 is worth special mention. It is possible and
quite common to aggregate various software programs together on one
distribution medium. Computer manufacturers do this when they ship a
pre-installed hard drive, and GNU/Linux distribution vendors do this to
give a one-stop CD or URL for a complete operating system with necessary
applications. The GPL very clearly permits such ``mere aggregation'' with
programs under any license. Despite what you hear from its critics, the
GPL is nothing like a virus, not only because the GPL is good for you and
a virus is bad for you, but also because simple contact with a GPL'd
code-base does not impact the license of other programs. Actual effort
must be expended by a programmer to cause a work to fall under the terms
of the GPL. Redistributors are always welcome to simply ship GPL'd
software alongside proprietary software or other unrelated Free Software,
as long as the terms of GPL are adhered to for those packages that are
truly GPL'd.

\section{GPLv2 \S 3: Producing Binaries}
\label{GPL-Section-3}
% FIXME: need name of a novelist who writes very obscurely and obliquely.

Software is a strange beast when compared to other copyrightable works.
It is currently impossible to make a film or a book that can be truly
obscured. Ultimately, the full text of a novel, even one written by
William Faulkner, must presented to the reader as words in some
human-readable language so that they can enjoy the work. A film, even one
directed by David Lynch, must be perceptible by human eyes and ears to
have any value.

Software is not so. While the source code, the human-readable
representation of software is of keen interest to programmers, users and
programmers alike cannot make the proper use of software in that
human-readable form. Binary code --- the ones and zeros that the computer
can understand --- must be predicable and attainable for the software to
be fully useful. Without the binaries, be they in object or executable
form, the software serves only the didactic purposes of computer science.

Under copyright law, binary representations of the software are simply
derivative works of the source code. Applying a systematic process (i.e.,
``compilation'') to a work of source code yields binary code. The binary
code is now a new work of expression fixed in the tangible medium of
electronic file storage.

Therefore, for GPL'd software to be useful, the GPL, since it governs the
rules for creation of derivative works, must grant permission for the
generation of binaries. Furthermore, notwithstanding the relative
popularity of source-based GNU/Linux distributions like Gentoo, users find
it extremely convenient to receive distribution of binary software. Such
distribution is the redistribution of derivative works of the software's
source code. \S 3 addresses the matter of creation and distribution of
binary versions.

Under GPLv2\S 3, binary versions may be created and distributed under the
terms of \S\S 1--2, so all the material previously discussed applies
here. However, \S 3 must go a bit further. Access to the software's
source code is an incontestable prerequisite for the exercise of the
fundamental freedoms to modify and improve the software. Making even
the most trivial changes to a software program at the binary level is
effectively impossible. \S 3 must ensure that the binaries are never
distributed without the source code, so that these freedoms are passed
through the distribution chain.

\S 3 permits distribution of binaries, and then offers three options for
distribution of source code along with binaries. The most common and the
least complicated is the option given under \S 3(a).

\S 3(a) offers the option to directly accompany the source code alongside
the distribution of the binaries. This is by far the most convenient
option for most distributors, because it means that the source-code
provision obligations are fully completed at the time of binary
distribution (more on that later).

Under \S 3(a), the source code provided must be the ``corresponding source
code.''  Here ``corresponding'' primarily means that the source code
provided must be that code used to produce the binaries being distributed.
That source code must also be ``complete.''  A later paragraph of \S 3
explains in detail what is meant by ``complete.''  In essence, it is all
the material that a programmer of average skill would need to actually use
the source code to produce the binaries she has received. Complete source
is required so that, if the licensee chooses, she should be able to
exercise her freedoms to modify and redistribute changes. Without the
complete source, it would not be possible to make changes that were
actually directly derived from the version received.

Furthermore, GPLv2\S 3 is defending against a tactic that has in fact been
seen in FSF's GPL enforcement. Under GPL, if you pay a high price for
a copy of GPL'd binaries (which comes with corresponding source, of
course), you have the freedom to redistribute that work at any fee you
choose, or not at all. Sometimes, companies attempt a GPL-violating
cozenage whereby they produce very specialized binaries (perhaps for
an obscure architecture). They then give source code that does
correspond, but withhold the ``incantations'' and build plans they
used to make that source compile into the specialized binaries.
Therefore, \S 3 requires that the source code include ``meta-material'' like
scripts, interface definitions, and other material that is used to
``control compilation and installation'' of the binaries. In this
manner, those further down the distribution chain are assured that
they have the unabated freedom to build their own derivative works
from the sources provided.

FSF (as authors of GPL) realizes that software distribution comes in many
forms. Embedded manufacturers, for example, have the freedom to put
GPL'd software into their PDAs with very tight memory and space
constraints. In such cases, putting the source right alongside the
binaries on the machine itself might not be an option. While it is
recommended that this be the default way that people comply with GPL, the
GPL does provide options when such distribution is infeasible.

\S 3, therefore, allows source code to be provided on any physical
``medium customarily used for software interchange.''  By design, this
phrase covers a broad spectrum. At best, FSF can viably release a new GPL
every ten years or so. Thus, phrases like this must be adaptive to
changes in the technology. When GPL version 2 was first published in June
1991, distribution on magnetic tape was still common, and CD was
relatively new. Today, CD is the default, and for larger systems DVD-R is
gaining adoption. This language must adapt with changing technology.

Meanwhile, the binding created by the word ``customarily'' is key. Many
incorrectly believe that distributing binary on CD and source on the
Internet is acceptable. In the corporate world, it is indeed customary to
simply download CDs worth of data over a T1 or email large file
attachments. However, even today in the USA, many computer users with
CD-ROM drives are not connected to the Internet, and most people connected
to the Internet are connected via a 56K dial-up connection. Downloading
CDs full of data is not customary for them in the least. In some cities
in Africa, computers are becoming more common, but Internet connectivity
is still available only at a few centralized locations. Thus, the
``customs'' here must be normalized for a worldwide userbase. Simply
providing source on the Internet --- while it is a kind, friendly and
useful thing to do --- is not usually sufficient.

Note, however, a major exception to this rule, given by the last paragraph
of \S 3. \emph{If} distribution of the binary files is made only on the
Internet (i.e., ``from a designated place''), \emph{then} simply providing
the source code right alongside the binaries in the same place is
sufficient to comply with \S 3.

\medskip

As is shown above, Under \S 3(a), embedded manufacturers can put the
binaries on the device and ship the source code along on a CD\@. However,
sometimes this turns out to be too costly. Including a CD with every
device could prove too costly, and may practically (although not legally)
prohibit using GPL'd software. For this situation and others like it, \S
3(b) is available.

\S 3(b) allows a distributor of binaries to instead provide a written
offer for source code alongside those binaries. This is useful in two
specific ways. First, it may turn out that most users do not request the
source, and thus the cost of producing the CDs is saved --- a financial
and environmental windfall. In addition, along with a \S 3(b) compliant
offer for source, a binary distributor might choose to \emph{also} give a
URL for source code. Many who would otherwise need a CD with source might
turn out to have those coveted high bandwidth connections, and are able to
download the source instead --- again yielding environmental and financial
windfalls.

However, note that regardless of how many users prefer to get the
source online, \S 3(b) does place lasting long-term obligations on the
binary distributor. The binary distributor must be prepared to honor
that offer for source for three years and ship it out (just as they
would have had to do under \S 3(a)) at a moment's notice when they
receive such a request. There is real organizational cost here:
support engineers must be trained how to route source requests, and
source CD images for every release version for the last three years
must be kept on hand to burn such CDs quickly. The requests might not
even come from actual customers; the offer for source must be valid
for ``any third party.''

That phrase is another place where some get confused --- thinking again
that full public distribution of source is required. The offer for source
must be valid for ``any third party'' because of the freedoms of
redistribution granted by \S\S 1--2. A company may ship a binary image
and an offer for source to only one customer. However, under GPL, that
customer has the right to redistribute that software to the world if she
likes. When she does, that customer has an obligation to make sure that
those who receive the software from her can exercise their freedoms under
GPL --- including the freedom to modify, rebuild, and redistribute the
source code.

GPLv2\S 3(c) is created to save her some trouble, because by itself \S 3(b)
would unfairly favor large companies. \S 3(b) allows the
separation of the binary software from the key tool that people can use
to exercise their freedom. The GPL permits this separation because it is
good for redistributors, and those users who turn out not to need the
source. However, to ensure equal rights for all software users, anyone
along the distribution chain must have the right to get the source and
exercise those freedoms that require it.

Meanwhile, \S 3(b)'s compromise primarily benefits companies who
distribute binary software commercially. Without \S 3(c), that benefit
would be at the detriment of the companies' customers; the burden of
source code provision would be unfairly shifted to the companies'
customers. A customer, who had received binaries with a \S 3(b)-compliant
offer, would be required under GPLv2 (sans \S 3(c)) to acquire the source,
merely to give a copy of the software to a friend who needed it. \S 3(c)
reshifts this burden to entity who benefits from \S 3(b).

\S 3(c) allows those who undertake \emph{noncommercial} distribution to
simply pass along a \S 3(b)-compliant source code offer. The customer who
wishes to give a copy to her friend can now do so without provisioning the
source, as long as she gives that offer to her friend. By contrast, if
she wanted to go into business for herself selling CDs of that software,
she would have to acquire the source and either comply via \S 3(a), or
write her own \S 3(b)-compliant source offer.

This process is precisely the reason why a \S 3(b) source offer must be
valid for all third parties. At the time the offer is made, there is no
way of knowing who might end up noncommercially receiving a copy of the
software. Companies who choose to comply via \S 3(b) must thus be
prepared to honor all incoming source code requests. For this and the
many other additional necessary complications under \S\S 3(b--c), it is
only rarely a better option than complying via \S 3(a).

%%%%%%%%%%%%%%%%%%%%%%%%%%%%%%%%%%%%%%%%%%%%%%%%%%%%%%%%%%%%%%%%%%%%%%%%%%%%%%%
\chapter{GPL's Implied Patent Grant}

We digress again briefly from our section-by-section consideration of GPLv2
to consider the interaction between the terms of GPL and patent law. The
GPLv2, despite being silent with respect to patents, actually confers on its
licensees more rights to a licensor's patents than those licenses that
purport to address the issue. This is the case because patent law, under
the doctrine of implied license, gives to each distributee of a patented
article a license from the distributor to practice any patent claims owned
or held by the distributor that cover the distributed article. The
implied license also extends to any patent claims owned or held by the
distributor that cover ``reasonably contemplated uses'' of the patented
article. To quote the Federal Circuit Court of Appeals, the highest court
for patent cases other than the Supreme Court:

\begin{quotation}
Generally, when a seller sells a product without restriction, it in
effect promises the purchaser that in exchange for the price paid, it will
not interfere with the purchaser's full enjoyment of the product
purchased. The buyer has an implied license under any patents of the
seller that dominate the product or any uses of the product to which the
parties might reasonably contemplate the product will be put.
\end{quotation}
Hewlett-Packard Co. v. Repeat-O-Type Stencil Mfg. Corp., Inc., 123 F.3d
1445 (Fed. Cir. 1997).

Of course, Free Software is licensed, not sold, and there are indeed
restrictions placed on the licensee, but those differences are not likely
to prevent the application of the implied license doctrine to Free
Software, because software licensed under the GPL grants the licensee the
right to make, use, and sell the software, each of which are exclusive
rights of a patent holder. Therefore, although the GPLv2 does not expressly
grant the licensee the right to do those things under any patents the
licensor may have that cover the software or its reasonably contemplated
uses, by licensing the software under the GPLv2, the distributor impliedly
licenses those patents to the GPLv2 licensee with respect to the GPLv2'd
software.

An interesting issue regarding this implied patent license of GPLv2'd
software is what would be considered ``uses of the [software] to which
the parties might reasonably contemplate the product will be put.'' A
clever advocate may argue that the implied license granted by GPLv2 is
larger in scope than the express license in other Free Software
licenses with express patent grants, in that, the patent license
clause of many of those licenses are specifically limited to the
patent claims covered by the code as licensed by the patentee.

To the contrary, GPLv2's implied patent license grants the GPLv2 licensee a
patent license to do much more than just that because the GPLv2 licensee,
under the doctrine of implied patent license, is free to practice any
patent claims held by the licensor that cover ``reasonably contemplated
uses'' of the GPL'd code, which may very well include creation and
distribution of derivative works since the GPL's terms, under which the
patented code is distributed, expressly permits such activity.

Further supporting this result is the Federal Circuit's pronouncement that
the recipient of a patented article has, not only an implied license to
make, use, and sell the article, but also an implied patent license to
repair the article to enable it to function properly, Bottom Line Mgmt.,
Inc. v. Pan Man, Inc., 228 F.3d 1352 (Fed. Cir. 2000). Additionally, the
Federal Circuit extended that rule to include any future recipients of the
patented article, not just the direct recipient from the distributor.
This theory comports well with the idea of Free Software, whereby software
is distributed amongst many entities within the community for the purpose
of constant evolution and improvement. In this way, the law of implied
patent license used by the GPLv2 ensures that the community mutually
benefits from the licensing of patents to any single community member.

Note that simply because GPLv2'd software has an implied patent license does
not mean that any patents held by a distributor of GPLv2'd code become
worthless. To the contrary, the patents are still valid and enforceable
against either:

\begin{enumerate}
 \renewcommand{\theenumi}{\alph{enumi}}
 \renewcommand{\labelenumi}{\textup{(\theenumi)}}

\item any software other than that licensed under the GPLv2 by the patent
  holder, and

\item any party that does not comply with the GPLv2
with respect to the licensed software.
\end{enumerate}

\newcommand{\compB}{$\mathcal{B}$}
\newcommand{\compA}{$\mathcal{A}$}

For example, if Company \compA{} has a patent on advanced Web browsing, but
also licenses a Web browsing software program under the GPLv2, then it
cannot assert the patent against any party that takes a license to its
program under the GPLv2. However, if a party uses that program without
complying with the GPLv2, then Company \compA{} can assert, not just copyright
infringement claims against the non-GPLv2-compliant party, but also
infringement of the patent, because the implied patent license only
extends to use of the software in accordance with the GPLv2. Further, if
Company \compB{} distributes a competitive advanced Web browsing program,
Company \compA{} is free to assert its patent against any user or
distributor of that product. It is irrelevant whether Company \compB's
program is distributed under the GPLv2, as Company \compB{} can not grant
implied licenses to Company \compA's patent.

This result also reassures companies that they need not fear losing their
proprietary value in patents to competitors through the GPLv2 implied patent
license, as only those competitors who adopt and comply with the GPLv2's
terms can benefit from the implied patent license. To continue the
example above, Company \compB{} does not receive a free ride on Company
\compA's patent, as Company \compB{} has not licensed-in and then
redistributed Company A's advanced Web browser under the GPLv2. If Company
\compB{} does do that, however, Company \compA{} still has not lost
competitive advantage against Company \compB{}, as Company \compB{} must then,
when it re-distributes Company \compA's program, grant an implied license
to any of its patents that cover the program. Further, if Company \compB{}
relicenses an improved version of Company A's program, it must do so under
the GPLv2, meaning that any patents it holds that cover the improved version
are impliedly licensed to any licensee. As such, the only way Company
\compB{} can benefit from Company \compA's implied patent license, is if it,
itself, distributes Company \compA's software program and grants an
implied patent license to any of its patents that cover that program.

%%%%%%%%%%%%%%%%%%%%%%%%%%%%%%%%%%%%%%%%%%%%%%%%%%%%%%%%%%%%%%%%%%%%%%%%%%%%%%%
\chapter{Defending Freedom on Many Fronts}

Chapters~\ref{run-and-verbatim} and ~\ref{source-and-binary} presented the
core freedom-defending provisions of GPLv2\@, which are in \S\S 0--3. \S\S
4--7 of the GPLv2 are designed to ensure that \S\S 0--3 are not infringed,
are enforceable, are kept to the confines of copyright law, and are not
trumped by other copyright agreements or components of other entirely
separate legal systems. In short, while \S\S 0--3 are the parts of the
license that defend the freedoms of users and programmers, \S\S 4--7 are
the parts of the license that keep the playing field clear so that \S\S
0--3 can do their jobs.

\section{GPLv2 \S 4: Termination on Violation}
\label{GPLs4}

\S 4 is GPLv2's termination clause. Upon first examination, it seems
strange that a license with the goal of defending users' and programmers'
freedoms for perpetuity in an irrevocable way would have such a clause.
However, upon further examination, the difference between irrevocability
and this termination clause becomes clear.

The GPL is irrevocable in the sense that once a copyright holder grants
rights for someone to copy, modify and redistribute the software under
terms of the GPL, they cannot later revoke that grant. Since the GPL has
no provision allowing the copyright holder to take such a prerogative, the
license is granted as long as the copyright remains in effect.\footnote{In
  the USA, due to unfortunate legislation, the length of copyright is
  nearly perpetual, even though the Constitution forbids perpetual
  copyright.} The copyright holder has the right to relicense the same
work under different licenses (see Section~\ref{Proprietary Relicensing}
of this tutorial), or to stop distributing the GPLv2'd version (assuming \S
3(b) was never used), but she may not revoke the rights under GPLv2
already granted.

In fact, when an entity looses their right to copy, modify and distribute
GPL'd software, it is because of their \emph{own actions}, not that of
the copyright holder. The copyright holder does not decided when \S 4
termination occurs (if ever), the actions of the licensee does.

Under copyright law, the GPL has granted various rights and freedoms to
the licensee to perform specific types of copying, modification, and
redistribution. By default, all other types of copying, modification, and
redistribution are prohibited. \S 4 says that if you undertake any of
those other types (e.g., redistributing binary-only in violation of \S 3),
then all rights under the license --- even those otherwise permitted for
those who have not violated --- terminate automatically.

\S 4 gives GPLv2 teeth. If licensees fail to adhere to the license, then
they are stuck. They must completely cease and desist from all
copying, modification and distribution of that GPL'd software.

At that point, violating licensees must gain the forgiveness of the
copyright holder to have their rights restored. Alternatively, they could
negotiate another agreement, separate from GPL, with the copyright
holder. Both are common practice.

At FSF, it is part of the mission to spread software freedom. When FSF
enforces GPL, the goal is to bring the violator back into compliance as
quickly as possible, and redress the damage caused by the violation.
That is FSF's steadfast position in a violation negotiation --- comply
with the license and respect freedom.

However, other entities who do not share the full ethos of software
freedom as institutionalized by FSF pursue GPL violations differently.
MySQL AB, a company that produces the GPL'd MySQL database, upon
discovering GPL violations typically negotiates a proprietary software
license separately for a fee. While this practice is not one that FSF
would ever consider undertaking or even endorsing, it is a legal way for
copyright holders to proceed.

\section{GPLv2 \S 5: Acceptance, Copyright Style}
\label{GPLs5}

\S 5 brings us to perhaps the most fundamental misconception and common
confusion about GPLv2\@. Because of the prevalence of proprietary software,
most users, programmers, and lawyers alike tend to be more familiar with
EULAs. EULAs are believed by their authors to be contracts, requiring
formal agreement between the licensee and the software distributor to be
valid. This has led to mechanisms like ``shrink-wrap'' and ``click-wrap''
as mechanisms to perform acceptance ceremonies with EULAs.

The GPL does not need contract law to ``transfer rights.''  No rights
are transfered between parties. By contrast, the GPL is a permission
slip to undertake activities that would otherwise have been prohibited
by copyright law. As such, it needs no acceptance ceremony; the
licensee is not even required to accept the license.

However, without the GPL, the activities of copying, modifying and
distributing the software would have otherwise been prohibited. So, the
GPL says that you only accepted the license by undertaking activities that
you would have otherwise been prohibited without your license under GPL\@.
This is a certainly subtle point, and requires a mindset quite different
from the contractual approach taken by EULA authors.

An interesting side benefit to \S 5 is that the bulk of users of Free
Software are not required to accept the license. Undertaking fair and
unregulated use of the work, for example, does not bind you to the GPL,
since you are not engaging in activity that is otherwise controlled by
copyright law. Only when you engage in those activities that might have an
impact on the freedom of others does license acceptance occur, and the
terms begin to bind you to fair and equitable sharing of the software. In
other words, the GPL only kicks in when it needs to for the sake of
freedom.

\section{Using GPL Both as a Contract and Copyright License}

\section{GPLv2 \S 6: GPL, My One and Only}
\label{GPLs6}

A point that was glossed over in Section~\ref{GPLs4}'s discussion of \S 4
was the irrevocable nature of the GPL\@. The GPLv2 is indeed irrevocable,
and it is made so formally by \S 6.

The first sentence in \S 6 ensures that as software propagates down the
distribution chain, that each licensor can pass along the license to each
new licensee. Under \S 6, the act of distributing automatically grants a
license from the original licensor to the next recipient. This creates a
chain of grants that ensure that everyone in the distribution has rights
under the GPLv2\@. In a mathematical sense, this bounds the bottom ---
making sure that future licensees get no fewer rights than the licensee before.

The second sentence of \S 6 does the opposite; it bounds from the top. It
prohibits any licensor along the distribution chain from placing
additional restrictions on the user. In other words, no additional
requirements may trump the rights and freedoms given by GPLv2\@.

The final sentence of \S 6 makes it abundantly clear that no individual
entity in the distribution chain is responsible for the compliance of any
other. This is particularly important for noncommercial users who have
passed along a source offer under \S 3(c), as they cannot be assured that
the issuer of the offer will honor their \S 3 obligations.

In short, \S 6 says that your license for the software is your one and
only copyright license allowing you to copy, modify and distribute the
software.

\section{GPLv2 \S 7: ``Give Software Liberty or Give It Death!''}
\label{GPLs7}

In essence, \S 7 is a verbosely worded way of saying for non-copyright
systems what \S 6 says for copyright. If there exists any reason that a
distributor knows of that would prohibit later licensees from exercising
their full rights under GPL, then distribution is prohibited.

Originally, this was designed as the title of this section suggests --- as
a last ditch effort to make sure that freedom was upheld. However, in
modern times, it has come to give much more. Now that the body of GPL'd
software is so large, patent holders who would want to be distributors of
GPL'd software have a tough choice. They must choose between avoiding
distribution of GPL'd software that exercises the teachings of their
patents, or grant a royalty-free, irrevocable, non-exclusive license to
those patents. Many companies, including IBM, the largest patent holder
in the world, have chosen the latter.

Thus, \S 7 rarely gives software death by stopping its distribution.
Instead, it is inspiring patent holders to share their patents in the same
freedom-defending way that they share their copyrighted works.

\section{GPLv2 \S 8: Excluding Problematic Jurisdictions}
\label{GPLs8}

\S 8 is rarely used by copyright holders. Its intention is that if a
particular country, say Unfreedonia, grants particular patents or allows
copyrighted interfaces (no country to our knowledge even permits those
yet), that the GPLv2'd software can continue in free and unabated
distribution in the countries where such controls do not exist.

It is a partial ``out'' from \S 7. Without \S 8, if a copyright holder
knew of a patent in a particular country licensed in a GPL-incompatible
way, then she could not distribute under GPL, because the work could
legitimately end up in the hands of citizens of Unfreedonia.

It is an inevitable but sad reality that some countries are freer than
others. \S 8 exists to permit distribution in those countries that are
free without otherwise negating parts of the license.

%%%%%%%%%%%%%%%%%%%%%%%%%%%%%%%%%%%%%%%%%%%%%%%%%%%%%%%%%%%%%%%%%%%%%%%%%%%%%%%
\chapter{Odds, Ends, and Absolutely No Warranty}

\S 0--7 constitute the freedom-defending terms of the GPLv2. The remainder
of the GPLv2 handles administrivia and issues concerning warranties and
liability.

\section{GPLv2 \S 9: FSF as Stewards of GPL}
\label{GPLs9}

FSF reserves the exclusive right to publish future versions of the GPL\@;
\S 9 expresses this. While the stewardship of the copyrights on the body
of GPL'd software around the world is shared among thousands of
individuals and organizations, the license itself needs a single steward.
Forking of the code is often regrettable but basically innocuous. Forking
of licensing is disastrous.

FSF has only released two versions of GPL --- in 1989 and 1991. GPL
version 3 is under current internal drafting. FSF's plan is to have a
long and engaging comment period. The goal of GPL is to defend freedom, and
a gigantic community depends on that freedom now. FSF hopes to take all
stakeholders' opinions under advisement.

\section{GPLv2 \S 10: Relicensing Permitted}
\label{GPLs10}

\S 10 reminds the licensee of what is already implied by the nature of
copyright law. Namely, the copyright holder of a particular software
program has the prerogative to grant alternative agreements under separate
copyright licenses.

\section{GPLv2 \S 11: No Warranty}
\label{GPLs11}

All warranty disclaimer language tends to be shouted in all capital
letters. Apparently, there was once a case where the disclaimer language
of an agreement was negated because it was not ``conspicuous'' to one of
the parties. Therefore, to make such language ``conspicuous,'' people
started placing it in bold or capitalizing the entire text. It now seems
to be voodoo tradition of warranty disclaimer writing.

Some have argued the GPL is unenforceable in some jurisdictions because
its disclaimer of warranties is impermissibly broad. However, \S 11
contains a jurisdictional savings provision, which states that it is to be
interpreted only as broadly as allowed by applicable law. Such a
provision ensures that both it, and the entire GPL, is enforceable in any
jurisdiction, regardless of any particular law regarding the
permissibility of certain warranty disclaimers.

Finally, one important point to remember when reading \S 11 is that \S 1
permits the sale of warranty as an additional service, which \S 11 affirms.

\section{GPLv2 \S 12: Limitation of Liability}
\label{GPLs12}

There are many types of warranties, and in some jurisdictions some of them
cannot be disclaimed. Therefore, usually agreements will have both a
warranty disclaimer and a limitation of liability, as we have in \S 12. \S
11 thus gets rid of all implied warranties that can legally be
disavowed. \S 12, in turn, limits the liability of the actor for any
warranties that cannot legally be disclaimed in a particular jurisdiction.

Again, some have argued the GPL is unenforceable in some jurisdictions
because its limitation of liability is impermissibly broad. However, \S
12, just like its sister, \S 11, contains a jurisdictional savings
provision, which states that it is to be interpreted only as broadly as
allowed by applicable law. As stated above, such a provision ensures that
both \S 12, and the entire GPL, is enforceable in any jurisdiction,
regardless of any particular law regarding the permissibility of limiting
liability.

So end the terms and conditions of the GNU General Public License.

%%%%%%%%%%%%%%%%%%%%%%%%%%%%%%%%%%%%%%%%%%%%%%%%%%%%%%%%%%%%%%%%%%%%%%%%%%%%%%%
\chapter{GPLv3}

\section{Understanding GPLv3 As An Upgraded GPLv2}

\section{GPLv3 \S 0: Giving In On ``Defined Terms''}

\section{GPLv3 \S 1: Understanding CCS}

\section{GPLv3 \S 2: Basic Permissions}

\section{GPLv3 \S 3: What Hath DMCA Wrought}

\section{GPLv3 \S 4: Verbatim Copying}

\section{GPLv3 \S 5: Modified Source}

\section{GPLv3 \S 6: Non-Source and Corresponding Source}

\section{GPLv3 \S 7: Explicit Compatibility}

\section{GPLv3 \S 8: A Lighter Termination}

\section{GPLv3 \S 9: Acceptance}

\section{GPLv3 \S 10: Explicit Downstream License}

\section{GPLv3 \S 11: Explicit Patent Licensing}

\section{GPLv3 \S 12: Familiar as GPLv2 \S 7}

\section{GPLv3 \S 13: The Great Affero Compromise}

\section{GPLv3 \S 14: So, When's GPLv4?}

\section{GPLv3 \S 15--17: Warranty Disclaimers and Liability Limitation}


%%%%%%%%%%%%%%%%%%%%%%%%%%%%%%%%%%%%%%%%%%%%%%%%%%%%%%%%%%%%%%%%%%%%%%%%%%%%%%%
\chapter{The Lesser GPL}

As we have seen in our consideration of the GPL, its text is specifically
designed to cover all possible derivative works under copyright law. Our
goal in designing GPL was to make sure that any derivative work of GPL'd
software was itself released under GPL when distributed. Reaching as far
as copyright law will allow is the most direct way to reach that goal.

However, while the strategic goal is to bring as much Free Software
into the world as possible, particular tactical considerations
regarding software freedom dictate different means. Extending the
copyleft effect as far as copyright law allows is not always the most
prudent course in reaching the goal. In particular situations, even
those of us with the goal of building a world where all published
software is Free Software realize that full copyleft does not best
serve us. The GNU Lesser General Public License (``GNU LGPL'') was
designed as a solution for such situations.

\section{The First LGPL'd Program}

The first example that FSF encountered where such altered tactics were
needed was when work began on the GNU C Library. The GNU C Library would
become (and today, now is) a drop-in replacement for existing C libraries.
On a Unix-like operating system, C is the lingua franca and the C library
is an essential component for all programs. It is extremely difficult to
construct a program that will run with ease on a Unix-like operating
system without making use of services provided by the C library --- even
if the program is written in a language other than C\@. Effectively, all
user application programs that run on any modern Unix-like system must
make use of the C library.

By the time work began on the GNU implementation of the C libraries, there
were already many C libraries in existence from a variety of vendors.
Every proprietary Unix vendor had one, and many third parties produced
smaller versions for special purpose use. However, our goal was to create
a C library that would provide equivalent functionality to these other C
libraries on a Free Software operating system (which in fact happens today
on modern GNU/Linux systems, which all use the GNU C Library).

Unlike existing GNU application software, however, the licensing
implications of releasing the GNU C Library (``glibc'') under GPL were
somewhat different. Applications released under GPL would never
themselves become part of proprietary software. However, if glibc were
released under GPL, it would require that any application distributed for
the GNU/Linux platform be released under GPL\@.

Since all applications on a Unix-like system depend on the C library, it
means that they must link with that library to function on the system. In
other words, all applications running on a Unix-like system must be
combined with the C library to form a new whole derivative work that is
composed of the original application and the C library. Thus, if glibc
were GPL'd, each and every application distributed for use on GNU/Linux
would also need to be GPL'd, since to even function, such applications
would need to be combined into larger derivative works by linking with
glibc.

At first glance, such an outcome seems like a windfall for Free Software
advocates, since it stops all proprietary software development on
GNU/Linux systems. However, the outcome is a bit more subtle. In a world
where many C libraries already exist, many of which could easily be ported
to GNU/Linux, a GPL'd glibc would be unlikely to succeed. Proprietary
vendors would see the excellent opportunity to license their C libraries
to anyone who wished to write proprietary software for GNU/Linux systems.
The de-facto standard for the C library on GNU/Linux would likely be not
glibc, but the most popular proprietary one.

Meanwhile, the actual goal of releasing glibc under GPL --- to ensure no
proprietary applications on GNU/Linux --- would be unattainable in this
scenario. Furthermore, users of those proprietary applications would also
be users of a proprietary C library, not the Free glibc.

The Lesser GPL was initially conceived to handle this scenario. It was
clear that the existence of proprietary applications for GNU/Linux was
inevitable. Since there were so many C libraries already in existence, a
new one under GPL would not stop that tide. However, if the new C library
were released under a license that permitted proprietary applications
to link with it, but made sure that the library itself remained Free,
an ancillary goal could be met. Users of proprietary applications, while
they would not have the freedom to copy, share, modify and redistribute
the application itself, would have the freedom to do so with respect to
the C library.

There was no way the license of glibc could stop or even slow the creation
of proprietary applications on GNU/Linux. However, loosening the
restrictions on the licensing of glibc ensured that nearly all proprietary
applications at least used a Free C library rather than a proprietary one.
This trade-off is central to the reasoning behind the LGPL\@.

Of course, many people who use the LGPL today are not thinking in these
terms. In fact, they are often choosing the LGPL because they are looking
for a ``compromise'' between the GPL and the X11-style liberal licensing.
However, understanding FSF's reasoning behind the creation of the LGPL is
helpful when studying the license.


\section{What's the Same?}

Much of the text of the LGPL is identical to the GPL\@. As we begin our
discussion of the LGPL, we will first eliminate the sections that are
identical, or that have the minor modification changing the word
``Program'' to ``Library.''

First, \S 1 of LGPL, the rules for verbatim copying of source, are
equivalent to those in GPL's \S 1.

Second, \S 8 of LGPL is equivalent \S 4 of GPL\@. In both licenses, this
section handles termination in precisely the same manner.

\S 9 in LGPL is equivalent to \S 5 in GPL\@. Both sections assert that
the license is a copyright license, and handle the acceptance of those
copyright terms.

LGPL's \S 10 is equivalent to GPL's \S 6. They both protect the
distribution system of Free Software under these licenses, to ensure that
up, down, and throughout the distribution chain, each recipient of the
software receives identical rights under the license and no other
restrictions are imposed.

LGPL's \S 11 is GPL's \S 7. As discussed, it is used to ensure that
other claims and legal realities, such as patent licenses and court
judgments, do not trump the rights and permissions granted by these
licenses, and requires that distribution be halted if such a trump is
known to exist.

LGPL's \S 12 adds the same features as GPL's \S 8. These sections are
used to allow original copyright holders to forbid distribution in
countries with draconian laws that would otherwise contradict these
licenses.

LGPL's \S 13 sets up FSF as the steward of the LGPL, just as GPL's \S 9
does for GPL. Meanwhile, LGPL's \S 14 reminds licensees that copyright
holders can grant exceptions to the terms of LGPL, just as GPL's \S 10
reminds licensees of the same thing.

Finally, the assertions of no warranty and limitations of liability are
identical; thus LGPL's \S 15 and \S 16 are the same as GPL's \S 11 and \S
12.

As we see, the entire latter half of the license is identical.
The parts which set up the legal boundaries and meta-rules for the license
are the same. It is our intent that the two licenses operate under the
same legal mechanisms and are enforced precisely the same way.

We strike a difference only in the early portions of the license.
Namely, in the LGPL we go into deeper detail of granting various permissions to
create derivative works, so the redistributors can make
some proprietary derivatives. Since we simply do not allow the
license to stretch as far as copyright law does regarding what
derivative works must be relicensed under the same terms, we must go
further to explain which derivative works we will allow to be
proprietary. Thus, we'll see that the front matter of the LGPL is a
bit more wordy and detailed with regards to the permissions granted to
those who modify or redistribute the software.

\section{Additions to the Preamble}

Most of LGPL's Preamble is identical, but the last seven paragraphs
introduce the concepts and reasoning behind creation of the license,
presenting a more generalized and briefer version of the story with which
we began our consideration of LGPL\@.

In short, FSF designed LGPL for those edge cases where the freedom of the
public can better be served by a more lax licensing system. FSF doesn't
encourage use of LGPL automatically for any software that happens to be a
library; rather, FSF suggests that it only be used in specific cases, such
as the following:

\begin{itemize}

\item To encourage the widest possible use of a Free Software library, so
  it becomes a de-facto standard over similar, although not
  interface-identical, proprietary alternatives

\item To encourage use of a Free Software library that already has
  interface-identical proprietary competitors that are more developed

\item To allow a greater number of users to get freedom, by encouraging
  proprietary companies to pick a Free alternative for its otherwise
  proprietary products

\end{itemize}

LGPL's preamble sets forth the limits to which the license seeks to go in
chasing these goals. LGPL is designed to ensure that users who happen to
acquire software linked with such libraries have full freedoms with
respect to that library. They should have the ability to upgrade to a newer
or modified Free version or to make their own modifications, even if they
cannot modify the primary software program that links to that library.

Finally, the preamble introduces two terms used throughout the license to
clarify between the different types of derivative works: ``works that use
the library,'' and ``works based on the library.''  Unlike GPL, LGPL must
draw some lines regarding derivative works. We do this here in this
license because we specifically seek to liberalize the rights afforded to
those who make derivative works. In GPL, we reach as far as copyright law
allows. In LGPL, we want to draw a line that allows some derivative works
copyright law would otherwise prohibit if the copyright holder exercised
his full permitted controls over the work.

\section{An Application: A Work that Uses the Library}

In the effort to allow certain proprietary derivative works and prohibit
others, LGPL distinguishes between two classes of derivative works:
``works based on the library,'' and ``works that use the library.''  The
distinction is drawn on the bright line of binary (or runtime) derivative
works and source code derivatives. We will first consider the definition
of a ``work that uses the library,'' which is set forth in LGPL \S 5.

We noted in our discussion of GPL \S 3 (discussed in
Section~\ref{GPL-Section-3} of this document) that binary programs when
compiled and linked with GPL'd software are derivative works of that GPL'd
software. This includes both linking that happens at compile-time (when
the binary is created) or at runtime (when the binary -- including library
and main program both -- is loaded into memory by the user). In GPL,
binary derivative works are controlled by the terms of the license (in GPL
\S 3), and distributors of such binary derivatives must release full
corresponding source\@.

In the case of LGPL, these are precisely the types of derivative works
we wish to permit. This scenario, defined in LGPL as ``a work that uses
the library,'' works as follows:

\newcommand{\workl}{$\mathcal{L}$}
\newcommand{\lplusi}{$\mathcal{L\!\!+\!\!I}$}

\begin{itemize}

\item A new copyright holder creates a separate and independent work,
  \worki{}, that makes interface calls (e.g., function calls) to the
  LGPL'd work, called \workl{}, whose copyright is held by some other
  party. Note that since \worki{} and \workl{} are separate and
  independent works, there is no copyright obligation on this new copyright
  holder with regard to the licensing of \worki{}, at least with regard to
  the source code.

\item The new copyright holder, for her software to be useful, realizes
  that it cannot run without combining \worki{} and \workl{}.
  Specifically, when she creates a running binary program, that running
  binary must be a derivative work, called \lplusi{}, that the user can
  run.

\item Since \lplusi{} is a derivative work of both \worki{} and \workl{},
  the license of \workl{} (the LGPL) can put restrictions on the license
  of \lplusi{}. In fact, this is what LGPL does.

\end{itemize}

We will talk about the specific restrictions LGPL places on ``works
that use the library'' in detail in Section~\ref{lgpl-section-6}. For
now, focus on the logic related to how the LGPL places requirements on
the license of \lplusi{}. Note, first of all, the similarity between
this explanation and that in Section~\ref{separate-and-independent},
which discussed the combination of otherwise separate and independent
works with GPL'd code. Effectively, what LGPL does is say that when a
new work is otherwise separate and independent, but has interface
calls out to an LGPL'd library, then it is considered a ``work that
uses the library.''

In addition, the only reason that LGPL has any control over the licensing
of a ``work that uses the library'' is for the same reason that GPL has
some say over separate and independent works. Namely, such controls exist
because the {\em binary combination\/} (\lplusi{}) that must be created to
make the separate work (\worki{}) at all useful is a derivative work of
the LGPL'd software (\workl{}).

Thus, a two-question test that will help indicate if a particular work is
a ``work that uses the library'' under LGPL is as follows:

\begin{enumerate}

\item Is the source code of the new copyrighted work, \worki{}, a
  completely independent work that stands by itself, and includes no
  source code from \workl{}?

\item When the source code is compiled, does it create a derivative work
  by combining with \workl{}, either by static (compile-time) or dynamic
  (runtime) linking, to create a new binary work, \lplusi{}?
\end{enumerate}

If the answers to both questions are ``yes,'' then \worki{} is most likely
a ``work that uses the library.''  If the answer to the first question
``yes,'' but the answer to the second question is ``no,'' then most likely
\worki{} is neither a ``work that uses the library'' nor a ``work based on
the library.''  If the answer to the first question is ``no,'' but the
answer to the second question is ``yes,'' then an investigation into
whether or not \worki{} is in fact a ``work based on the library'' is
warranted.

\section{The Library, and Works Based On It}

In short, a ``work based on the library'' could be defined as any
derivative work of LGPL'd software that cannot otherwise fit the
definition of a ``work that uses the library.''  A ``work based on the
library'' extends the full width and depth of copyright derivative works,
in the same sense that GPL does.

Most typically, one creates a ``work based on the library'' by directly
modifying the source of the library. Such a work could also be created by
tightly integrating new software with the library. The lines are no doubt
fuzzy, just as they are with GPL'd works, since copyright law gives us no
litmus test for derivative works of a software program.

Thus, the test to use when considering whether something is a ``work
based on the library'' is as follows:

\begin{enumerate}

\item Is the new work, when in source form, a derivative work under
  copyright law of the LGPL'd work?

\item Is there no way in which the new work fits the definition of a
  ``work that uses the library''?
\end{enumerate}


If the answer is ``yes'' to both these questions, then you most likely
have a ``work based on the library.''  If the answer is ``no'' to the
first but ``yes'' to the second, you are in a gray area between ``work
based on the library'' and a ``work that uses the library.''

In our years of work with the LGPL, however, we have never seen a work
of software that was not clearly one or the other; the line is quite
bright. At times, though, we have seen cases where a derivative work
appeared in some ways to be a work that used the library and in other
ways a work based on the library. We overcame this problem by
dividing the work into smaller subunits. It was soon discovered that
what we actually had were three distinct components: the original
LGPL'd work, a specific set of works that used that library, and a
specific set of works that were based on the library. Once such
distinctions are established, the licensing for each component can be
considered independently and the LGPL applied to each work as
prescribed.


\section{Subtleties in Defining the Application}

In our discussion of the definition of ``works that use the library,'' we
left out a few more complex details that relate to lower-level programming
details. The fourth paragraph of LGPL's \S 5 covers these complexities,
and it has been a source of great confusion. Part of the confusion comes
because a deep understanding of how compiler programs work is nearly
mandatory to grasp the subtle nature of what \S 5, \P 4 seeks to
cover. It helps some to note that this is a border case that we cover in
the license only so that when such a border case is hit, the implications
of using LGPL continue in the expected way.

To understand this subtle point, we must recall the way that a compiler
operates. The compiler first generates object code, which are the binary
representations of various programming modules. Each of those modules is
usually not useful by itself; it becomes useful to a user of a full program
when those modules are {\em linked\/} into a full binary executable.

As we have discussed, the assembly of modules can happen at compile-time
or at runtime. Legally, there is no distinction between the two --- both
create a derivative work by copying and combining portions of one work and
mixing them with another. However, under LGPL, there is a case in the
compilation process where the legal implications are different.
Specifically, while we know that a ``work that uses the library'' is one
whose final binary is a derivative work, but whose source is not, there
are cases where the object code --- that intermediate step between source
and final binary --- is a derivative work created by copying verbatim code
from the LGPL'd software.

For efficiency, when a compiler turns source code into object code, it
sometimes places literal portions of the copyrighted library code into the
object code for an otherwise separate independent work. In the normal
scenario, the derivative would not be created until final assembly and
linking of the executable occurred. However, when the compiler does this
efficiency optimization, at the intermediate object code step, a
derivative work is created.

LGPL's \S 5, \P 4 is designed to handle this specific case. The intent of
the license is clearly that simply compiling software to ``make use'' of
the library does not in itself cause the compiled work to be a ``work
based on the library.''  However, since the compiler copies verbatim,
copyrighted portions of the library into the object code for the otherwise
separate and independent work, it would actually cause that object file to be a
``work based on the library.''  It is not FSF's intent that a mere
compilation idiosyncrasy would change the requirements on the users of the
LGPL'd software. This paragraph removes that restriction, allowing the
implications of the license to be the same regardless of the specific
mechanisms the compiler uses underneath to create the ``work that uses the
library.''

As it turns out, we have only once had anyone worry about this specific
idiosyncrasy, because that particular vendor wanted to ship object code
(rather than final binaries) to their customers and was worried about
this edge condition. The intent of clarifying this edge condition is
primarily to quell the worries of software engineers who understand the
level of verbatim code copying that a compiler often does, and to help
them understand that the full implications of LGPL are the same regardless
of the details of the compilation progress.

\section{LGPLv2 \S 6 \& LGPLv3 \S 5: Combining the Works}
\label{lgpl-section-6}
Now that we have established a good working definition of works that
``use'' and works that ``are based on'' the library, we will consider the
rules for distributing these two different works.

The rules for distributing ``works that use the library'' are covered in
\S 6 of LGPL\@. \S 6 is much like GPL's \S 3, as it requires the release
of source when a binary version of the LGPL'd software is released. Of
course, it only requires that source code for the library itself be made
available. The work that ``uses'' the library need not be provided in
source form. However, there are also conditions in LGPL \S 6 to make sure
that a user who wishes to modify or update the library can do so.

LGPL \S 6 lists five choices with regard to supplying library source
and granting the freedom to modify that library source to users. We
will first consider the option given by \S 6(b), which describes the
most common way currently used for LGPL compliance on a ``work that
uses the library.''

\S 6(b) allows the distributor of a ``work that uses the library'' to
simply use a dynamically linked, shared library mechanism to link with the
library. This is by far the easiest and most straightforward option for
distribution. In this case, the executable of the work that uses the
library will contain only the ``stub code'' that is put in place by the
shared library mechanism, and at runtime the executable will combine with
the shared version of the library already resident on the user's computer.
If such a mechanism is used, it must allow the user to upgrade and
replace the library with interface-compatible versions and still be able
to use the ``work that uses the library.''  However, all modern shared
library mechanisms function as such, and thus \S 6(b) is the simplest
option, since it does not even require that the distributor of the ``work
based on the library'' ship copies of the library itself.

\S 6(a) is the option to use when, for some reason, a shared library
mechanism cannot be used. It requires that the source for the library be
included, in the typical GPL fashion, but it also has a requirement beyond
that. The user must be able to exercise her freedom to modify the library
to its fullest extent, and that means recombining it with the ``work based
on the library.''  If the full binary is linked without a shared library
mechanism, the user must have available the object code for the ``work
based on the library,'' so that the user can relink the application and
build a new binary.

The remaining options in \S 6 are very similar to the other choices
provided by GPL \S 3. There are some additional options, but time does
not permit us in this course to go into those additional options. In
almost all cases of distribution under LGPL, either \S 6(a) or \S 6(b) are
exercised.

\section{Distribution of the Combined Works}

Essentially, ``works based on the library'' must be distributed under the
same conditions as works under full GPL\@. In fact, we note that LGPL's
\S 2 is nearly identical in its terms and requirements to GPL's \S 2.
There are again subtle differences and additions, which time does not
permit us to cover in this course.

\section{And the Rest}

The remaining variations between LGPL and GPL cover the following
conditions:

\begin{itemize}

\item Allowing a licensing ``upgrade'' from LGPL to GPL\@ (in LGPL \S 3)

\item Binary distribution of the library only, covered in LGPL \S 4,
  which is effectively equivalent to LGPL \S 3

\item Creating aggregates of libraries that are not derivative works of
  each other, and distributing them as a unit (in LGPL \S 7)

\end{itemize}


Due to time constraints, we cannot cover these additional terms in detail,
but they are mostly straightforward. The key to understanding LGPL is
understanding the difference between a ``work based on the library'' and a
``work that uses the library.''  Once that distinction is clear, the
remainder of LGPL is close enough to GPL that the concepts discussed in
our more extensive GPL unit can be directly applied.

%%%%%%%%%%%%%%%%%%%%%%%%%%%%%%%%%%%%%%%%%%%%%%%%%%%%%%%%%%%%%%%%%%%%%%%%%%%%%%%
\chapter{Integrating the GPL into Business Practices}

Since GPL'd software is now extremely prevalent through the industry, it
is useful to have some basic knowledge about using GPL'd software in
business and how to build business models around GPL'd software.

\section{Using GPL'd Software In-House}

As discussed in Sections~\ref{GPLs0} and~\ref{GPLs5} of this tutorial,
the GPL only governs the activities of copying, modifying and
distributing software programs that are not governed by the license.
Thus, in FSF's view, simply installing the software on a machine and
using it is not controlled or limited in any way by GPL\@. Using Free
Software in general requires substantially fewer agreements and less
license compliance activity than any known proprietary software.

Even if a company engages heavily in copying the software throughout the
enterprise, such copying is not only permitted by \S\S 1 and 3, but it is
encouraged!  If the company simply deploys unmodified (or even modified)
Free Software throughout the organization for its employees to use, the
obligations under the license are very minimal. Using Free Software has a
substantially lower cost of ownership --- both in licensing fees and in
licensing checking and handling -- than the proprietary software
equivalents.

\section{Business Models}
\label{Business Models}

Using Free Software in house is certainly helpful, but a thriving
market for Free Software-oriented business models also exists. There is the
traditional model of selling copies of Free Software distributions.
Many companies, including IBM and Red Hat, make substantial revenue
from this model. IBM primarily chooses this model because they have
found that for higher-end hardware, the cost of the profit made from
proprietary software licensing fees is negligible. The real profit is
in the hardware, but it is essential that software be stable, reliable
and dependable, and the users be allowed to have unfettered access to
it. Free Software, and GPL'd software in particular (because IBM can
be assured that proprietary versions of the same software will not
exists to compete on their hardware) is the right choice.

Red Hat has actually found that a ``convenience fee'' for Free Software,
when set at a reasonable price (around \$60 or so), can produce some
profit. Even though Red Hat's system is fully downloadable on their
Web site, people still go to local computer stores and buy copies of their
box set, which is simply a printed version of the manual (available under
a Free license as well) and the Free Software system it documents.

\medskip

However, custom support, service, and software improvement contracts
are the most widely used models for GPL'd software. The GPL is
central to their success, because it ensures that the code base
remains common, and that large and small companies are on equal
footing for access to the technology. Consider, for example, the GNU
Compiler Collection (GCC). Cygnus Solutions, a company started in the
early 1990s, was able to grow steadily simply by providing services
for GCC --- mostly consisting of new ports of GCC to different or new,
embedded targets. Eventually, Cygnus was so successful that
it was purchased by Red Hat where it remains a profitable division.

However, there are very small companies like CodeSourcery, as well as
other medium-sized companies like MontaVista and OpenTV that compete in
this space. Because the code-base is protect by GPL, it creates and
demands industry trust. Companies can cooperate on the software and
improve it for everyone. Meanwhile, companies who rely on GCC for their
work are happy to pay for improvements, and for ports to new target
platforms. Nearly all the changes fold back into the standard
versions, and those forks that exist remain freely available.

\medskip

\label{Proprietary Relicensing}

A final common business model that is perhaps the most controversial is
proprietary relicensing of a GPL'd code base. This is only an option for
software in which a particular entity is the sole copyright holder. As
discussed earlier in this tutorial, a copyright holder is permitted under
copyright law to license a software system under her copyright as many
different ways as she likes to as many different parties as she wishes.

Some companies, such as MySQL AB and TrollTech, use this to their
financial advantage with regard to a GPL'd code base. The standard
version is available from the company under the terms of the GPL\@.
However, parties can purchase separate proprietary software licensing for
a fee.

This business model is problematic because it means that the GPL'd code
base must be developed in a somewhat monolithic way, because volunteer
Free Software developers may be reluctant to assign their copyrights to
the company because it will not promise to always and forever license the
software as Free Software. Indeed, the company will surely use such code
contributions in proprietary versions licensed for fees.

\section{Ongoing Compliance}

GPL compliance is in fact a very simple matter -- much simpler than
typical proprietary software agreements and EULAs. Usually, the most
difficult hurdle is changing from a proprietary software mindset to one
that seeks to foster a community of sharing and mutual support. Certainly
complying with the GPL from a users' perspective gives substantially fewer
headaches than proprietary license compliance.

For those who go into the business of distributing {\em modified\\}
versions of GPL'd software, the burden is a bit higher, but not by
much. The glib answer is that by releasing the whole product as Free
Software, it is always easy to comply with the GPL. However,
admittedly to the dismay of FSF, many modern and complex software
systems are built using both proprietary and GPL'd components that are
not legally derivative works of each other. Sometimes, it is easier simply to
improve existing GPL'd application than to start from scratch. In
exchange for that benefit, the license requires that the modifier give
back to the commons that made the work easier in the first place. It is a
reasonable trade-off and a way to help build a better world while also
making a profit.

Note that FSF does provide services to assist companies who need
assistance in complying with the GPL. You can contact FSF's GPL
Compliance Labs at $<$compliance@fsf.org$>$.

If you are particularly interested in matters of GPL compliance, we
recommend the second course in this series, {\em GPL Compliance Case
  Studies and Legal Ethics in Free Software Licensing\/}, in which we
discuss some real GPL violation cases that FSF has worked to resolve.
Consideration of such cases can help give insight on how to handle GPL
compliance in new situations.


% =====================================================================
% END OF FIRST DAY SEMINAR SECTION
% =====================================================================


% compliance-guide.tex                            -*- LaTeX -*-

\part{A Practical Guide to GPL Compliance}
\label{gpl-compliance-guide}

{\parindent 0in
This part is: \\
\begin{tabbing}
Copyright \= \copyright{} 2008, 2014 \= \hspace{.2in} Bradley M. Kuhn. \\
Copyright \= \copyright{} 2014 \> \hspace{.2in} Free Software Foundation, Inc. \\
Copyright \> \copyright{} 2008, 2014 \> \hspace{.2in} Software Freedom Law Center. \\
\end{tabbing}

\vspace{1in}

\begin{center}
Authors of this part are: \\

Bradley M. Kuhn \\
Aaron Williamson \\
Karen M. Sandler \\

\vspace{1in}

Copy editors of this part include: \\
Martin Michlmayr

\vspace{3in}

The copyright holders of this part hereby grant the freedom to copy, modify,
convey, Adapt, and/or redistribute this work under the terms of the Creative
Commons Attribution Share Alike 4.0 International License.  A copy of that
license is available at
\url{https://creativecommons.org/licenses/by-sa/4.0/legalcode}.
\end{center}
}

\bigskip

\chapter*{Executive Summary}

This is a guide to effective compliance with the GNU General Public
License (GPL) and related licenses.  Copyleft advocates
usually seek to assist the community with
GPL compliance cooperatively.   This guide focuses on complying from the
start, so that readers can learn to avoid enforcement actions entirely, or, at
least, minimize  the negative impact when enforcement actions occur.
This guide  introduces and explains basic legal concepts related to the GPL and its
enforcement by copyright holders. It also outlines business practices and
methods that lead to better GPL compliance.  Finally, it recommends proper
post-violation responses to the concerns of copyright holders.

\chapter{Background}

Copyright law grants exclusive rights to authors.  Authors who chose copyleft
seek to protect the freedom of users and developers to copy, share, modify
and redistribute the software.  However, copyleft is ultimately implemented
through copyright, and the GPL is primarily and by default a copyright
license.  (See \S~\ref{explaining-copyright} for more about the interaction
between copyright and copyleft.)  Copyright law grants an unnatural exclusive
control to copyright holders regarding copyright-controlled permissions
related to the work.  Therefore, copyright holders (or their agents) are the
ultimately the sole authorities to enforce copyleft and protect the rights of
users.  Actions for copyright infringement are the ultimate legal mechanism
for enforcement.  Therefore, copyright holders, or collaborative groups of
copyright holders, have historically been the actors in GPL enforcement.

The earliest of these efforts began soon after the GPL was written by
Richard M.~Stallman (RMS) in 1989, and consisted of informal community efforts,
often in public Usenet discussions.\footnote{One example is the public
  outcry over NeXT's attempt to make the Objective-C front-end to GCC
  proprietary.  RMS, in fact, handled this enforcement action personally and
  the Objective-C front-end is still part of upstream GCC today.}  Over the next decade, the Free Software Foundation (FSF),
which holds copyrights in many GNU programs, was the only visible entity
actively enforcing its GPL'd copyrights on behalf of the software freedom
community.
FSF's enforcement
was generally a private process; the FSF contacted violators
confidentially and helped them to comply with the license.  Most
violations were pursued this way until the early 2000's.

By that time, Linux-based systems such as GNU/Linux and BusyBox/Linux had become very common, particularly in
embedded devices such as wireless routers.  During this period, public
ridicule of violators in the press and on Internet fora supplemented
ongoing private enforcement and increased pressure on businesses to
comply.  In 2003, the FSF formalized its efforts into the GPL Compliance
Lab, increased the volume of enforcement, and built community coalitions
to encourage copyright holders to together settle amicably with violators.
Beginning in 2004, Harald Welte took a more organized public enforcement
approach and launched \href{http://gpl-violations.org/}{gpl-violations.org}, a website and mailing
list for collecting reports of GPL violations.  On the basis of these
reports, Welte successfully pursued many enforcement actions in Europe, including
formal legal action.  Harald earns the permanent fame as the first copyright
holder to bring legal action in a court regarding GPL compliance.

In 2007, two copyright holders in BusyBox, in conjunction with the
Software Freedom Conservancy (``Conservancy''), filed the first copyright infringement lawsuit
based on a violation of the GPL\@ in the USA. While  lawsuits are of course
quite public, the vast majority of Conservancy's enforcement actions 
are resolved privately via
cooperative communications with violators.  As both FSF and Conservancy have worked to bring
individual companies into compliance, both organizations have encountered numerous
violations resulting from preventable problems such as inadequate
attention to licensing of upstream software, misconceptions about the
GPL's terms, and poor communication between software developers and their
management.  This document highlights these problems and describe
best practices to encourage corporate Free Software users to reevaluate their
approach to GPL'd software and avoid future violations.

Both FSF and Conservancy continue GPL enforcement and compliance efforts
for software under the GPL, the GNU Lesser
Public License (LGPL) and other copyleft licenses.  In doing so, both organizations have
found that most violations stem from a few common, avoidable mistakes.  All copyleft advocates  hope to educate the community of
commercial distributors, redistributors, and resellers on how to avoid
violations in the first place, and to respond adequately and appropriately
when a violation occurs.

\section{Who Has Compliance Obligations?}

All distributors of modified or unmodified versions of copylefted works
unmodified versions of the works have compliance obligations.  Common methods
of modifying the works include innumerable common acts, such as:

\begin{itemize}

  \item embedding those works as executable copies
    into a device,

  \item transferring a digital copy of executable copies to someone else,

  \item posting a patch to the copylefted software to a public mailing list.

\end{itemize}

Such distributors have obligations to (at least) the users to whom they (or
intermediary parties) distribute those copies.  In some cases, distributors
have obligations to third parties not directly receiving their distribution
of the works (depending on the distributors chosen licensing options, as
described later in \S~\ref{binary-distribution-permission}).  In addition,
distributors have compliance obligations to upstream parties, such as
preservation of reasonable legal notices embedded in the code, and
appropriate labeling of modified versions.

Online service providers and distributors alike have other compliance
obligations.  In general, they must refrain from imposing any additional
restrictions on downstream parties. Most typically, such compliance problems
arise from ``umbrella licenses:'' EULAs, or sublicenses that restrict
downstream users' rights under copyleft. (See \S~\ref{GPLv2s6} and
\S~\ref{GPLv3s10}).

Patent holders having claims reading on GPL'd works they distribute must
refrain from enforcing those claims against parties to whom they distribute.
Furthermore, patent holders holding copyrights on GPLv3'd works must further
grant an explicit patent license for any patent claims reading on the version
they distributed, and therefore cannot enforce those specific patent claims
against anyone making, using or selling a work based on their distributed
version.  All parties must refrain from acting as a provider of services or
distributor of licensed works if they have accepted, or had imposed on them
by judicial action, any legal conditions that would prevent them from meeting
any obligation under GPL\@.  (See \S~\ref{GPLv2s7}, \S~\ref{GPLv3s11} and
\S~\ref{GPLv3s12}.

\section{What Are The Risks of Non-Compliance?}

Copyleft experts have for decades observed a significant mismatch between the
assumptions most businesses make about copyleft compliance and the realities.
Possibly due to excessive marketing of proprietary tools and services from
the for-profit compliance industry, businesses perennially focus on the wrong
concerns.  This tutorial seeks to educate those businesses about what
actually goes wrong, what causes disputes, and how to resolve those disputes.

Many businesses currently invest undue resources to avoid unlikely risks that
have low historical incidence of occurrence and low cost of remediation,
while leaving unmanaged the risks that have historically resulted in all the
litigation and other adverse outcomes.  For example, some ``compliance
industry''\footnote{``Compliance industry'' refers to third-party for-profit
  companies that market proprietary software tools and/or consulting services
  that purport to aid businesses with their Free Software license compliance
  obligations, such as those found in GPL and other copyleft licenses.  This
  tutorial leaves the term in quotes throughout, primarily to communicate the
  skepticism most of this tutorial's authors feel regarding the mere
  existence of this industry.  Not only do copyleft advocates object on
  principle to proprietary software tools in general, and to their ironic use
  specifically to comply with copyleft, but also to the ``compliance
  industry'' vendors' marketing messaging, which some copyleft advocates
  claim as a cause in the risk misassessments discussed herein.  Bradley
  M.~Kuhn, specifically, regularly uses the term ``compliance industrial
  complex''
  \href{http://en.wikipedia.org/wiki/Military-industrial_complex}{to
    analogize the types of problems in this industry to those warned against
    in the phrase of origin}.} vendors insist that great effort must be
expended to carefully list, in the menus or manuals of embedded electronics
products, copyright notices for every last copyright holder that contributed
to the Free Software included in the product.  While nearly all Free Software
licenses, including copylefts like GPL, require preservation and display of
copyright notices, failure to meet this specific requirement is trivially
remedied.  Therefore, businesses should spend just reasonable efforts to
properly display copyright notices, and note that failure to do so is simply
remedied: add the missing copyright notice!

\section{Understanding Who's Enforcing}
\label{compliance-understanding-whos-enforcing}

The mismatch between actual compliance risk and compliance risk management
typically results from a misunderstanding of licensor intentions.  For-profit
businesses often err by assuming other actors have kindred motivations.  The
primary enforcers of the GPL, however, have goals that for-profit businesses
will find strange and perhaps downright alien.

Specifically, community-oriented GPL enforcement organizations (called
``COGEOs'' throughout the remainder of this tutorial) are typically
non-profit charities (such as the FSF and Software Freedom Conservancy) who
declare, as part of their charitable mission, advancement of software freedom
for all users.  In the USA, these COGEOs are all classified as charitable
under the IRS's 501(c)(3) designation, which is reserved for organizations
that have a mission to enhance the public good.

As such, these COGEOs enforce GPL primarily to pursue the policy goals and
motivations discussed throughout this tutorial: to spread software freedom
further.  As such, COGEOs are unified in their primary goal to bring the
violator back into compliance as quickly as possible, and redress the damage
caused by the violation.  COGEOs are steadfast in their position in a
violation negotiation: comply with the license and respect freedom.

Certainly, other entities do not share the full ethos of software freedom as
institutionalized by COGEOs, and those entities pursue GPL violations
differently.  Oracle, a company that produces the GPL'd MySQL database, upon
discovering GPL violations typically negotiates a proprietary software
license separately for a fee.  While this practice is not one a COGEO would
undertaking nor endorsing, a copyleft license technically permits this
behavior.  To put a finer point on this practice already discussed
in~\S~\ref{Proprietary Relicensing}, copyleft advocates usually find copyleft
enforcement efforts focused on extract alternative proprietary licenses
distasteful at best, and a corrupt manipulation of copyleft at worst.  Much
to the advocates' chagrin, such for-profit enforcement efforts seem to
increase rather than decrease.

Thus, unsurprisingly, for-profit adopters of GPL'd software often incorrectly
assume that all copyright holders seek royalties.  Businesses therefore focus
on the risk of so-called ``accidental'' (typically as the result of
unsupervised activity by individual programmers) infringe copyright by
incorporating ``snippets'' of copylefted code into their own proprietary
computer program.  ``Compliance industry'' flagship products, therefore,
focus on ``code scanning'' services that purport to detect accidental
inclusions.  Such effort focuses on proprietary software development and view
Free Software as a foreign interloper.  Such approach not only ignores
current reality that many companies build their products directly on major
copylefted projects (e.g., Android vendor's use of the kernel named Linux),
but also creates a culture of fear among developers, leading them into a
downward spiral of further hiding their necessary reliance on copylefted
software in the company's products.

Fortunately, COGEOs regard GPL compliance failures as an opportunity to
improve compliance.  Every compliance failure downstream represents a loss of
rights by their users. The COGEOs are the guardian of its users' and
developers' rights.  Their activity seeks to restore those rights, and
to protect the project's contributors' intentions in the making of their
software. 

\chapter{Best Practices to Avoid Common Violations}
\label{best-practices}

Unlike highly permissive licenses (such as the ISC license), which
typically only require preservation of copyright notices, licensees face many
important requirements from the GPL.  These requirements are
carefully designed to uphold certain values and standards of the software
freedom community.  While the GPL's requirements may initially appear
counter-intuitive to those more familiar with proprietary software
licenses, by comparison, its terms are in fact clear and quite favorable to
licensees.  Indeed, the GPL's terms actually simplify compliance when
violations occur.

GPL violations occur (or, are compounded) most often when companies lack sound
practices for the incorporation of GPL'd components into their
internal development environment.  This section introduces some best
practices for software tool selection, integration and distribution,
inspired by and congruent with software freedom methodologies.  Companies should
establish such practices before building a product based on GPL'd
software.\footnote{This document addresses compliance with GPLv2,
  GPLv3, LGPLv2, and LGPLv3.  Advice on avoiding the most common
  errors differs little for compliance with these four licenses.
  \S~\ref{lgpl} discusses the key differences between GPL and LGPL
  compliance.}

\section{Evaluate License Applicability}
\label{derivative-works}
Political discussion about the GPL often centers around determining the
``work'' that must be licensed under GPL, or in other words, ``what is the
derivative and/or combined work that was created''.  Nearly ever esoteric
question asked by lawyers seek to consider that question
\footnote{\tutorialpartsplit{In fact, a companion work, \textit{Detailed Analysis of the GNU GPL and Related
      Licenses} contains an entire section discussing derivative works}{This tutorial in fact
  also addresses the issue at length in~\S~\ref{derivative-works}}.} (perhaps because
that question explores exciting legal issues while the majority of the GPL
deals with much more mundane ones).
Of course, GPL was designed
primarily to embody the licensing feature of copyleft.

However, most companies who add
complex features to and make combinations with GPL'd software
are already well aware of their
more complex obligations under the license that require complex legal
analysis.  And, there are few companies overall that engage in such
activities. Thus,  in practical reality, this issue is not relevant to the vast
majority of companies distributing GPL'd software.

Thus, experienced  GPL enforcers find that few redistributors'
compliance challenges relate directly to combined work issues in copyleft.
Instead, the distributions of GPL'd
systems most often encountered typically consist of a full operating system
including components under the GPL (e.g., Linux, BusyBox) and components
under the LGPL (e.g., the GNU C Library).  Sometimes, these programs have
been patched or slightly improved by direct modification of their sources,
and thus the result is unequivocally a modified version.  Alongside these programs,
companies often distribute fully independent, proprietary programs,
developed from scratch, which are designed to run on the Free Software operating
system but do not combine with, link to, modify, derive from, or otherwise
create a combined work with
the GPL'd components.\footnote{However, these programs do often combine
  with LGPL'd libraries. This is discussed in detail in \S~\ref{lgpl}.}
In the latter case, where the work is unquestionably a separate work of
creative expression, no copyleft provisions are invoked.
The core compliance issue faced, thus, in such a situation, is not an discussion of what is or is not a
combined, derivative, and/or modified version of the work, but rather, issues related to distribution and
conveyance of binary works based on GPL'd source, but without Complete,
Corresponding Source.

As such, issues of software delivery are the primary frustration for GPL
enforcers. In particular, the following short list accounts for at least 95\%
of the GPL violations ever encountered:

\begin{itemize}

\item The violator fails to provide required information about the presence
  of copylefted programs and their applicable license terms in the product
  they have purchased.

\item The violator fails to reliably deliver \hyperref[CCS
  Definition]{complete, corresponding source} (CCS) for copylefted programs
  the violator knew were included (i.e., the CCS is either delivered but
  incomplete, or is not delivered at all).

\item Requestors are ignored when they communicate with violator's published
  addresses requesting fulfillment of businesses' obligations.
\end{itemize}

This tutorial therefore focuses primarily on these issue.
Admittedly, a tiny
minority of compliance situations relate to question of derivative,
combined, or modified versions of the work.  Those
situations are so rare, and the details from situation to situation differ
greatly.  Thus, such situations require a highly
fact-dependent analysis and cannot be addressed in a general-purpose
document such as this one.

\medskip

Most companies accused of violations lack a basic understanding
of how to comply even in the straightforward scenario.  This document
provides those companies with the fundamental and generally applicable prerequisite knowledge.
For answers to rarer and more complicated legal questions, such as whether
your software is a derivative or combined work of some copylefted software, consult
with an attorney.\footnote{If you would like more information on the
  application of derivative works doctrine to software, a detailed legal
  discussion is presented in our colleague Dan Ravicher's article,
  \textit{Software Derivative Work: A Circuit Dependent Determination} and in
  \tutorialpartsplit{\textit{Detailed Analysis of the GNU GPL and Related
      Licenses}'s Section on derivative works}{\S~\ref{derivative-works} of
    this tutorial}.}

This discussion thus assumes that you have already identified the
``work'' covered by the license, and that any components not under the GPL
(e.g., applications written entirely by your developers that merely happen
to run on a Linux-based operating system) distributed in conjunction with
those works are separate works within the meaning of copyright law and the GPL\@.  In
such a case, the GPL requires you to provide complete corresponding
source (CCS)\footnote{For more on CCS,  see
\tutorialpartsplit{\textit{Detailed Analysis of the GNU GPL and Related
      Licenses}'s Section on GPLv2~\S2 and GPLv3~\S1.}{\S~\ref{GPLv2s2} and \S~\ref{GPLv3s1} of
    this tutorial}.}
for the GPL'd components and your modifications thereto, but not
for independent proprietary applications.  The procedures described in
this document address this typical scenario.


\section{Monitor Software Acquisition}

Software engineers deserve the freedom to innovate and import useful
software components to improve products.  However, along with that
freedom should come rules and reporting procedures to make sure that you
are aware of what software that you include with your product.

The most typical response to an initial enforcement action is: ``We
didn't know there was GPL'd stuff in there''.  This answer indicates
failure in the software acquisition and procurement process.  Integration
of third-party proprietary software typically requires a formal
arrangement and management/legal oversight before the developers
incorporate the software.  By contrast, developers often obtain and
integrate Free Software without intervention nor oversight. That ease of acquisition, however,
does not mean the oversight is any less necessary.  Just as your legal
and/or management team negotiates terms for inclusion of any proprietary
software, they should gently facilitate all decisions to bring Free Software into your
product.

Simple, engineering-oriented rules help provide a stable foundation for
Free Software integration.  For example, simply ask your software developers to send an email to a
standard place describing each new Free Software component they add to the system,
and have them include a brief description of how they will incorporate it
into the product.  Further, make sure developers use a revision control
system (such as Git or Mercurial), and
store the upstream versions of all software in a ``vendor branch'' or
similar mechanism, whereby they can easily track and find the main version
of the software and, separately, any local changes.

Such procedures are best instituted at your project's launch.  Once 
chaotic and poorly-sourced development processes begin, cataloging the
presence of GPL'd components  becomes challenging.

Such a situation often requires use of a tool to ``catch up'' your knowledge
about what software your product includes.  Most commonly, companies choose
some software licensing scanning tool to inspect the codebase.  However,
there are few tools that are themselves Free Software.  Thus, GPL enforcers
usually recommend the GPL'd
\href{http://fossology.org/}{FOSSology system}, which analyzes a
source code base and produces a list of Free Software licenses that may apply to
the code.  FOSSology can help you build a catalog of the sources you have
already used to build your product.  You can then expand that into a more
structured inventory and process.

\section{Track Your Changes and Releases}

As explained in further detail below, the most important component of GPL
compliance is the one most often ignored: proper inclusion of CCS in all
distributions  of GPL'd
software.  To comply with GPL's CCS requirements, the distributor
\textit{must} always know precisely what sources generated a given binary
distribution.

In an unfortunately large number of our enforcement cases, the violating
company's engineering team had difficulty reconstructing the CCS
for binaries distributed by the company.  Here are three simple rules to
follow to decrease the likelihood of this occurrence:

\begin{itemize}

\item Ensure that your
developers are using revision control systems properly.

\item Have developers mark or ``tag'' the full source tree corresponding to
  builds distributed to customers.

\item Check that your developers store all parts of the software
development in the revision control system, including {\sc readme}s, build
scripts, engineers' notes, and documentation.
\end{itemize}

Your developers will benefit anyway from these rules.  Developers will be
happier in their jobs if their tools already track the precise version of
source that corresponds to any deployed binary.

\section{Avoid the ``Build Guru''}

Too many software projects rely on only one or a very few team members who
know how to build and assemble the final released product.  Such knowledge
centralization not only creates engineering redundancy issues, but also
thwarts GPL compliance.  Specifically, CCS does not just require source code,
but scripts and other material that explain how to control compilation and
installation of the executable and object code.

Thus, avoid relying on a ``build guru'', a single developer who is the only one
who knows how to produce your final product. Make sure the build process
is well defined.  Train every developer on the build process for the final
binary distribution, including (in the case of embedded software)
generating a final firmware image suitable for distribution to the
customer.  Require developers to use revision control for build processes.
Make a rule that adding new components to the system without adequate
build instructions (or better yet, scripts) is unacceptable engineering
practice.

\chapter{Details of Compliant Distribution}

Distribution of GPL'd works has requirements; copyleft will not function
without placing requirements on redistribution.  However, some requirements
are more likely to cause compliance difficult than others.  This
chapter\footnote{Note that this chapter refers heavily to specific provisions
  and language in
  \hyperref[GPLv2s3-full-text]{GPLv2\S3}
  and \hyperref[GPLv3s6-full-text]{GPLv3\S6}.
  It may be helpful  to review \S~\ref{GPLv2s3} and \S~\ref{GPLv3s6} first,
  and then have a copy of each license open while reading this
  section.}  explains some the specific requirements placed upon
distributors of GPL'd software that redistributors are most likely to
overlook, yielding compliance problems.

First, \hyperref[GPLv2s1]{GPLv2\S1} and \hyperref[GPLv2s4]{GPLv2\S4} require
that the full license text must accompany every distribution (either in
source or binary form) of each licensed work.  Strangely, this requirement is
responsible for a surprisingly significant fraction of compliance errors; too
often, physical products lack required information about the presence of
GPL'd programs and the applicable license terms.  Automated build processes
can and should carry a copy of the license from the the source distribution
into the final binary firmware package for embedded products.  Such
automation usually achieves compliance regarding license inclusion
requirements\footnote{At least one COGEO recommends the
  \href{https://www.yoctoproject.org/}{Yocto Project}, since its engineers
  have designed such features into it build process.}

\section{Binary Distribution Permission}
\label{binary-distribution-permission}

% be careful below, you cannot refill the \if section, so don't refill
% this paragraph without care.

The various versions of the GPL are copyright licenses that grant
permission to make certain uses of software that are otherwise restricted
by copyright law.  This permission is conditioned upon compliance with the
GPL's requirements.

This section walks through the requirements (of both GPLv2 and GPLv3) that
apply when you distribute GPL'd programs in binary (i.e., executable or
object code) form, which is typical for embedded applications.  Because a
binary application derives from a program's original sources, you need
permission from the copyright holder to distribute it.  \S~3 of GPLv2 and
\S~6 of GPLv3 contain the permissions and conditions related to binary
distributions of GPL'd programs.\footnote{These sections cannot be fully
  understood in isolation; read the entire license thoroughly before
  focusing on any particular provision.  However, once you have read and
  understood the entire license, look to these sections to guide
  compliance for binary distributions.}  Failure to provide or offer CCS is the
single largest failure mode leading to compliance disputes.



GPL's binary distribution sections offer a choice of compliance methods,
each of which we consider in turn.  Each option refers to the
``Corresponding Source'' code for the binary distribution, which includes
the source code from which the binary was produced.  This abbreviated and
simplified definition is sufficient for the binary distribution discussion
in this section, but you may wish to refer back to this section after
reading the thorough discussion of ``Corresponding Source'' that appears
in \S~\ref{corresponding-source}.

\subsection{Option (a): Source Alongside Binary}

GPLv2~\S~3(a) and v3~\S~6(a) embody the easiest option for providing
source code: including Corresponding Source with every binary
distribution.  While other options appear initially less onerous, this
option invariably minimizes potential compliance problems, because when
you distribute Corresponding Source with the binary, \emph{your GPL
  obligations are satisfied at the time of distribution}.  This is not
true of other options, and for this reason, we urge you to seriously
consider this option.  If you do not, you may extend the duration of your
obligations far beyond your last binary distribution.

Compliance under this option is straightforward.  If you ship a product
that includes binary copies of GPL'd software (e.g., in firmware, or on a
hard drive, CD, or other permanent storage medium), you can store the
Corresponding Source alongside the binaries.  Alternatively, you can
include the source on a CD or other removable storage medium in the box
containing the product.

GPLv2 refers to the various storage mechanisms as ``medi[a] customarily
used for software interchange''.  While the Internet has attained primacy
as a means of software distribution where super-fast Internet connections
are available, GPLv2 was written at a time when downloading software was
not practical (and was often impossible).  For much of the world, this
condition has not changed since GPLv2's publication, and the Internet
still cannot be considered ``a medium customary for software
interchange''.  GPLv3 clarifies this matter, requiring that source be
``fixed on a durable physical medium customarily used for software
interchange''.  This language affirms that option (a) requires binary
redistributors to provide source on a physical medium.

Please note that while selection of option (a) requires distribution on a
physical medium, voluntary distribution via the Internet is very useful.  This
is discussed in detail in \S~\ref{offer-with-internet}.

\subsection{Option (b): The Offer}
\label{offer-for-source}

Many distributors prefer to ship only an offer for source with the binary
distribution, rather than the complete source package.  This
option has value when the cost of source distribution is a true
per-unit cost.  For example, this option might be a good choice for
embedded products with permanent storage too small to fit the source, and
which are not otherwise shipped with a CD but \emph{are} shipped with a
manual or other printed material.

However, this option increases the duration of your obligations
dramatically.  An offer for source must be good for three full years from
your last binary distribution (under GPLv2), or your last binary or spare
part distribution (under GPLv3).  Your source code request and
provisioning system must be designed to last much longer than your product
life cycle. Thus, it also increases your compliance costs in the long
run.

In addition, if you are required to comply with the terms of GPLv2, you
{\bf cannot} use a network service to provide the source code.  For GPLv2,
the source code offer is fulfilled only with physical media.  This usually
means that you must continue to produce an up-to-date ``source code CD''
for years after the product's end-of-life.

\label{offer-with-internet}

Under GPLv2, it is acceptable and advisable for your offer for source code
to include an Internet link for downloadable source \emph{in addition} to
offering source on a physical medium.  This practice enables those with
fast network connections to get the source more quickly, and typically
decreases the number of physical media fulfillment requests.
(GPLv3~\S~6(b) permits provision of source with a public
network-accessible distribution only and no physical media.  We discuss
this in detail at the end of this section.)

The following is a suggested compliant offer for source under GPLv2 (and
is also acceptable for GPLv3) that you would include in your printed
materials accompanying each binary distribution:

\begin{quote}
The software included in this product contains copyrighted software that
is licensed under the GPL\@.  A copy of that license is included in this
document on page $X$\@.  You may obtain the complete Corresponding Source
code from us for a period of three years after our last shipment of this
product, which will be no earlier than 2011-08-01, by sending a money
order or check for \$5 to: \\
GPL Compliance Division \\
Our Company \\
Any Town, US 99999 \\
\\
Please write ``source for product $Y$'' in the memo line of your
payment.

You may also find a copy of the source at
\url{http://www.example.com/sources/Y/}.

This offer is valid to anyone in receipt of this information.
\end{quote}

There are a few important details about this offer.  First, it requires a
copying fee.  GPLv2 permits ``a charge no more than your cost of
physically performing source distribution''.  This fee must be reasonable.
If your cost of copying and mailing a CD is more than around \$10, you
should perhaps find a cheaper CD stock and shipment method.  It is simply
not in your interest to try to overcharge the community.  Abuse of this
provision in order to make a for-profit enterprise of source code
provision will likely trigger enforcement action.

Second, note that the last line makes the offer valid to anyone who
requests the source.  This is because v2~\S~3(b) requires that offers be
``to give any third party'' a copy of the Corresponding Source.  GPLv3 has
a similar requirement, stating that an offer must be valid for ``anyone
who possesses the object code''.  These requirements indicated in
v2~\S~3(c) and v3~\S~6(c) are so that noncommercial redistributors may
pass these offers along with their distributions.  Therefore, the offers
must be valid not only to your customers, but also to anyone who received
a copy of the binaries from them.  Many distributors overlook this
requirement and assume that they are only required to fulfill a request
from their direct customers.

The option to provide an offer for source rather than direct source
distribution is a special benefit to companies equipped to handle a
fulfillment process.  GPLv2~\S~3(c) and GPLv3~\S~6(c) avoid burdening
noncommercial, occasional redistributors with fulfillment request
obligations by allowing them to pass along the offer for source as they
received it.

Note that commercial redistributors cannot avail themselves of the option
(c) exception, and so while your offer for source must be good to anyone
who receives the offer (under v2) or the object code (under v3), it
\emph{cannot} extinguish the obligations of anyone who commercially
redistributes your product.  The license terms apply to anyone who
distributes GPL'd software, regardless of whether they are the original
distributor.  Take the example of Vendor $V$, who develops a software
platform from GPL'd sources for use in embedded devices.  Manufacturer $M$
contracts with $V$ to install the software as firmware in $M$'s device.
$V$ provides the software to $M$, along with a compliant offer for source.
In this situation, $M$ cannot simply pass $V$'s offer for source along to
its customers.  $M$ also distributes the GPL'd software commercially, so
$M$ too must comply with the GPL and provide source (or $M$'s \emph{own}
offer for source) to $M$'s customers.

This situation illustrates that the offer for source is often a poor
choice for products that your customers will likely redistribute.  If you
include the source itself with the products, then your distribution to
your customers is compliant, and their (unmodified) distribution to their
customers is likewise compliant, because both include source.  If you
include only an offer for source, your distribution is compliant but your
customer's distribution does not ``inherit'' that compliance, because they
have not made their own offer to accompany their distribution.

The terms related to the offer for source are quite different if you
distribute under GPLv3.  Under v3, you may make source available only over
a network server, as long as it is available to the general public and
remains active for three years from the last distribution of your product
or related spare part.  Accordingly, you may satisfy your fulfillment
obligations via Internet-only distribution.  This makes the ``offer for
source'' option less troublesome for v3-only distributions, easing
compliance for commercial redistributors.  However, before you switch to a
purely Internet-based fulfillment process, you must first confirm that you
can actually distribute \emph{all} of the software under GPLv3.  Some
programs are indeed licensed under ``GPLv2, \emph{or any later version}''
(often abbreviated ``GPLv2-or-later'').  Such licensing gives you the
option to redistribute under GPLv3.  However, a few popular programs are
only licensed under GPLv2 and not ``or any later version''
(``GPLv2-only'').  You cannot provide only Internet-based source request
fulfillment for the latter programs.

If you determine that all GPL'd works in your whole product allow upgrade
to GPLv3 (or were already GPLv3'd to start), your offer for source may be
as simple as this:

\begin{quote}
The software included in this product contains copyrighted software that
is licensed under the GPLv3\@.  A copy of that license is included in this
document on page $X$\@.  You may obtain the complete Corresponding Source
code from us for a period of three years after our last shipment of this
product and/or spare parts therefor, which will be no earlier than
2011-08-01, on our website at
\url{http://www.example.com/sources/productnum/}.
\end{quote}

\medskip

Under both GPLv2 and GPLv3, source offers must be accompanied by a copy of
the license itself, either electronically or in print, with every
distribution.
 
Finally, it is unacceptable to use option (b) merely because you do not have
Corresponding Source ready.  We find that some companies choose this option
because writing an offer is easy, but producing a source distribution as
an afterthought to a hasty development process is difficult.  The offer
for source does not exist as a stop-gap solution for companies rushing to
market with an out-of-compliance product.  If you ship an offer for source
with your product but cannot actually deliver \emph{immediately} on that
offer when your customers request it, you should expect an enforcement
action.

\subsection{Option (c): Noncommercial Offers}

As discussed in the last section, GPLv2~\S~3(c) and GPLv3~\S~6(c) apply
only to noncommercial use.  These options are not available to businesses
distributing GPL'd software.  Consequently, companies that redistribute
software packaged for them by an upstream vendor cannot merely pass along
the offer they received from the vendor; they must provide their own offer
or corresponding source to their distributees.  We talk in detail about
upstream software providers in \S~\ref{upstream}.

\subsection{Option 6(d) in GPLv3: Internet Distribution}

Under GPLv2, your formal provisioning options for Corresponding Source
ended with \S~3(c).  But even under GPLv2, pure Internet source
distribution was a common practice and generally considered to be
compliant.  GPLv2 mentions Internet-only distribution almost as aside in
the language, in text at the end of the section after the three
provisioning options are listed.  To quote that part of GPLv2~\S~3:
\begin{quote}
If distribution of executable or object code is made by offering access to
copy from a designated place, then offering equivalent access to copy the
source code from the same place counts as distribution of the source code,
even though third parties are not compelled to copy the source along with
the object code.
\end{quote}

When that was written in 1991, Internet distribution of software was the
exception, not the rule.  Some FTP sites existed, but generally software
was sent on magnetic tape or CDs.  GPLv2 therefore mostly assumed that
binary distribution happened on some physical media.  By contrast,
GPLv3~\S~6(d) explicitly gives an option for this practice that the
community has historically considered GPLv2-compliant.

Thus, you may fulfill your source-provision obligations by providing the
source code in the same way and from the same location.  When exercising
this option, you are not obligated to ensure that users download the
source when they download the binary, and you may use separate servers as
needed to fulfill the requests as long as you make the source as
accessible as the binary.  However, you must ensure that users can easily
find the source code at the time they download the binary. GPLv3~\S~6(d)
thus clarifies a point that has caused confusion about source provision in
v2.  Indeed, many such important clarifications are included in v3 which
together provide a compelling reason for authors and redistributors alike
to adopt GPLv3.

\subsection{Option 6(e) in GPLv3: Software Torrents}

Peer-to-peer file sharing arose well after GPLv2 was written, and does not
easily fit any of the v2 source provision options.  GPLv3~\S~6(e)
addresses this issue, explicitly allowing for distribution of source and
binary together on a peer-to-peer file sharing network.  If you distribute
solely via peer-to-peer networks, you can exercise this option.  However,
peer-to-peer source distribution \emph{cannot} fulfill your source
provision obligations for non-peer-to-peer binary distributions.  Finally,
you should ensure that binaries and source are equally seeded upon initial
peer-to-peer distribution.

\section{Preparing Corresponding Source}
\label{corresponding-source}

Most enforcement cases involve companies that have unfortunately not
implemented procedures like our \S~\ref{best-practices} recommendations
and have no source distribution arranged at all.  These companies must
work backwards from a binary distribution to come into compliance.  Our
recommendations in \S~\ref{best-practices} are designed to make it easy to
construct a complete and Corresponding Source release from the outset.  If
you have followed those principles in your development, you can meet the
following requirements with ease.  If you have not, you may have
substantial reconstruction work to do.

\subsection{Assemble the Sources}

For every binary that you produce, you should collect and maintain a copy
of the sources from which it was built.  A large system, such as an
embedded firmware, will probably contain many GPL'd and LGPL'd components
for which you will have to provide source.  The binary distribution may
also contain proprietary components which are separate and independent
works that are covered by neither the GPL nor LGPL\@.

The best way to separate out your sources is to have a subdirectory for
each component in your system.  You can then easily mark some of them as
required for your Corresponding Source releases.  Collecting
subdirectories of GPL'd and LGPL'd components is the first step toward
preparing your release.

\subsection{Building the Sources}

Few distributors, particularly of embedded systems, take care to read the
actual definition of Corresponding Source in the GPL\@.  Consider
carefully the definition, from GPLv3:
\begin{quote}
  The ``Corresponding Source'' for a work in object code form means all
  the source code needed to generate, install, and (for an executable
  work) run the object code and to modify the work, including scripts to
  control those activities.
\end{quote}

and the definition from GPLv2:
\begin{quote}
The source code for a work means the preferred form of the work for making
modifications to it.  For an executable work, complete source code means
all the source code for all modules it contains, plus any associated
interface definition files, plus the scripts used to control compilation
and installation of the executable.
\end{quote}

Note that you must include ``scripts used to control compilation and
installation of the executable'' and/or anything ``needed to generate,
install, and (for an executable work) run the object code and to modify
the work, including scripts to control those activities''.  These phrases
are written to cover different types of build environments and systems.
Therefore, the details of what you need to provide with regard to scripts
and installation instructions vary depending on the software details.  You
must provide all information necessary such that someone generally skilled
with computer systems could produce a binary similar to the one provided.

Take as an example an embedded wireless device.  Usually, a company
distributes a firmware, which includes a binary copy of
Linux\footnote{``Linux'' refers only to the kernel, not the larger system
  as a whole.} and a filesystem.  That filesystem contains various binary
programs, including some GPL'd binaries, alongside some proprietary
binaries that are separate works (i.e., not derived from, nor based on
freely-licensed sources).  Consider what, in this case, constitutes adequate
``scripts to control compilation and installation'' or items ``needed to
generate, install and run'' the GPL'd programs.

Most importantly, you must provide some sort of roadmap that allows
technically sophisticated users to build your software.  This can be
complicated in an embedded environment.  If your developers use scripts to
control the entire compilation and installation procedure, then you can
simply provide those scripts to users along with the sources they act
upon.  Sometimes, however, scripts were never written (e.g., the
information on how to build the binaries is locked up in the mind of your
``build guru'').  In that case, we recommend that you write out build
instructions in a natural language as a detailed, step-by-step {\sc
  readme}.

No matter what you offer, you need to give those who receive source a
clear path from your sources to binaries similar to the ones you ship.  If
you ship a firmware (kernel plus filesystem), and the filesystem contains
binaries of GPL'd programs, then you should provide whatever is necessary
to enable a reasonably skilled user to build any given GPL'd source
program (and modified versions thereof), and replace the given binary in
your filesystem.  If the kernel is Linux, then the users must have the
instructions to do the same with the kernel.  The best way to achieve this
is to make available to your users whatever scripts or process your
engineers would use to do the same.

These are the general details for how installation instructions work.
Details about what differs when the work is licensed under LGPL is
discussed in \S~\ref{lgpl}, and specific details that are unique to
GPLv3's installation instructions are in \S~\ref{user-products}.

\subsection{What About the Compiler?}

The GPL contains no provision that requires distribution of the compiler
used to build the software.  While companies are encouraged to make it as
easy as possible for their users to build the sources, inclusion of the
compiler itself is not normally considered mandatory.  The Corresponding
Source definition -- both in GPLv2 and GPLv3 -- has not been typically
read to include the compiler itself, but rather things like makefiles,
build scripts, and packaging scripts.

Nonetheless, in the interest of goodwill and the spirit of the GPL, most
companies do provide the compiler itself when they are able, particularly
when the compiler is based on GCC\@ or another copylefted compiler.  If you have
a GCC-based system, it is your prerogative to redistribute that GCC
version (binaries plus sources) to your customers.  We in the software freedom
community encourage you to do this, since it often makes it easier for
users to exercise their software freedom.  However, if you chose to take
this recommendation, ensure that your GCC distribution is itself
compliant.

If you have used a proprietary, third-party compiler to build the
software, then you probably cannot ship it to your customers.  We consider
the name of the compiler, its exact version number, and where it can be
acquired as information that \emph{must} be provided as part of the
Corresponding Source.  This information is essential to anyone who wishes
to produce a binary.  It is not the intent of the GPL to require you to
distribute third-party software tools to your customer (provided the tools
themselves are not based on the GPL'd software shipped), but we do believe
it requires that you give the user all the essential non-proprietary facts
that you had at your disposal to build the software.  Therefore, if you
choose not to distribute the compiler, you should include a {\sc readme}
about where you got it, what version it was, and who to contact to acquire
it, regardless of whether your compiler is Free Software, proprietary, or
internally developed.

\section{Best Practices and Corresponding Source}

\S~\ref{best-practices} and \S~\ref{corresponding-source} above are
closely related.  If you follow the best practices outlined above, you
will find that preparing your Corresponding Source release is an easier
task, perhaps even a trivial one.

Indeed, the enforcement process itself has historically been useful to
software development teams.  Development on a deadline can lead
organizations to cut corners in a way that negatively impacts its
development processes.  We have frequently been told by violators that
they experience difficulty when determining the exact source for a binary
in production (in some cases because their ``build guru'' quit during the
release cycle).  When management rushes a development team to ship a
release, they are less likely to keep release sources tagged and build
systems well documented.

We suggest that, if contacted about a violation, product builders use GPL
enforcement as an opportunity to improve their development practices.  No
developer would argue that their system is better for having a mysterious
build system and no source tracking.  Address these issues by installing a
revision system, telling your developers to use it, and requiring your
build guru to document his or her work!


\section{Non-Technical Compliance Issues}

Certainly, the overwhelming majority of compliance issues are, in fact,
either procedural or technical.  Thus, the primary material in this chapter
so far has covered those issues.  However, a few compliance issues do require
more direct consideration of a legal situation.  This portion guide does not
consider those in detail, as a careful reading of the earlier chapters of
Part~\ref{gpl-lgpl-part} shows various places where legal considerations are
necessary for considering compliance activity.

For example, specific compliance issues related to
\hyperref[GPLv2s7]{GPLv2\S7}, \hyperref[GPLv3s7]{GPLv3\S7}, and
\hyperref[GPLv3s7]{GPLv3\S11} demand a more traditional approach to legal
license compliance.  Of course, such analysis and consideration can be
complicated, and some are considered in the enforcement case studies that
follow in the next part.  However, compliance issues related to such sections
are not rare, and, as is typical, no specific training is available for
dealing with extremely rare occurrences.

\section{Self-Assessment of Compliance}

Most companies that adopt copylefted software believe they have complied.
Humans usually have difficult admitting their own mistakes, particularly
systematic ones.  Therefore, perhaps the most important necessary step to
stay in compliance is a company's regular evaluation of their own compliance.

First, exercise a request CCS for all copylefted works from all your upstream
providers of software and of components embedding software.  Then, perform
your own CCS check on this material first, and verify that it meets the
requirements.  This tutorial presents later a case study of a COGEO's CCS
check in \S~\ref{pristine-example}, which you can emulate when examining
their own CCS\@.

Second, measure all copyleft compliance from the position of the
users\footnote{Realizing of course that user very well may not be your own
  customer.} downstream from you exercising their rights under GPL\@.  Have
those users received notice of the copylefted software included in your
product?  Is CCS available to the users easily (preferably by automated
means)?  Ask yourself these questions frequently.  If you cannot answer these
questions with certainty in the positive, dig deeper and modify your process.

Avoid ``compliance industry'' marketing distractions and concentrate on the
copylefted software you already know is in your product.  Historically, the
risk from a copylefted code snippet that some programmer dropped in your
proprietary product careless of the consequences is a problem far more
infrequent and less difficult to resolve.  Efficient management of the risks
of higher concern lies in making sure you can provide, for example, precisely
CCS for a copy of Coreboot, the kernel named Linux, BusyBox, or GNU tar that
you included in a product your company shipped two years ago than in the risk
of 10 lines of GPL'd Java code an engineer accidentally pasted into the
source of your ERP system.

Thus, reject the ``compliance industry'' suggestions that code scanners find
and help solve fundamental compliance problems.  Consider how COGEO's tend to
use code scanners.  FOSSology is indeed an important part of a violation
investigation, but such is the last step and catches only some (usually
minor) licensing notice problems.  Thus, code scanners can help solve minor
compliance problems once you have resolved the major ones.  Code scanners
do not manage risk.

\chapter{When The Letter Comes}

Unfortunately, many GPL violators ignore their obligations until they are
contacted by a copyright holder or the lawyer of a copyright holder.  You
should certainly contact your own lawyer if you have received a letter
alleging that you have infringed copyrights that were licensed to you
under the GPL\@.  This section outlines a typical enforcement case and
provides some guidelines for response.  These discussions are
generalizations and do not all apply to every alleged violation.  However,
COGEO's in particular universally follow the processes described herein.

\section{Communication Is Key}

GPL violations are typically only escalated when a company ignores the
copyright holder's initial communication or fails to work toward timely
compliance.  Accused violators should respond very promptly to the
initial request.  As the process continues, violators should follow up weekly with the
copyright holders to make sure everyone agrees on targets and deadlines
for resolving the situation.

Ensure that any staff who might receive communications regarding alleged
GPL violations understands how to channel the communication appropriately
within your organization.  Often, initial contact is addressed for general
correspondence (e.g., by mail to corporate headquarters or by e-mail to
general informational or support-related addresses).  Train the staff that
processes such communications to escalate them to someone with authority
to take action.  An uninformed response to such an inquiry (e.g., from
a first-level technical support person) can cause negotiations to fail
prematurely.

Answer promptly by multiple means (paper letter, telephone call, and
email), even if your response merely notifies the sender that you are
investigating the situation and will respond by a certain date.  Do not
let the conversation lapse until the situation is fully resolved.
Proactively follow up with synchronous communication means to be sure
communications sent by non-reliable means (such as email) were received.

Remember that the software freedom community generally values open communication and
cooperation, and these values extend to GPL enforcement.  You will
generally find that software freedom developers and their lawyers are willing to
have a reasonable dialogue and will work with you to resolve a violation
once you open the channels of communication in a friendly way.

Furthermore, if the complaint comes from a COGEO, assume they are
well-prepared.  COGEO's fully investigate compliance issues before raising
the issue.  The claims and concerns will be substantiated, and immediate
denials will likely lead the COGEO to suspect malice rather than honest
mistake.

However, the biggest and most perennial mistake that all COGEOs see during
enforcement is this: failure to include the violators' software development
teams in the enforcement discussions and negotiations.  As described above,
CCS verification and approval is the most time-consuming and difficult part
of resolving most compliance matters.  Without direct contact between
software developers on both sides, the resolution of the technical issues
involved in demonstrating that the binary distributed was built from the
source provided is likely to be tortuous, expensive, and tense. Your lawyers
will certainly be understandably reluctant to expose your employees to direct
inquiry from potentially adverse parties.  However, facilitated exchanges of
information among software engineers communicating on technical subjects
shortens the time to resolution, substantially reduces the cost of reaching
resolution, and prevents unnecessary escalation due to mutual
misunderstanding.  Furthermore, such frank technical discussion will often be
the only way to avoid compliance litigation once a violation has occurred.

Fortunately, these frank discussions will improve your company's
relationships.  Free Software development communities improve software to
benefit everyone, which includes you and your company.  When you use
copylefted community software in your products, you are part of that
community.  Therefore, resolving a compliance matter is an occasion to
strengthen your relationship to the community, by increasing communication
between your developers and the project whose work you use for business
benefit.

\section{Termination}

Many redistributors overlook the GPL's termination provision (GPLv2~\S~4 and
GPLv3~\S~8).  Under v2, violators forfeit their rights to redistribute and
modify the GPL'd software until those rights are explicitly reinstated by
the copyright holder.  In contrast, v3 allows violators to rapidly resolve
some violations without consequence.

If you have redistributed an application under GPLv2\footnote{This applies
  to all programs licensed to you under only GPLv2 (``GPLv2-only'').
  However, most so-called GPLv2 programs are actually distributed with
  permission to redistribute under GPLv2 \emph{or any later version of the
    GPL} (``GPLv2-or-later'').  In the latter cases, the redistributor can
  choose to redistribute under GPLv2, GPLv3, GPLv2-or-later or even
  GPLv3-or-later.  Where the redistributor has chosen v2 explicitly, the
  v2 termination provision will always apply.  If the redistributor has
  chosen v3, the v3 termination provision will always apply.  If the
  redistributor has chosen GPLv2-or-later, then the redistributor may want
  to narrow to GPLv3-only upon violation, to take advantage of the
  termination provisions in v3.}, but have violated the terms of GPLv2,
you must request a reinstatement of rights from the copyright holders
before making further distributions, or else cease distribution and
modification of the software forever.  Different copyright holders
condition reinstatement upon different requirements, and these
requirements can be (and often are) wholly independent of the GPL\@.  The
terms of your reinstatement will depend upon what you negotiate with the
copyright holder of the GPL'd program.

Since your rights under GPLv2 terminate automatically upon your initial
violation, \emph{all your subsequent distributions} are violations and
infringements of copyright.  Therefore, even if you resolve a violation on
your own, you must still seek a reinstatement of rights from the copyright
holders whose licenses you violated, lest you remain liable for
infringement for even compliant distributions made subsequent to the
initial violation.

GPLv3 is more lenient.  If you have distributed only v3-licensed programs,
you may be eligible under v3~\S~8 for automatic reinstatement of rights.
You are eligible for automatic reinstatement when:
\begin{itemize}
\item you correct the violation and are not contacted by a copyright
  holder about the violation within sixty days after the correction, or

\item you receive, from a copyright holder, your first-ever contact
  regarding a GPL violation, and you correct that violation within thirty
  days of receipt of copyright holder's notice.
\end{itemize}

In addition to these permanent reinstatements provided under v3, violators
who voluntarily correct their violation also receive provisional
permission to continue distributing until they receive contact from the
copyright holder.  If sixty days pass without contact, that reinstatement
becomes permanent.  Nonetheless, you should be prepared to cease
distribution during those initial sixty days should you receive a
termination notice from the copyright holder.

Given that much discussion of v3 has focused on its so-called more
complicated requirements, it should be noted that v3 is, in this regard,
more favorable to violators than v2.

However, note that most Linux-based systems typically include some software
licensed under GPLv2-only, and thus the copyright holders have withheld
permission to redistribute under terms of GPLv3.  In larger aggregate
distributions which include GPLv2-only works (such as the kernel named
Linux), redistributors must operate as if termination is immediate and
permanent, since the technological remove of GPLv2-only works from the larger
distribution requires much more engineering work than the negotiation
required to seek restoration of rights for distribution under GPLv2-only
after permanent termination.

\chapter{Standard Requests}

As we noted above, different copyright holders have different requirements
for reinstating a violator's distribution rights.  Upon violation, you no
longer have a license under the GPL\@.  Copyright holders can therefore
set their own requirements outside the license before reinstatement of
rights.  We have collected below a list of reinstatement demands that
copyright holders often require.

\begin{itemize}

\item {\bf Compliance on all Free Software copyrights}.  Copyright holders of Free Software
  often want a company to demonstrate compliance for all GPL'd software in
  a distribution, not just their own.  A copyright holder may refuse to
  reinstate your right to distribute one program unless and until you
  comply with the licenses of all Free Software in your distribution.
 
\item {\bf Notification to past recipients}.  Users to whom you previously
  distributed non-compliant software should receive a communication
  (email, letter, bill insert, etc.) indicating the violation, describing
  their rights under the GPL, and informing them how to obtain a gratis source
  distribution.  If a customer list does not exist (such as in reseller
  situations), an alternative form of notice may be required (such as a
  magazine advertisement).

\item {\bf Appointment of a GPL Compliance Officer.}  The software freedom community
  values personal accountability when things go wrong.  Copyright holders
  often require that you name someone within the violating company
  officially responsible for Free Software license compliance, and that this
  individual serve as the key public contact for the community when
  compliance concerns arise.

\item {\bf Periodic Compliance Reports.}  Many copyright holders wish to
  monitor future compliance for some period of time after the violation.
  For some period, your company may be required to send regular reports on
  how many distributions of binary and source have occurred.
\end{itemize}

These are just a few possible requirements for reinstatement.  In the
context of a GPL violation, and particularly under v2's termination
provision, the copyright holder may have a range of requests in exchange
for reinstatement of rights.  These software developers are talented
professionals from whose work your company has benefited.  Indeed, you are
unlikely to find a better value or more generous license terms for similar
software elsewhere.  Treat the copyright holders with the same respect you
treat your corporate partners and collaborators.

\chapter{Special Topics in Compliance}

There are several other issues that are less common, but also relevant in
a GPL compliance situation.  To those who face them, they tend to be of
particular interest.

\section{LGPL Compliance}
\label{lgpl}

GPL compliance and LGPL compliance mostly involve the same issues.  As we
discussed in \S~\ref{derivative-works}, questions of modified versions of
software are highly fact-dependent and cannot be easily addressed in any
overview document.  The LGPL adds some additional complexity to the
analysis.  Namely, the various LGPL versions permit proprietary licensing
of certain types of modified versions.  These issues are discussed in greater
detail in Chapter~\ref{LGPLv2} and~\ref{LGPLv3}.  However, as a rule of thumb, once you have determined
(in accordance with LGPLv3) what part of the work is the ``Application''
and what portions of the source are ``Minimal Corresponding Source'', then
you can usually proceed to follow the GPL compliance rules that
discussed above, replacing our discussion of ``Corresponding Source'' with
``Minimal Corresponding Source''.

LGPL also requires that you provide a mechanism to combine the Application
with a modified version of the library, and outlines some options for
this.  Also, the license of the whole work must permit ``reverse
engineering for debugging such modifications'' to the library.  Therefore,
you should take care that the EULA used for the Application does not
contradict this permission.

Thus, under the terms of LGPL, you must refrain from license terms on works
based on the licensed work that prohibit replacement of the licensed
components of the larger non-LGPL'd work, or prohibit decompilation or
reverse engineering in order to enhance or fix bugs in the LGPL'd components.

LGPLv3 is not surprisingly easier to understand and examine from a compliance
lens, since the FSF was influenced in LGPLv3's drafting by questions and
comments on LGPLv2.1 over a period of years.  Admittedly, LGPLv2.1 is still
in wide use, and thus compliance with LGPLv2.1 remains a frequent topic you
may encounter.  The best advice there is careful study of
Chapter~\ref{LGPLv2}.

However, to repeat a key point here made within that chapter: Note though
that, since the LGPLv2.1 can be easily upgraded to GPLv2-or-later, in the
worst case you simply need to comply as if the software was licensed under
GPLv2.  The only reason you must consider the question of whether you have a
``work that uses the library'' or a ``work based on the library'' is when you
wish to take advantage of the ``weak copyleft'' effect of the Lesser GPL\@.
If GPLv2-or-later is an acceptable license (i.e., if you plan to copyleft the
entire work anyway), you may find this an easier option.

\section{Upstream Providers}
\label{upstream}

With ever-increasing frequency, software development (particularly for
embedded devices) is outsourced to third parties.  If you rely on an
upstream provider for your software, note that you \emph{cannot ignore
  your GPL compliance requirements} simply because someone else packaged
the software that you distribute.  If you redistribute GPL'd software
(which you do, whenever you ship a device with your upstream's software in
it), you are bound by the terms of the GPL\@.  No distribution (including
redistribution) is permissible absent adherence to the license terms.

Therefore, you should introduce a due diligence process into your software
acquisition plans.  This is much like the software-oriented
recommendations we make in \S~\ref{best-practices}.  Implementing
practices to ensure that you are aware of what software is in your devices
can only improve your general business processes.  You should ask a clear
list of questions of all your upstream providers and make sure the answers
are complete and accurate.  The following are examples of questions you
should ask:
\begin{itemize}

\item What are all the licenses that cover the software in this device?

\item From which upstream vendors, be they companies or individuals, did
  \emph{you} receive your software before distributing it to us?

\item What are your GPL compliance procedures?

\item If there is GPL'd software in your distribution, we will be
  redistributors of this GPL'd software.  What mechanisms do you have in
  place to aid us with compliance?

\item If we follow your recommended compliance procedures, will you
  formally indemnify us in case we are nonetheless found to be in
  violation of the GPL?

\end{itemize}

This last point is particularly important.  Many GPL enforcement actions are
escalated because of petty finger-pointing between the distributor and its
upstream.  In our experience, agreements regarding GPL compliance issues
and procedures are rarely negotiated up front.  However, when they are,
violations are resolved much more smoothly (at least from the point of
view of the redistributor).

Consider the cost of potential violations in your acquisition process.
Using Free Software allows software vendors to reduce costs significantly, but be
wary of vendors who have done so without regard for the licenses.  If your
vendor's costs seem ``too good to be true,'' you may ultimately bear the
burden of the vendor's inattention to GPL compliance.  Ask the right
questions, demand an account of your vendors' compliance procedures, and
seek indemnity from them.

In particular, any time your vendor incorporates copylefted software, you
\textit{must} exercise your own rights as a user to request CCS for all the
copylefted programs that your suppliers provided to you.  Furthermore, you
must ensure that CCS is correct and adequate yourself.  Good vendors should
help you do this, and make it easy.  If those vendors cannot, pick a
different vendor before proceeding with the product. 

\section{Mergers and Acquisitions}

Often, larger companies often encounter copyleft licensing during a Mergers
and Acquisitions (M\&A) process.  Ultimately, a merger or acquisition causes
all of the other company's problems to become yours.  Therefore, for most
concerns, the acquirer ``simply'' must apply the compliance analysis and
methodologies discussed earlier across the acquired company's entire product
line.  Of course, this is not so simple, as such effort may be substantial,
but a well-defined process for compliance investigation means the required
work, while voluminous, is likely rote.

A few sections of GPL require careful attention and legal analysis to
determine the risk of acquisitions.  Those handling M\&A issues should pay
particular attention to the requirements of GPLv2~\S7 and GPLv3~\S10--12 ---
focusing on how they relate to the acquired assets may be of particular
importance.

For example, GPLv3\S10 clarifies that in business acquisitions, whether by
sale of assets or transfers of control, the acquiring party is downstream
from the party acquired.  This results in new automatic downstream licenses
from upstream copyright holders, licenses to all modifications made by the
acquired business, and rights to source code provisioning for the
now-downstream purchaser.  However, despite this aid given by explicit
language in GPLv3, acquirers must still confirm compliance by the acquired
(even if GPLv3\S10 does assert the the acquirers rights under GPL, that does
not help if the acquired is out of compliance altogether).  Furthermore, for
fear of later reprisal by the acquirer if a GPL violation is later discovered
in the acquired's product line, the acquired may need to seek a waiver and
release of from additional damages beyond a requirement to comply fully (and
a promise of rights restoration) if a GPL violation by the acquired is later
uncovered during completion of the acquisition or thereafter.

Finally, other advice available regarding handling of GPL compliance in an
M\&A situation tends to ignore the most important issue: most essential
copylefted software is not wholly copyrighted by the entities involved in the
M\&A transaction.  Therefore, copyleft obligations likely reach out to the
customers of all entities involved, as well as to the original copyright
holders of the copylefted work.  As such, notwithstanding the two paragraphs
in GPLv3\S10, the entities involved in M\&A should read the copyleft licenses
through the lens of third parties whose software freedom rights under those
licenses are of equal importance to then entities inside the transaction.

\section{User Products and Installation Information}
\label{user-products}

GPLv3 requires you to provide ``Installation Information'' when v3
software is distributed in a ``User Product.''  During the drafting of v3,
the debate over this requirement was contentious.  However, the provision
as it appears in the final license is reasonable and easy to understand.

If you put GPLv3'd software into a User Product (as defined by the
license) and \emph{you} have the ability to install modified versions onto
that device, you must provide information that makes it possible for the
user to install functioning, modified versions of the software.  Note that
if no one, including you, can install a modified version, this provision
does not apply.  For example, if the software is burned onto an
non-field-upgradable ROM chip, and the only way that chip can be upgraded
is by producing a new one via a hardware factory process, then it is
acceptable that the users cannot electronically upgrade the software
themselves.

Furthermore, you are permitted to refuse support service, warranties, and
software updates to a user who has installed a modified version.  You may
even forbid network access to devices that behave out of specification due
to such modifications.  Indeed, this permission fits clearly with usual
industry practice.  While it is impossible to provide a device that is
completely unmodifiable\footnote{Consider that the iPhone, a device
  designed primarily to restrict users' freedom to modify it, was unlocked
  and modified within 48 hours of its release.}, users are generally on
notice that they risk voiding their warranties and losing their update and
support services when they make modifications.\footnote{A popular t-shirt
  in the software freedom community reads: ``I void warranties.''.  Our community is
  well-known for modifying products with full knowledge of the
  consequences.  GPLv3's ``Installation Instructions'' section merely
  confirms that reality, and makes sure GPL rights can be fully exercised,
  even if users exercise those rights at their own peril.}

GPLv3 is in many ways better for distributors who seek some degree of
device lock-down.  Technical processes are always found for subverting any
lock-down; pursuing it is a losing battle regardless.  With GPLv3, unlike
with GPLv2, the license gives you clear provisions that you can rely on
when you are forced to cut off support, service or warranty for a customer
who has chosen to modify.

% FIXME-soon: write a full section on Javascript compliance.  Here's a
%             potentially useful one-sentence introduction for such a
%             section.

% Non-compliance with GPLv3 in the
% distribution of Javascript on the Web is becoming more frequent
%FIXME-soon: END

\section{Beware The Consultant in Enforcers' Clothing}

There are admittedly portions of the GPL enforcement community that function
somewhat like the
\href{http://en.wikipedia.org/wiki/Hacker_%28computer_security%29#Classifications}{computer
  security and network penetration testing hacker community}.  By analogy,
most COGEO's consider themselves
\href{http://en.wikipedia.org/wiki/White_hat_%28computer_security%29}{white hats},
while some might appropriately call
\hyperref[Proprietary Relicensing]{proprietary relicensing} by the name ``\href{http://en.wikipedia.org/wiki/Hacker_%28computer_security%29#Black_hat}{black hats}''.
And, to finalize the analogy, there are indeed few
\href{http://en.wikipedia.org/wiki/Grey_hat}{grey hat} GPL enforcers.

Grey hat GPL enforcers usually have done some community-oriented GPL
enforcement themselves, typically working as a volunteer for a COGEO, but make
their living as a ``hired gun'' consultant to find GPL violations and offer
to ``fix them'' for companies.  Other such operators hold copyrights in some
key piece of copylefted software and enforce as a mechanism to find out who
is most likely to fund improvements on the software.

A few companies report that they have formed beneficial consulting or
employment relationships with developers they first encountered through
enforcement.  In some such cases, companies have worked with such consultants
to alter the mode of use of the project's code in the company's products.
More often in these cases, the communication channels opened in the course of
the inquiry served other consulting purposes later.

Feelings and opinions about this behavior are mixed within the larger
copyleft community.  Some see it as a reasonable business model and others
renounce it as corrupt behavior.  However, from the point of view of a GPL
violator, the most important issue is to determine the motivations of the
enforcer.  The COGEOs such as the FSF and Conservancy have made substantial
public commitments to enforce in a way that is uniform, transparent, and
publicly documented.  Since these organizations are public charities, they
are accountable to the IRS and the public at large in their annual Form 990
filings, and everyone can examine their revenue models and scrutinize their
work.

However, entities and individuals who do GPL enforcement centered primarily
around a profit motive are likely the most dangerous enforcement entities for
one simple reason: an agreement to comply fully with the GPL for past and
future products, which is always the paramount goal to COGEOs, may not be an
adequate resolution for a proprietary relicensing company or grey hat GPL
enforcer.  Therefore, violators are advised to consider carefully who has
made the enforcement inquiry and ask when and where they have made public
commitments and reports regarding their enforcement work, perhaps asking them
to directly mimic the detailed public disclosures done by COGEOs.

\chapter{Conclusion}

GPL compliance need not be an onerous process.  Historically, struggles
have been the result of poor development methodologies and communications,
rather than any unexpected application of the GPL's source code disclosure
requirements.

Compliance is straightforward when the entirety of your enterprise is
well-informed and well-coordinated.  The receptionists should know how to
route a GPL source request or accusation of infringement.  The lawyers
should know the basic provisions of Free Software licenses and your source
disclosure requirements, and should explain those details to the software
developers.  The software developers should use a version control system
that allows them to associate versions of source with distributed
binaries, have a well-documented build process that anyone skilled in the
art can understand, and inform the lawyers when they bring in new
software.  Managers should build systems and procedures that keep everyone
on target.  With these practices in place, any organization can comply
with the GPL without serious effort, and receive the substantial benefits
of good citizenship in the software freedom community, and lots of great code
ready-made for their products.

\vfill

% LocalWords:  redistributors NeXT's Slashdot Welte gpl ISC embedders BusyBox
% LocalWords:  someone's downloadable subdirectory subdirectories filesystem
% LocalWords:  roadmap README upstream's Ravicher's FOSSology readme CDs iPhone
% LocalWords:  makefiles violator's Michlmayr Stallman RMS GPL'd Harald LGPL
%%  LocalWords:  GPL's resellers copylefted sublicenses GPLv unmanaged MySQL
%%  LocalWords:  misassessments licensor COGEOs COGEO LGPLv CCS Requestors
%%  LocalWords:  codebase Yocto distributees COGEO's Coreboot ERP reseller
%%  LocalWords:  redistributor reinstatements decompilation acquired's grey
%%  LocalWords:  upgradable unmodifiable Relicensing relicensing


%      Tutorial Text for the Detailed Study and Analysis of GPL and LGPL course

% License: CC-By-SA-4.0

% The copyright holders hereby grant the freedom to copy, modify, convey,
% Adapt, and/or redistribute this work under the terms of the Creative
% Commons Attribution Share Alike 4.0 International License.

% This text is distributed in the hope that it will be useful, but
% WITHOUT ANY WARRANTY; without even the implied warranty of
% MERCHANTABILITY or FITNESS FOR A PARTICULAR PURPOSE.

% You should have received a copy of the license with this document in
% a file called 'CC-By-SA-4.0.txt'.  If not, please visit
% https://creativecommons.org/licenses/by-sa/4.0/legalcode to receive
% the license text.


\part{Case Studies in GPL Enforcement}

{\parindent 0in
This part is: \\
\begin{tabbing}
Copyright \= \copyright{} 2003, 2004, 2014 \= \hspace{1.mm} \=  \kill
Copyright \> \copyright{} 2014 \>  Bradley M. Kuhn. \\
Copyright \= \copyright{} 2014 \> \hspace{.2in} Denver Gingerich \\
Copyright \= \copyright{} 2003, 2004, 2014 \= \hspace{.2in} Free Software Foundation, Inc. \\
\end{tabbing}

\vspace{1in}

\begin{center}
Authors of this part are: \\

Bradley M. Kuhn \\
John Sullivan
\vspace{3in}

Copy editors of this part include: \\
Martin Michlmayr

\vspace{3in}

The copyright holders hereby grant the freedom to copy, modify, convey,
Adapt, and/or redistribute this work under the terms of the Creative Commons
Attribution Share Alike 4.0 International License.  A copy of that license is
available at \url{https://creativecommons.org/licenses/by-sa/4.0/legalcode}.
\end{center}
}
% =====================================================================
% START OF SECOND DAY SEMINAR SECTION
% =====================================================================

\chapter*{Preface}

This one-day course presents the details of five different GPL
compliance cases handled by FSF's GPL Compliance Laboratory. Each case
offers unique insights into problems that can arise when the terms of
the GPL are not properly followed, and how diplomatic negotiation between
the violator and the copyright holder can yield positive results for
both parties.

Attendees should have successfully completely the course, a ``Detailed
Study and Analysis of the GPL and LGPL,'' as the material from that
course forms the building blocks for this material.

This course is of most interest to lawyers who have clients or
employers that deal with Free Software on a regular basis. However,
technical managers and executives whose businesses use or distribute
Free Software will also find the course very helpful.

\bigskip

These course materials are merely a summary of the highlights of the
course presented. Please be aware that during the actual GPL course, class
discussion supplements this printed curriculum. Simply reading it is
not equivalent to attending the course.

%FIXME-LATER: write these

%\chapter{Not All GPL Enforcement is Created Equal}

%\section{For-Profit Enforcement}

%\section{Community and Non-Profit Enforcement}

\chapter{Overview of Community Enforcement}

The GPL is a Free Software license with legal teeth. Unlike licenses like
the X11-style or various BSD licenses, the GPL (and by extension, the LGPL) is
designed to defend as well as grant freedom. We saw in the last course
that the GPL uses copyright law as a mechanism to grant all the key freedoms
essential in Free Software, but also to ensure that those freedoms
propagate throughout the distribution chain of the software.

\section{Termination Begins Enforcement}

As we have learned, the assurance that Free Software under the GPL remains
Free Software is accomplished through various terms of the GPL: \S 3 ensures
that binaries are always accompanied with source; \S 2 ensures that the
sources are adequate, complete and usable; \S 6 and \S 7 ensure that the
license of the software is always the GPL for everyone, and that no other
legal agreements or licenses trump the GPL. It is \S 4, however, that ensures
that the GPL can be enforced.

Thus, \S 4 is where we begin our discussion of GPL enforcement. This
clause is where the legal teeth of the license are rooted. As a copyright
license, the GPL governs only the activities governed by copyright law ---
copying, modifying and redistributing computer software. Unlike most
copyright licenses, the GPL gives wide grants of permission for engaging with
these activities. Such permissions continue, and all parties may exercise
them until such time as one party violates the terms of the GPL\@. At the
moment of such a violation (i.e., the engaging of copying, modifying or
redistributing in ways not permitted by the GPL) \S 4 is invoked. While other
parties may continue to operate under the GPL, the violating party loses their
rights.

Specifically, \S 4 terminates the violators' rights to continue
engaging in the permissions that are otherwise granted by the GPL\@.
Effectively, their rights revert to the copyright defaults ---
no permission is granted to copy, modify, nor redistribute the work.
Meanwhile, \S 5 points out that if the violator has no rights under
the GPL, they are prohibited by copyright law from engaging in the
activities of copying, modifying and distributing. They have lost
these rights because they have violated the GPL, and no other license
gives them permission to engage in these activities governed by copyright law.

\section{Ongoing Violations}

In conjunction with \S 4's termination of violators' rights, there is
one final industry fact added to the mix: rarely does one engage in a
single, solitary act of copying, distributing or modifying software.
Almost always, a violator will have legitimately acquired a copy of a
GPL'd program, either making modifications or not, and then begun
distributing that work. For example, the violator may have put the
software in boxes and sold them at stores. Or perhaps the software
was put up for download on the Internet. Regardless of the delivery
mechanism, violators almost always are engaged in {\em ongoing\/}
violation of the GPL\@.

In fact, when we discover a GPL violation that occurred only once --- for
example, a user group who distributed copies of a GNU/Linux system without
source at one meeting --- we rarely pursue it with a high degree of
tenacity. In our minds, such a violation is an educational problem, and
unless the user group becomes a repeat offender (as it turns out, they
never do), we simply forward along a FAQ entry that best explains how user
groups can most easily comply with the GPL, and send them on their merry way.

It is only the cases of {\em ongoing\/} GPL violation that warrant our
active attention. We vehemently pursue those cases where dozens, hundreds
or thousands of customers are receiving software that is out of
compliance, and where the company continually offers for sale (or
distributes gratis as a demo) software distributions that include GPL'd
components out of compliance. Our goal is to maximize the impact of
enforcement and educate industries who are making such a mistake on a
large scale.

In addition, such ongoing violation shows that a particular company is
committed to a GPL'd product line. We are thrilled to learn that someone
is benefiting from Free Software, and we understand that sometimes they
become confused about the rules of the road. Rather than merely
giving us a postmortem to perform on a past mistake, an ongoing violation
gives us an active opportunity to educate a new contributor to the GPL'd
commons about proper procedures to contribute to the community.

Our central goal is not, in fact, to merely clear up a particular violation.
In fact, over time, we hope that our compliance lab will be out of
business. We seek to educate the businesses that engage in commerce
related to GPL'd software to obey the rules of the road and allow them to
operate freely under them. Just as a traffic officer would not revel in
reminding people which side of the road to drive on, so we do not revel in
violations. By contrast, we revel in the successes of educating an
ongoing violator about the GPL so that GPL compliance becomes a second-nature
matter, allowing that company to join the GPL ecosystem as a contributor.

\section{How are Violations Discovered?}

Our enforcement of the GPL is not a fund-raising effort; in fact, FSF's GPL
Compliance Lab runs at a loss (in other words, it is subsided by our
donors). Our violation reports come from volunteers, who have encountered,
in their business or personal life, a device or software product that
appears to contain GPL'd software. These reports are almost always sent
via email to $<$license-violation@fsf.org$>$.

Our first order of business, upon receiving such a report, is to seek
independent confirmation. When possible, we get a copy of the software
product. For example, if it is an offering that is downloadable from a
Web site, we download it and investigate ourselves. When it is not
possible for us to actually get a copy of the software, we ask the
reporter to go through the same process we would use in examining the
software.

By rough estimation, about 95\% of violations at this stage can be
confirmed by simple commands. Almost all violators have merely made an
error and have no nefarious intentions. They have made no attempt to
remove our copyright notices from the software. Thus, given the
third-party binary, {\tt tpb}, usually, a simple command (on a GNU/Linux
system) such as the following will find a Free Software copyright notice
and GPL reference:
\begin{quotation}
{\tt strings tpb | grep Copyright}
\end{quotation}
In other words, it is usually more than trivial to confirm that GPL'd
software is included.

Once we have confirmed that a violation has indeed occurred, we must then
determine whose copyright has been violated. Contrary to popular belief,
FSF does not have the power to enforce the GPL in all cases. Since the GPL
operates under copyright law, the powers of enforcement --- to seek
redress once \S 4 has been invoked --- lie with the copyright holder of
the software. FSF is one of the largest copyright holders in the world of
GPL'd software, but we are by no means the only one. Thus, we sometimes
discover that while GPL'd code is present in the software, there is no
software copyrighted by FSF present.

In cases where FSF does not hold copyright interest in the software, but
we have confirmed a violation, we contact the copyright holders of the
software, and encourage them to enforce the GPL\@. We offer our good offices
to help negotiate compliance on their behalf, and many times, we help as a
third party to settle such GPL violations. However, what we will describe
primarily in this course is FSF's first-hand experience enforcing its own
copyrights and the GPL\@.

\section{First Contact}

The Free Software community is built on a structure of voluntary
cooperation and mutual help. Our community has learned that cooperation
works best when you assume the best of others, and only change policy,
procedures and attitudes when some specific event or occurrence indicates
that a change is necessary. We treat the process of GPL enforcement in
the same way. Our goal is to encourage violators to join the cooperative
community of software sharing, so we want to open our hand in friendship.

Therefore, once we have confirmed a violation, our first assumption is
that the violation is an oversight or otherwise a mistake due to confusion
about the terms of the license. We reach out to the violator and ask them
to work with us in a collaborative way to bring the product into
compliance. We have received the gamut of possible reactions to such
requests, and in this course, we examine four specific examples of such
compliance work.

% FIXME: make this section properly TeX-formatted
\chapter{ThinkPenguin Wireless Router: Excellent CCS}

Too often, case studies examine failure and mistakes.  Indeed, most of the
chapters that follow herein will consider the myriad difficulties discovered
in community-oriented GPL enforcement for the last two decades.  However, to
begin, we offer a study in how copyleft compliance can be done correctly.

This example is, in fact, more than ten years in the making.  Since almost
the inception of for-profit corporate adoption of Free Software, companies
have requested a clear example of a model citizen to emulate.  Sadly, while
community-oriented enforcers have vetted uncounted thousands of ``Complete,
Corresponding Source'' CCS candidates from hundreds of companies, the CCS
release describes the first one CCS experts have declared a ``pristine
example''.

% FIXME (above): link to a further discussion of CCS in the compliance guide
% when a good spot exists, then (below) link to a ``CCS iteration''
% discussion in compliance-guide.tex when one exists.  (the ``iteration
% process'' is discussed in~\ref{} of this guide)

Of course, most CCS examined for the last decade has (eventually) complied
with the GPL, perhaps after many iterations of review by the enforcer.
However, in the experience of the two primary community-oriented enforcers,
Conservancy and the FSF, such CCS results routinely fix the description of
``barely complies with GPL's requirements''.  To use an academic analogy:
while a ``C'' is certainly a passing grade, any instructor prefers to
disseminate to the class an exemplar sample that earned an ``A''.

Fortunately, thanks in large part to the FSF's
``Respects Your Freedom'' (RYF) certification campaign\footnote{\href{RYF is
    a campaign by FSF to certify products that truly meet the principles of
    software freedom}.  Products must meet
  \href{http://www.fsf.org/resources/hw/endorsement/criteria}{strict
    standards for RYF certification}, and among them is a pristine example of
  CCS\@}, electronics products have begun to appear on the market that are
held to a higher standard of copyleft compliance.  As such, for the first
time in the history of copyleft, CCS experts have pristine examples to study
and present as exemplars worthy of emulation.

This case study therefore examines the entire life-cycle of a GPL compliance
investigation: from product purchase, to source request, to CCS review.
Specifically, this chapter discusses the purchase, CCS provision, and a
step-by-step build and installation analysis of a specific, physical,
embedded electronics product.  The product in question is
\href{https://www.thinkpenguin.com/gnu-linux/free-software-wireless-n-broadband-router-gnu-linux-tpe-nwifirouter}{the
  ``TPE-NWIFIROUTER'' wireless router by ThinkPenguin}.\footnote{The FSF of
  course performed a thorough CCS check as part of the certification process.
  The analysis discussed herein was independently performed by Software
  Freedom Conservancy without reviewing any findings of the FSF, and thus the
  analysis provides a ``true to form'' analysis as it occurs when Conservancy
  investigates a potential GPL violation.  In this case, obviously, no
  violation was uncovered.}

\section{Consumer Purchase and Unboxing}

The process for copyleft compliance investigation, when properly conducted,
determines whether users inclined to exercise their rights under a copyleft
license will be successful in their attempt.  Therefore, at every stage, the
investigator seeks to take actions that reasonably technically knowledgeable
users would during the ordinary course of their acquisition and use of
products.  As such, the investigator typically purchases the device on the
open market to verify that distribution of the copylefted software therein
complies with binary distribution requirements (such as those
\tutorialpartsplit{discussed in \textit{Detailed Analysis of the GNU GPL and
    Related Licenses}}{discussed here in \S~\ref{GPLv2s3} and
  \S~\ref{GPLv3s6}}).

% FIXME: Above is my only use of \tutorialpartsplit in this chapter.  I just
% got lazy and that should be fixed by someone.

\label{thinkpenguin-included-ccs}

Therefore, the investigator first purchased the TPE-NWIFIROUTER through an
online order, and when the package arrived, examined the contents of the box.
The investigator immediately discovered that ThinkPenguin had taken advice
from \S~\ref{offer-for-source} in this guide, and had chosen to use
\hyperref[GPLv2s3a]{GPLv2\S3(a)} and \hyperref[GPLv3s6]{GPLv3s6}, rather than
using the \hyperref[offer-for-source]{problematic offer for source
  provisions}.  This choice not only speeds up the investigation (since there
is no CCS offer to test), but also simplifies the compliance requirements for
ThinkPenguin.

\section{Root Filesystem and Kernel Compilation}

The CD found in the box was labeled ``libreCMC v1.2.1 source code'', and
contained 407 megabytes of data.  The investigator copied this ISO and
examined its contents.  Upon doing so, the investigator immediately found a
file called ``README'' at the top-level directory:

\lstset{tabsize=2}
\begin{lstlisting}[language=bash]
  $ dd if=/dev/cdrom of=libreCMC_v1.2.1_SRC.iso
  $ mkdir libCMC
  $ sudo mount -o loop ./libreCMC_v1.2.1_SRC.iso libCMC
  mount: block device /path/to/libreCMC_v1.2.1_SRC.iso is write-protected, mounting read-only
  $ ls -1 libCMC
  bin
  librecmc-u-boot.tar.bz2
  librecmc-v1.2.1.tar.bz2
  README
  u-boot_reflash
  $ cat libCMC/README
\end{lstlisting}
\label{thinkpenguin-toplevel-readme}
The investigator therefore knew immediately to begin the CCS check by
studying the contents of the ``README'', which contained the appropriate
details to get started with a build:
\begin{quotation}

In order to build firmware images for your router,the following needs to be
installed:

gcc, binutils, bzip2, flex, python, perl, make, find, grep, diff, unzip,
gawk, getopt, libz-dev and libc headers.

Please use ``make menuconfig'' to configure your appreciated configuration
for the toolchain and firmware. Please note that the default configuration is
what was used to build the firmware image for your router. It is advised that
you use this configuration.

Simply running ``make'' will build your firmware.  The build system will
download all sources, build the cross-compile toolchain, the kernel and all
chosen applications.

To build your own firmware you need to have access to a GNU/Linux system
(case-sensitive filesystem required).
\end{quotation}

In other words, the first ``script'' that investigator ran in building
testing this CCS candidate was the above, which ran on the investigator's own
brain --- like a script of a play.  Less glibly, instructions written in
English are particularly necessary for parts of the build and installation
process that cannot require some amount of actual intelligence to complete.
In this case, the investigator was able to determine the requirements for the
host system to use when constructing the firmware for the embedded device.

GPL does not, of course, give specific guidance on the form or location of
such instructions.  Community-oriented GPL enforcers generally use a
reasonableness standard to evaluate such instructions.  If an investigator of
average skill in embedded firmware construction can surmise the proper
procedures to build and install a replacement firmware, the instructions are
likely sufficient to meet GPL's requirements.  However, in this case, the
instructions are more abundant and give more detail.

These instructions are more general than typical.  Often, top-level build
instructions will specifically name a host distribution to use, such as
``Debian 7 installed on a amd64 system with the following packages
installed''.  If the build will not complete on any other system,
instructions should have such details.  However, in this case, the CCS can
build on a wide range of distributions, and thus no specific distribution was
specified.

\label{thinkpenguin-specific-host-system}

In this specific case, the developers of the libreCMC project (on which the
TPE-NWIFIROUTER is based) have clearly made effort to ensure the CCS builds
on a variety of host systems.  The investigator was in fact dubious upon
seeing these instructions, since finicky embedded build processes usually
require a very specific host system.   Even in this case, a
\hyperref[thinkpenguin-glibc-214-issue]{minor annoyance was found that more
  detailed instructions would address}.

Anyway, since these instructions did not specify a specific host system, the
investigator simply used his own amd64 Debian 6 desktop system.  Before
beginning, the investigator used the following command:

\lstset{tabsize=2}
\begin{lstlisting}[language=bash]
  $ dpkg --list | egrep '^iii' | less
\end{lstlisting}

to verify that the required packages listed in the README were
installed\footnote{The ``dpkg'' command is a Debian-specific way of
  finding installed packages.}.


Next, the investigator then extracted the primary source package with the
following command:

\lstset{tabsize=2}
\begin{lstlisting}[language=bash]
  $ tar --posix -jxpf libCMC/librecmc-v1.2.1.tar.bz2
\end{lstlisting}

The investigator did notice an additional source release, entitled
``librecmc-u-boot.tar.bz2''.  The investigator concluded upon simple
inspection that the instructions found in ``u-boot\verb0_0reflash'' were
specific instructions for that part of the CCS\@.  This was a minor
annoyance, and ideally the ``README'' would list that fact, but the existing
layout met the reasonable standard that community-oriented GPL enforcers
typically apply, since the skilled investigator could determine the correct
course of action with a few moments of study.

The investigator then noted the additional step offered by the ``README'',
which read:
\begin{quotation}
Please use ``make menuconfig'' to configure your appreciated configuration
for the toolchain and firmware. Please note that the default configuration is
what was used to build the firmware image for your router. It is advised that
you use this configuration.
\end{quotation}

This instruction actually goes above and beyond the requirements of GPL\@.
Specifically, the instruction guides users in their first step toward
exercising the freedom to modify the software.  While the GPL does contain
requirements that facility the freedom to modify (such as ensuring the CCS is
in the ``preferred form \ldots for making modifications to it'' form, it
does not require that you write specific instructions explaining how
modifications might be undertaken.  This instruction therefore exemplifies
the exceptional quality of this particular CCS\@.

%FIXME: add a \hyperref to some ``preferred for for modification'' stuff above.

However, for purposes of the CCS verification process, typically the
investigator avoids any unnecessary changes to the source code during the
build process, lest the investigator err and cause the build to fail through
his own modification, and thus incorrectly identify the CCS as inadequate.
Therefore, the investigator proceeded to simply run:

\lstset{tabsize=2}
\begin{lstlisting}[language=bash]
  $ cd libCMC
  $ make
\end{lstlisting}

and waited approximately 40 minutes for the build to complete\footnote{Build
  times will likely vary widely on various host systems}.  The investigator
kept a
\href{https://gitorious.org/copyleft-org/tutorial/source/master:enforcement-case-studies_log-output/thinkpenguin_librecmc-complete.log}{full
  log of the build}, which is not included herein due its size (approximately
7.2K of text).
\label{thinkpenguin-main-build}

Upon competition of the ``make'' process, the investigator immediately found
(almost to his surprise) several large firmware files in the ``bin/ar71xx''
directory.  Typically, this step in the CCS verification process is
harrowing.  In most cases, the ``make'' step will fail due to a missing
package or because toolchain paths are not setup correctly.

From experience, the investigator is sure that ThinkPenguin's engineers did
the most important step in self-CCS verification: use one's own instructions
on a clean system.  Ideally, an employee with similar skills but
unfamiliar with the specific product can most easily verify CCS  and identify
problems before a violation occurs.

% FIXME: Is there stuff about the above in the compliance guide?  If so, link
% to it.  If not, write it, then link to it. :)

However, upon completing the ``make'', the investigator was unclear which
filesystem and kernel images to install on the TPE-NWIFIROUTER hardware.
Ideally, the original ``README'' would indicate which image is appropriate
for the included hardware.  However, this was ultimately an annoyance rather
than a compliance issue due to other information available.  Specifically,
the web UI (see next section) on the TPE-NWIFIROUTER performs firmware image
installation.  While ideal would be to find
\href{http://librecmc.org/librecmc/wiki?name=Tp+MR3020}{instructions similar
  to these} in the README itself.  However, application of the reasonableness
standard indicates compliance, since a knowledgeable user was able to
determine the proper course of action.


\section{U-Boot Compilation}

%FIXME: link to u-boot reflash, maybe put it in log-output dir?

The investigator then turned his attention to the file,
``u-boot\verb0_0reflash'' instructions.  These instructions explained how to
build and install the bootloader for the device.

The investigator followed the instructions for compiling u-Boot, and found
them quite straight-forward.  The investigator discovered two minor
annoyances, however, while building U-Boot: 

\begin{itemize}

 \item the variable \verb0$U-BOOT_SRC0 was used as a placeholder for the name
   of the extracted source directory.  This was easy to surmise and was not a
   compliance issue (per the reasonableness standard), but explicitly stating
   that at the top of the instructions would be helpful.

\item Toolchain binaries were included and used by default by the build
  process.  These binaries were not the appropriate ones for the
  investigator's host system, and the build failed with the following error:

\lstset{tabsize=2}
\begin{lstlisting}
mips-librecmc-linux-uclibc-gcc.bin: /lib/libc.so.6:
   version `GLIBC`_2.14' not found
     (required by mips-librecmc-linux-uclibc-gcc.bin)
\end{lstlisting}

   (The
\href{https://gitorious.org/copyleft-org/tutorial/source/master:enforcement-case-studies_log-output/thinkpenguin_u-boot-build_fail.log}{complete
  log output from the failure} is too lengthy to include herein.)

   This issue is an annoyance, not a compliance problem.  It was clear from
   context that these binaries were simply for a different architecture, and
   the investigator simply removed ``toolchain/bin'' and used a symlink the
   utilize the toolchain already built earlier (during the compilation
   discussed in \S~\ref{thinkpenguin-main-build}):

\lstset{tabsize=2}
\begin{lstlisting}
  $ ln -s \
  ../../staging_dir/toolchain-mips_34kc_gcc-4.6-linaro_uClibc-0.9.33.2/bin \
  toolchain/bin
\end{lstlisting}


   After this change, the U-Boot build completed successfully.
\end{itemize}

The
\href{https://gitorious.org/copyleft-org/tutorial/source/master:enforcement-case-studies_log-output/thinkpenguin_u-boot-finish_build.log}{full
  log of the build} is not included herein due its size (approximately 3.8K
of text).  After that, the investigator found a new U-Boot image in the
``bin'' directory.

\section{Root Filesystem and Kernel Installation}

The investigator next tested installation of the firmware.  In particular,
the investigator connected the TPE-NWIFIROUTER to a local network, and
visited \url{http://192.168.10.1/}, logged in, and chose the option sequence:
``System $\Rightarrow$ Backup / Flash Firmware''.

From there, the investigator chose the ``Flash new firmware image'' section
and selected the
``librecmc-ar71xx-generic-tl-wr841n-v8-squashfs-sysupgrade.bin'' image from
the ``bin/ar71xx'' directory.  The investigator chose the ``v8'' image upon
verifying the physical router read ``v8.2'' on its bottom.  The investigator
chose the ``sysupgrade'' version of the image because this was clearly a
system upgrade (as a firmware already came preinstalled on the
TPE-NWIFIROUTER).

Upon clicking ``Flash image\ldots'', the web interface prompted the
investigator to confirm the MD5 hash of the image to flash.  The investigator
did so, and then clicked ``Proceed'' to flash the image.  The process took
about one minute, at which point the web page refreshed to the login screen.
Upon logging in, the investigator was able to confirm in ``Kernel Log''
section of the interface that the newly built copy of Linux had indeed been
installed.

The investigator confirmed that a new version of ``busybox'' had also been
installed by using SSH to connect to the router and ran the command
``busybox'', which showed the newly-compiled version (via its date of
compilation).

%FIXME: dg: can you get me  a screen shot for the Kernel Log above, and paste
%in the output of running busybox ?

%% \section{U-Boot Installation}

%% The U-Boot installation process is substantially more complicated than the
%% firmware update.  The investigator purchased the optional a serial cable
%% along with the TPE-NWIFIROUTER, in order to complete the U-Boot installation
%% per the instructions in'' -boot\verb0_0reflash''.

%% However, we were
%% only able to read data from the serial port; we were unable to interrupt the
%% boot process or access the U-Boot console to complete the U-Boot re-flash.  Here
%% are the steps we tried:

%% * We found the serial cable included was a USB serial adapter that had a male
%%   USB type A connector on one end and 4 female jumper wires at the other end.
%%   These female jumper wires were red, black, white, and green.
%% * The instructions did not specify how to connect these wires, but we were able
%%   to determine this in part using the "v8.4" image (close to our "v8.2" router)
%%   at \url{http://wiki.openwrt.org/toh/tp-link/tl-wr841nd#serial.console} .  Aside from
%%   power and ground (red and black), we did have to guess which of the wires was
%%   RX and TX.  By experimentation we found that green was RX and white was TX.
%%   When we tried the other way, we received no data to our serial console at boot
%%   time.
%% * We did have to use the included jumper pin gender changer with the USB serial
%%   adapter, which we put through the holes on the router's mainboard and then
%%   connected to the USB serial adapter.  The fit was fairly loose so it would be
%%   nice if future router versions included a tighter gender changer or (ideally)
%%   had the jumper pins soldered onto the board to begin with (so no gender
%%   changer would be required).
%% * We used 115200 8N1 as our serial console settings (with no hardware or
%%   software flow control).  This was tested with both the minicom and screen
%%   commands.  We found that if we connected all 4 wires on the USB serial adapter
%%   that the router would start without additional power and our console would
%%   receive the startup messages.  We could replicate the same behavior by
%%   omitting the power cable from the USB serial adapter (red wire) and connecting
%%   the main power adapter to the router instead.
%% * While we did see the U-Boot and kernel boot logs in our serial console, we
%%   were unable to interrupt the boot process as u-boot\verb0_0reflash indicated we
%%   should.  We suspect this is a misconfiguration of our serial console, but it's
%%   unclear exactly how it is misconfigured, as we were able to receive data fine
%%   (we just couldn't send data to the router).
%% * As a result, we were unable to complete the U-Boot installation test.  We did
%%   appreciate that installation instructions were included, though these
%%   instructions should be updated to include more specifics about connecting the
%%   serial cable.  Since ThinkPenguin does have the option to ship a serial
%%   adapter with the router, it would be helpful if instructions specific to that
%%   adapter were included, as the wiring configuration one should use was unclear.
%% * Additionally, instructions for removing the router's case should be included.
%%   We found that the two screws that needed removal to open the case were hidden
%%   underneath rubber feet on the case.  Indicating which feet need removal to
%%   unscrew the case would be helpful.  The instructions should also note that the
%%   case needs to be carefully separated once the screws are removed; it
%%   effectively snaps apart, but care must be taken to avoid breaking the plastic
%%   fasteners that keep the case together after the screws are removed.

\section{Firmware Comparison}

To ensure that CCS did corresponds properly to the firmware original
installed on the TPE-NWIFIROUTER, the investigator compared the built
firmware image with the filesystem originally found on the device itself.
The comparison steps we as follows:

\begin{enumerate}
  
\item Extract the filesystem from the image we built by running
  \href{https://gitorious.org/copyleft-org/gpl-compliance-scripts/source/master:find-firmware.pl}{find-firmware.pl}
  on ``bin/ar71xx/librecmc-ar71xx-generic-tl-wr841n-v8-squashfs-factory.bin''
  bottom), and running
  \href{http://www.binaryanalysis.org/en/content/show/download}{bat-extratools}'
  ``squashfs4.2/squashfs-tools/bat-unsquashfs42'' (at ) on the resulting
  morx0.squash and use the filesystem in the new squashfs-root directory for
  comparison.

\item Login to the router's web interface (at \url{http://192.168.10.1/ }) from a computer that is
  connected to the router.
  
\item Set a password using the provided link at the top (since the router's
  UI warns that no password is set and asks the user to change it).
  
\item Login to the router via SSH, using the root user with the
  aforementioned password.
  
\item Compare representative directory listings and binaries to ensure the set of
  included files (on the router) is similar to those found in the firmware image
  we created (whose contents are now in the local squashfs-root directory).  In
  particular, we did the following comparisons:

  \begin{enumerate}
  \item List the /bin folder (``ls -l /bin'') and confirm the list of files is the same
    and that the file sizes are similar.
    
  \item Check the ``strings'' output of ``/bin/busybox'' to confirm it was similar in both
   places (similar number of lines and content of lines).  (One cannot directly
   compare the binaries because the slight compilation variations will cause
   some bits to be different.)
 \item Do the above two steps for ``/lib/modules'', ``/usr/bin'', and other directories with
   a significant number of binaries.
   
 \item Check that the kernel is sufficiently similar.  The investigator
   compared the "dmesg" output both before and after flashing the new
   firmware.  As the investigator expected, the kernel version string was
   similar, but had a different build date and user@host indicator.  (The
   kernel binary itself is not easily accessible from an SSH login, but was
   retrievable using the U-Boot console (the start address of the kernel in
   flash appears to be 0x9F000000, based on the ``u-boot\verb0_0reflash''
   instructions).
  \end{enumerate}
\end{enumerate}

\section{Minor Annoyances}

As discussed in detail above, there were a few minor annoyances, none of
which were GPL violations.  Rather, the annoyances briefly impeded the
build and installation.  However, the investigator, as a reasonably skilled
build engineer for embedded devices, was able to complete the process with
the instructions provided.

To summarize, no GPL compliance issues were found, and the CCS release was
one of the best ever reviewed by an investigator.  However, the following
annoyances were discovered:

\begin{itemize}
\item Failure to explain how to extract the source tarball and then where to run the
  ``make'' command.
\item Failure to explain how to install the kernel and root filesystem on the
  device; the user must assume the web UI must be used.

\item Including pre-built toolchain binaries that don't work on all systems,
  and failure to built  toolchain binaries to the right location.
\end{itemize}

\section{Lessons Learned}

Companies that seek to redistribute copylefted software can benefit greatly
from ThinkPenguin's example.  Here are just a few of the many lessons that
can be learned here:

\begin{enumerate}

\item Even though copyleft licenses have them,
  \hyperref[thinkpenguin-included-ccs]{\bf avoid the offer-for-source
    provisions.}  Not only does including the CCS alongside binary
  distribution make violation investigation and compliance confirmation
  substantially easier, but more importantly it also
  \hyperref[offer-for-source]{completes the distributor's CCS compliance
    obligations at the time of distribution} (provided, of course, that the
  distributor is otherwise in compliance with copyleft.
  
\item {\bf Include top-level build instructions in a natural language (such
  as English) in a \hyperref[thinkpenguin-toplevel-readme]{clear and
    conspicuous place}.}  Copyleft licenses require that someone reasonably
  skilled in the art can reproduce your work.  Ultimately, sometimes
  instructions written in English are necessary, and often easier than trying
  to write programmed scripts to do everything.  The ``script'' included can
  certainly be more like the script of a play and less like a Bash script.

\item {\bf Write build/install instructions to the appropriate level of
  specificity}.  The upstream engineers
  in this case study \hyperref[thinkpenguin-specific-host-system]{clearly did
    additional work to ensure functionality on a wide variety of host build
    systems}; this is quite rare.  When in doubt, include the maximum level
  of detail build engineers can provide with the CCS instructions.

\item {\bf Seek to adhere to the spirit of copyleft, not just the letter of
  the license}.  ThinkPenguin uses encouragement of  users to improve and
  make their devices better as commercial differentiator.  Copyleft advocates
  remain baffled why other companies have not realized how large the market for
  users who seek hackable devices continues to grow.  By going beyond the
  mere minimal requirements of GPL, companies can immediately reap the
  benefits in that target market.
  
\end{enumerate}

%%%%%%%%%%%%%%%%%%%%%%%%%%%%%%%%%%%%%%%%%%%%%%%%%%%%%%%%%%%%%%%%%%%%%%%%%%%%%%%
\chapter{Bortez: Modified GCC SDK}

In our first case study, we will consider Bortez, a company that
produces software and hardware toolkits to assist OEM vendors, makers
of consumer electronic devices.

\section{Facts}

One of Bortez's key products is a Software Development Kit (``SDK'')
designed to assist developers building software for a specific class of
consumer electronics devices.

FSF received a report that the SDK may be based on the GNU Compiler
Collection (which is an FSF-copyrighted collection of tools for software
development in C, C++ and other popular languages). FSF investigated the
claim, but was unable to confirm the violation. The violation reporter
was unresponsive to follow-up requests for more information.

Since FSF was unable to confirm the violation, we did not pursue it any
further. Bogus reports do happen, and we do not want to burden companies
with specious GPL violation complaints. FSF shelved the matter until
more evidence was discovered.

FSF was later able to confirm the violation when two additional reports
surfaced from other violation reporters, both of whom had used the SDK
professionally and noticed clear similarities to FSF's GNU GCC\@. FSF's
Compliance Engineer asked the reporters to run standard tests to confirm
the violation, and it was confirmed that Bortez's SDK was indeed a
modified version of GCC\@. Bortez had ported to Windows and added a number
of features, including support for a specific consumer device chipset and
additional features to aid in the linking process (``LP'') for those
specific devices. FSF explained the rights that the GPL afforded these
customers and pointed out, for example, that Bortez only needed to provide
source to those in possession of the binaries, and that the users may need
to request that source (if \S 3(b) was exercised). The violators
confirmed that such requests were not answered.

FSF brought the matter to the attention of Bortez, who immediately
escalated the matter to their attorneys. After a long negotiation,
Bortez acknowledged that their SDK was indeed a modified version of
GCC\@. Bortez released most of the source, but some disagreement
occurred over whether LP was also derivative of GCC\@. After repeated
FSF inquiries, Bortez reaudited the source to discover that FSF's
analysis was correct. Bortez determined that LP included a number of
source files copied from the GCC code-base.

\label{davrik-build-problems}
Once the full software release was made available, FSF asked the violation
reporters if it addressed the problem. Reports came back that the source
did not properly build. FSF asked Bortez to provide better build
instructions with the software, and such build instructions were
incorporated into the next software release.

At FSF's request as well, Bortez informed customers who had previously
purchased the product that the source was now available by announcing
the availability on its Web site and via a customer newsletter.

Bortez did have some concerns regarding patents. They wished to include a
statement with the software release that made sure they were not granting
any patent permission other than what was absolutely required by the GPL\@.
They understood that their patent assertions could not trump any rights
granted by the GPL\@. The following language was negotiated into the release:

\begin{quotation}
Subject to the qualifications stated below, Bortez, on behalf of itself
and its Subsidiaries, agrees not to assert the Claims against you for your
making, use, offer for sale, sale, or importation of the Bortez's GNU
Utilities or derivative works of the Bortez's GNU Utilities
(``Derivatives''), but only to the extent that any such Derivatives are
licensed by you under the terms of the GNU General Public License. The
Claims are the claims of patents that Bortez or its Subsidiaries have
standing to enforce that are directly infringed by the making, use, or
sale of an Bortez Distributed GNU Utilities in the form it was distributed
by Bortez and that do not include any limitation that reads on hardware;
the Claims do not include any additional patent claims held by Bortez that
cover any modifications of, derivative works based on or combinations with
the Bortez's GNU Utilities, even if such a claim is disclosed in the same
patent as a Claim. Subsidiaries are entities that are wholly owned by
Bortez.

This statement does not negate, limit or restrict any rights you already
have under the GNU General Public License version 2.
\end{quotation}

This quelled Bortez's concerns about other patent licensing they sought to
do outside of the GPL'd software, and satisfied FSF's concerns that Bortez
give proper permissions to exercise teachings of patents that were
exercised in their GPL'd software release.

Finally, a GPL Compliance Officer inside Bortez was appointed to take
responsibility for all matters of GPL compliance inside the company.
Bortez is responsible for informing FSF if the position is given to
someone else inside the company, and making sure that FSF has direct
contact with Bortez's Compliance Officer.

\section{Lessons}

This case introduces a number of concepts regarding GPL enforcement.

\begin{enumerate}

\item {\bf Enforcement should not begin until the evidence is confirmed.}
  Most companies that distribute GPL'd software do so in compliance, and at
  times, violation reports are mistaken. Even with extensive efforts in
  GPL education, many users do not fully understand their rights and the
  obligations that companies have. By working through the investigation
  with reporters, the violation can be properly confirmed, and {\bf the
    user of the software can be educated about what to expect with GPL'd
    software}. When users and customers of GPL'd products know their
  rights, what to expect, and how to properly exercise their rights
  (particularly under \S 3(b)), it reduces the chances for user
  frustration and inappropriate community outcry about an alleged GPL
  violation.

\item {\bf GPL compliance requires friendly negotiation and cooperation.}
  Often, attorneys and managers are legitimately surprised to find out
  GPL'd software is included in their company's products. Engineers
  sometimes include GPL'd software without understanding the requirements.
  This does not excuse companies from their obligations under the license,
  but it does mean that care and patience are essential for reaching GPL
  compliance. We want companies to understand that participating and
  benefiting from a collaborative Free Software community is not a burden,
  so we strive to make the process of coming into compliance as smooth as
  possible.

\item {\bf Confirming compliance is a community effort.}  The whole point
  of making sure that software distributors respect the terms of the GPL is to
  allow a thriving software sharing community to benefit and improve the
  work. FSF is not the expert on how a compiler for consumer electronic
  devices should work. We therefore inform the community who originally
  brought the violation to our attention and ask them to assist in
  evaluation and confirmation of the product's compliance. Of course, FSF
  coordinates and oversees the process, but we do not want compliance for
  compliance's sake; rather, we wish to foster a cooperating community of
  development around the Free Software in question, and encourage the
  once-violator to begin participating in that community.

\item {\bf Informing the harmed community is part of compliance.} FSF asks
  violators to make some attempt --- such as via newsletters and the
  company's Web site --- to inform those who already have the products as
  to their rights under the GPL\@. One of the key thrusts of the GPL's \S 1 and
  \S 3 is to {\em make sure the user knows she has these rights\/}. If a
  product was received out of compliance by a customer, she may never
  actually discover that she has such rights. Informing customers, in a
  way that is not burdensome but has a high probability of successfully
  reaching those who would seek to exercise their freedoms, is essential
  to properly remedy the mistake.

\item {\bf Lines between various copyright, patent, and other legal
  mechanisms must be precisely defined and considered.}  The most
  difficult negotiation point of the Bortez case was drafting language
  that simultaneously protected Bortez's patent rights outside of the
  GPL'd source, but was consistent with the implicit patent grant in
  the GPL\@. As we discussed in the first course of this series, there is
  indeed an implicit patent grant with the GPL, thanks to \S 6 and \S 7.
  However, many companies become nervous and wish to make the grant
  explicit to assure themselves that the grant is sufficiently narrow for
  their needs. We understand that there is no reasonable way to determine
  what patent claims read on a company's GPL holdings and which do not, so
  we do not object to general language that explicitly narrows the patent
  grant to only those patents that were, in fact, exercised by the GPL'd
  software as released by the company.

\end{enumerate}

%%%%%%%%%%%%%%%%%%%%%%%%%%%%%%%%%%%%%%%%%%%%%%%%%%%%%%%%%%%%%%%%%%%%%%%%%%%%%%%
\chapter{Bracken: a Minor Violation in a GNU/Linux Distribution}

In this case study, we consider a minor violation made by a company whose
knowledge of the Free Software community and its functions is deep.

\section{The Facts} 

Bracken produces a GNU/Linux operating system product that is sold
primarily to OEM vendors to be placed in appliance devices used for a
single purpose, such as an Internet-browsing-only device. The product
is almost 100\% Free Software, mostly licensed under the GPL and related
Free Software licenses.

FSF found out about this violation through a report first posted on a
  Slashdot\footnote{Slashdot is a popular news and discussion site for
  technical readers.} comment, and then it was brought to our attention again
  by another Free Software copyright holder who had discovered the
  same violation.

Bracken's GNU/Linux product is delivered directly from their Web site.
This allowed FSF engineers to directly download and confirm the
violation quickly. Two primary problems were discovered with the
online distribution:

\begin{itemize}

\item No source code nor offer for source code was provided for a number
  of components for the distributed GNU/Linux system; only binaries were
  available

\item An End User License Agreement (``EULA'') was included that
  contradicted the permissions granted by the GPL\@

\end{itemize}

FSF contacted Bracken and gave them the details of the violation. Bracken
immediately ceased distribution of the product temporarily and set forth
a plan to bring themselves back into compliance. This plan included the
following steps:

\begin{itemize}

\item Bracken attorneys would rewrite the EULA to comply with the GPL and
  would vet the new EULA through FSF before use

\item Bracken engineers would provide source side-by-side with the
  binaries for the GNU/Linux distribution on the site (and on CD's, if
  ever they distributed that way)

\item Bracken attorneys would run an internal seminar for its engineers
  regarding proper GPL compliance to help ensure that such oversights
  regarding source releases would not occur in the future

\item Bracken would resume distribution of the product only after FSF
  formally restored Bracken's distribution rights
\end{itemize}

This case was completed in about a month. FSF approved the new EULA
text. The key portion in the EULA relating to the GPL read as follows:

\begin{quotation}
Many of the Software Programs included in Bracken Software are distributed
under the terms of agreements with Third Parties (``Third Party
Agreements'') which may expand or limit the Licensee's rights to use
certain Software Programs as set forth in [this EULA]. Certain Software
Programs may be licensed (or sublicensed) to Licensee under the GNU
General Public License and other similar license agreements listed in part
in this section which, among other rights, permit the Licensee to copy,
modify and redistribute certain Software Programs, or portions thereof,
and have access to the source code of certain Software Programs, or
portions thereof. In addition, certain Software Programs, or portions
thereof, may be licensed (or sublicensed) to Licensee under terms stricter
than those set forth in [this EULA]. The Licensee must review the
electronic documentation that accompanies certain Software Programs, or
portions thereof, for the applicable Third Party Agreements. To the
extent any Third Party Agreements require that Bracken provide rights to
use, copy or modify a Software Program that are broader than the rights
granted to the Licensee in [this EULA], then such rights shall take
precedence over the rights and restrictions granted in this Agreement
solely for such Software Programs.
\end{quotation}

FSF restored Bracken's distribution rights shortly after the work was
completed as described.

\section{Lessons Learned}

This case was probably the most quickly and easily resolved of all GPL
violations in the history of FSF's Compliance Lab. The ease with which
the problem was resolved shows a number of cultural factors that play a
role in GPL compliance.

\begin{enumerate}

\item {\bf Companies that understand Free Software culture better have an
  easier time with compliance.}  Bracken's products were designed and
  built around the GNU/Linux system and Free Software components. Their
  engineers were deeply familiar with the Free Software ecosystem, and
  their lawyers had seen and reviewed the GPL before. The violation was
  completely an honest mistake. Since the culture inside the company had
  already adapted to the cooperative style of resolution in the Free
  Software world, there was very little work for either party to bring the
  product into compliance.

\item {\bf When people in key positions understand the Free Software
  nature of their software products, compliance concerns are as
  mundane as minor software bugs.}  Even the most functional system or
  structure has its problems, and successful business often depends on
  agile response to the problems that do come up; avoiding problems
  altogether is a pipe dream. Minor GPL violations can and do happen
  even with well-informed redistributors. However, resolution is
  reached quickly when the company --- and in particular, the lawyers,
  managers, and engineers working on the Free Software product lines
  --- have adapted to Free Software culture that the lower-level
  engineer already understood

\item {\bf Legally, distribution must stop when a violation is
  identified.}  In our opinion, Bracken went above and beyond the call of
  duty by ceasing distribution while the violation was being resolved.
  Under GPL \S 4, the redistributor loses the right to distribute the
  software, and thus they are in ongoing violation of copyright law if
  they distribute before rights are restored. It is FSF's policy to
  temporarily allow distribution while compliance negotiations are ongoing
  and only in the most extreme cases (where the other party appears to be
  negotiating in bad faith) does FSF even threaten an injunction on
  copyright grounds. However, Bracken --- as a good Free Software citizen
  --- chose to be on the safe side and do the legally correct thing while
  the violation case was pending. From start to finish, it took less
  than a month to resolve. This lapse in distribution did not, to FSF's
  knowledge, impact Bracken's business in any way.

\item {\bf EULAs are a common area for GPL problems.}  Often, EULAs
  are drafted from boilerplate text that a company uses for all its
  products. Even the most diligent attorneys forget or simply do not
  know that a product contains software licensed under the GPL and other
  Free Software licenses. Drafting a EULA that accounts for such
  licenses is straightforward; the text quoted above works just fine.
  The EULA must be designed so that it does not trump rights and
  permissions already granted by the GPL\@. The EULA must clearly state
  that if there is a conflict between it and the GPL, with regard to GPL'd
  code, the GPL is the overriding license.

\item {\bf Compliance Officers are rarely necessary when companies are
  educated about GPL compliance.}  As we saw in the Bortez case, FSF asks
  that a formal ``GPL Compliance Officer'' be appointed inside a
  previously violating organization to shepherd the organization to a
  cooperative approach to GPL compliance. However, when FSF
  sees that an organization already has such an approach, there is no
  need to request that such an officer be appointed.

\end{enumerate}


%%%%%%%%%%%%%%%%%%%%%%%%%%%%%%%%%%%%%%%%%%%%%%%%%%%%%%%%%%%%%%%%%%%%%%%%%%%%%%%
\chapter{Vigorien: Security, Export Controls, and GPL Compliance}

This case study introduces how concerns of ``security through obscurity''
and regulatory problems can impact GPL compliance matters.

\section{The Facts}

Vigorien distributes a back-up solution product that allows system
administrators to create encrypted backups of file-systems on
Unix-like computers. The product is based on GNU tar, a backup utility
that replaces the standard Unix utility simply called tar, but has
additional features.

Vigorien's backup solution added cryptographic features to GNU tar, and
included a suite of utilities and graphical user interfaces surrounding
GNU tar to make backups convenient.

FSF discovered the violation from a user report, and determined that the
cryptographic features were the only part of the product that constituted
a derivative work of GNU tar; the extraneous utilities merely made
shell calls out to GNU tar. FSF requested that Vigorien come into
compliance with the GPL by releasing the source of GNU tar, with the
cryptographic modifications, to its customers.

Vigorien released the original GNU tar sources, but kept the cryptographic
modifications proprietary. They argued that the security of their system
depending on keeping the software proprietary and that regardless, USA
export restrictions on cryptographic software prohibited such a release.
FSF disputed the first claim, pointing out that Vigorien had only one
option if they did not want to release the source: they would have to
remove GNU tar from the software and not distribute it further. Vigorien
rejected this suggestion, since GNU tar was an integral part of the
product, and the security changes were useless without GNU tar.

Regarding the export control claims, FSF proposed a number of options,
including release of the source from one of Vigorien's divisions overseas
where no such restrictions occurred, but Vigorien argued that the problem
was insoluble because they operated primarily in the USA\@.

The deadlock on the second issue was resolved when those cryptographic
export restrictions were lifted shortly thereafter, and FSF again raised
the matter with Vigorien. At that point, they dropped the first claim and
agreed to release the remaining source module to their customers. They
did so, and the violation was resolved.


\section{Lessons Learned}

\begin{enumerate}

\item {\bf Removing the GPL'd portion of the product is always an
  option.}  Many violators' first response is to simply refuse to
  release the source code as the GPL requires. FSF offers the option to
  simply remove the GPL'd portions from the product and continue along
  without them. Every case where this has been suggested has led to
  the same conclusion. Like Vigorien, the violator argues that the
  product cannot function without the GPL'd components, and they
  cannot effectively replace them.

  Such an outcome is simply further evidence that the combined work in
  question is indeed a modified version of the original GPL'd component.
  If the other components cannot stand on their own and be useful without
  the GPL'd portions, then one cannot effectively argue that the work as a
  whole is not a based on the GPL'd portions.

\item {\bf The whole product is not always covered.}  In this case,
  Vigorien had additional works aggregated. The backup system was a suite
  of utilities, some of which were the GPL and some of which were not. While
  the cryptographic routines were tightly coupled with GNU tar and clearly
  made a whole new combined work of both components, the various GUI utilities were separate and
  independent works merely aggregated with the distribution of the
  GNU-tar-based product.


\item {\bf ``Security'' concerns do not exonerate a distributor from GPL
  obligations, and ``security through obscurity'' does not work anyway.}
  The argument that ``this is security software, so it cannot be released
  in source form'' is not a valid defense for explaining why the terms of
  the GPL are ignored. If companies do not want to release source code
  for some reason, then they should not base the work on GPL'd software.
  No external argument for noncompliance can hold weight if the work as
  a whole is indeed a modified version of a GPL'd program.

  The ``security concerns'' argument is often floated as a reason to keep
  software proprietary, but the computer security community has on
  numerous occasions confirmed that such arguments are entirely specious.
  Security experts have found --- since the beginnings of the field of
  cryptography in the ancient world --- that sharing results about systems
  and having such systems withstand peer review and scrutiny builds the
  most secure systems. While full disclosure may help some who wish to
  compromise security, it helps those who want to fix problems even more
  by identifying them early.

\item {\bf External regulatory problems can be difficult to resolve.}
  The GPL, though grounded in copyright law, does not have the power to trump
  regulations like export controls. While Vigorien's ``security
  concerns'' were specious, their export control concerns were not. It is
  indeed a difficult problem that FSF acknowledges. We want compliance
  with the GPL and respect for users' freedoms, but we certainly do not expect
  companies to commit criminal offenses for the sake of compliance. We
  will see more about this issue in our next case study.
\end{enumerate}


%%%%%%%%%%%%%%%%%%%%%%%%%%%%%%%%%%%%%%%%%%%%%%%%%%%%%%%%%%%%%%%%%%%%%%%%%%%%%%%
\chapter{Haxil, Polgara, and Thesulac: Mergers, Upstream Providers and Radio Devices}

This case study considers an ongoing (at the time of writing) violation
that has occurred. By the end of the investigation period, three
companies were involved and many complex issues arose.

\section{The Facts}

Haxil produced a consumer electronics device which included a mini
GNU/Linux distribution to control the device. The device was of interest
to many technically-minded consumers, who purchased the device and very
quickly discovered that Free Software was included without source.
Mailing lists throughout the Free Software community erupted with
complaints about the problem, and FSF quickly investigated.

FSF confirmed that FSF-copyrighted GPL'd software was included. In
addition, the whole distribution included GPL'd works from hundreds of
individual copyright holders, many of whom were, at this point, up in
arms about the violation.

Meanwhile, Haxil was in the midst of being acquired by Polgara. Polgara
was as surprised as everyone else to discover the product was based on
GPL'd software; this fact had not been part of the disclosures made during
acquisition. FSF contacted Haxil, Polgara, and the product managers
who had transitioned into the ``Haxil division'' of the newly-merged
Polgara company. Polgara's General Counsel's office worked with FSF on
the matter.

FSF formed a coalition with the other primary copyright holders
to pursue the enforcement effort on their behalf. FSF communicated
directly with Polgara's representatives to begin working through the
issues on behalf of itself and the Free Software community at large.

Polgara pointed out that the software distribution they used was mostly
contributed by an upstream provider, Thesulac, and Haxil's changes to that
code base were minimal. Polgara negotiated with Thesulac to obtain the
source, although the issue moved very slowly in the channels between
Polgara and Thesulac.

FSF encouraged a round-table meeting so that high bandwidth communication
could occur between FSF, Polgara and Thesulac. Polgara and Thesulac
agreed, and that discussion began. Thesulac provided nearly complete
sources to Polgara, and Polgara made a full software release on their
Web site. At the time of writing, that software still has some build
problems (similar to those that occurred with Bortez, as described in
Section~\ref{davrik-build-problems}). FSF continues to negotiate with
Polgara and Thesulac to resolve these problems, which have a clear path to
a solution and are expected to resolve.

Similar to the Vigorien case, Thesulac has regulatory concerns. In this
case, it is not export controls --- an issue that has since been resolved
--- but radio spectrum regulation. Since this consumer electronic device
contains a software-programmable radio transmitter, regulations in (at
least) the USA and Japan prohibit release of those portions of the code
that operate the device. Since this is a low-level programming issue, the
changes to operate the device form a single combined work with the kernel named
Linux.  A decade later, this situation remains largely unresolved.

\section{Lessons Learned}

\begin{enumerate}

\item {\bf Community outrage, while justified, can often make negotiation
  more difficult.}  FSF has a strong policy never to publicize names of
  GPL violators if they are negotiating in a friendly way and operating in
  good faith toward compliance. Most violations are honest mistakes, and
  FSF sees no reason to publicly admonish violators who genuinely want to
  come into compliance with the GPL and to work hard staying in compliance.

  This case was so public in the Free Software community that both Haxil's
  and Polgara's representatives were nearly shell-shocked by the time FSF
  began negotiations. There was much work required to diffuse the
  situation. We empathize with our community and their outrage about GPL
  violations, but we also want to follow a path that leads expediently
  to compliance. In our experience, public outcry works best as a last
  resort, not the first.

\item {\bf For software companies, GPL compliance belongs on a corporate
  acquisition checklist. }  Polgara was truly amazed that Haxil had used
  GPL'd software in a major new product line but never informed Polgara
  during the acquisition process. While GPL compliance is not a
  particularly difficult matter, it is an additional obligation that comes
  along with the product line. When planning mergers and joint ventures,
  one should include lists of GPL'd components contained in the products
  discussed.

\item {\bf Compliance problems of upstream providers do not excuse a
  violation for the downstream distributor.}  To paraphrase \S 6, upstream
  providers are not responsible for enforcing compliance of their
  downstream, nor are downstream distributors responsible for compliance
  problems of upstream providers. However, engaging in distribution of
  GPL'd works out of compliance is still just that: a compliance problem.
  When FSF carries out enforcement, we are patient and sympathetic when
  the problem appears to be upstream. In fact, we urge the violator to
  point us to the upstream provider so we may talk to them directly. In
  this case, we were happy to begin negotiations with Thesulac. However,
  Polgara still has an obligation to bring their product into compliance,
  regardless of Thesulac's response.

\item {\bf It behooves upstream providers to advise downstream
  distributors about compliance matters.}  FSF has encouraged Thesulac to
  distribute a ``good practices for GPL compliance'' document with their
  product. Polgara added various software components to Thesulac's
  product, and it is conceivable that such additions can introduce
  compliance. In FSF's opinion, Thesulac is in no way legally responsible
  for such a violation introduced by their customer, but it behooves them
  from a marketing standpoint to educate their customers about using the
  product. We can argue whether or not it is your coffee vendor's fault
  if you burn yourself with their product, but (likely) no one on either
  side would dispute the prudence of placing a ``caution: hot'' label on
  the cup.

\item {\bf FSF enforcement often avoids redundant enforcement cases from
  many parties.}  Most Free Software systems have hundreds of copyright
  holders. Some have thousands. FSF is in a unique position as one of
  the largest single copyright holders on GPL'd software and as a
  respected umpire in the community, neutrally enforcing the rules of the
  GPL road. FSF works hard in the community to convince copyright
  holders that consolidating GPL claims through FSF is better for them,
  and more likely to yield positive compliance results.

  A few copyright holders engage in the ``proprietary relicensing''
  business, so they use GPL enforcement as a sales channel for that
  business. FSF, as a community-oriented, not-for-profit organization,
  seeks only to preserve the freedom of Free Software in its enforcement
  efforts. As it turns out, most of the community of copyright holders
  of Free Software want the same thing. Share and share alike is a
  simple rule to follow, and following that rule to FSF's satisfaction
  usually means you are following it to the satisfaction of the entire
  Free Software community.

\end{enumerate}

%%%%%%%%%%%%%%%%%%%%%%%%%%%%%%%%%%%%%%%%%%%%%%%%%%%%%%%%%%%%%%%%%%%%%%%%%%%%%%%


%%%%%%%%%%%%%%%%%%%%%%%%%%%%%%%%%%%%%%%%%%%%%%%%%%%%%%%%%%%%%%%%%%%%%%%%%%%%%%%
% COMMENT OUT THIS CHAPTER.
% FIXME: is this material moot now that we include the compliance guide?
% Either way, it should be merged into compliance guide.
%\chapter{Good Practices for Compliance}

Generally, from the experience of GPL enforcement, we glean the following
general practices that can help in GPL compliance for organizations that
distribute products based on GPL'd software:

\begin{itemize}

\item Talk to your software engineers and ask them where they got the
  components they use in the products they build. Find out if GPL'd
  components are present.

\item Teach your engineering staff to pay attention to license documents.
  Give them easy-to-follow policies to get approval for using a Free
  Software component.

\item Build a ``Free Software Licensing'' committee that handles requests
  and questions about the GPL and other Free Software licenses.

\item Add ``What parts of your products are under the GPL or other Free
  Software licenses?'' to your checklist of questions to ask when you
  consider mergers, acquisitions, or joint ventures.

\item Encourage your engineers to participate collaboratively with GPL'd
  software development. The more knowledge about the Free Software world
  your organization has, the better equipped it is to deal with this
  rapidly changing field.

\item When someone points out a potential GPL violation in one of your
  products, do not assume the product line is doomed. The GPL is not a virus;
  merely having GPL'd code in one part of a product does not necessarily
  mean that every related product must also be GPL'd. And, even if some
  software needs to be released that was not before, the product will
  surely survive. In FSF's enforcement efforts, we have not yet
  seen a product line die because source was released to customers in
  compliance with the GPL.

\end{itemize}

%%%%%%%%%%%%%%%%%%%%%%%%%%%%%%%%%%%%%%%%%%%%%%%%%%%%%%%%%%%%%%%%%%%%%%%%%%%%%%%
% LocalWords:  proprietarize redistributors sublicense yyyy Gnomovision EULAs
% LocalWords:  Yoyodyne FrontPage improvers Berne copyrightable Stallman's GPLs
% LocalWords:  Lessig Lessig's UCITA pre PDAs CDs reshifts GPL's Gentoo glibc
% LocalWords:  TrollTech administrivia LGPL's MontaVista OpenTV Mitek Arce DVD
% LocalWords:  unprotectable protectable Unfreedonia chipset CodeSourcery Iqtel
% LocalWords:  impermissibly Bateman faire minimis Borland uncopyrightable Mgmt
% LocalWords:  franca downloadable Bortez Bortez's
% LocalWords:  Slashdot sublicensed Vigorien Vigorien's Haxil Polgara
% LocalWords:  Thesulac Polgara's Haxil's Thesulac's SDK CD's


\appendix

\part{Appendices}

% attributions.tex                                                  -*- LaTeX -*-
%    Part containing all attributions in one place. 
%
% Copyright (C) 2014, Bradley M. Kuhn

\chapter{Citations of Incorporated Material from Other Published Works}

As a public, collaborative project, this Guide is primarily composed of the
many contributions received via its
\href{https://gitorious.org/copyleft-org/tutorial/source/master:CONTRIBUTING.md}{public
  contribution process}.  Please
\href{https://gitorious.org/copyleft-org/tutorial/history/master}{review its
  Git logs} for full documentation of all contributions.
  
Below is a list of CC-By-SA-licensed works, with specific titles and
publication dates, from which any material was incorporated into this Guide.
The specific methods and details of incorporation are fully
documented in the
\href{https://gitorious.org/copyleft-org/tutorial/history/master}{Git logs}
of the project.
  
\begin{itemize}
\item \textit{Detailed Analysis of the GNU GPL and Related Licenses}, written by
Bradley M. Kuhn, Daniel B.~Ravicher, and John Sullivan and published by the Free Software Foundation for its CLE courses on 2004-01-20,
2004-08-24, and 2014-03-24.
\item \hrefnofollow{http://gplv3.fsf.org/gpl-rationale-2006-01-16.html}{\textit{GPLv3 First Discussion Draft Rationale}}, written and published by the Free
  Software Foundation on 2006-01-16.
\item \hrefnofollow{http://gplv3.fsf.org/opinions-draft-2.html}{\textit{GPLv3 Second Discussion Draft Rationale}}, written and published by the Free
  Software Foundation circa 2006-07.
\item \hrefnofollow{http://gplv3.fsf.org/gpl3-dd3-guide}{\textit{GPLv3 Third Discussion Draft Rationale}}, written and published by the Free
  Software Foundation on   2007-03-28.
\item \hrefnofollow{http://gplv3.fsf.org/dd3-faq}{\textit{GPLv3  Discussion Draft 3 FAQ}}, written and published by the Free1 Software Foundation on   2007-03-28.
\item \hrefnofollow{http://gplv3.fsf.org/gpl3-dd4-guide.html}{\textit{GPLv3 Final Discussion Draft Rationale}} written and published by the Free
  Software Foundation onon 2007-05-31.
\item \hrefnofollow{http://www.gnu.org/licences/gpl3-final-rationale.pdf}{\textit{GPLv3 Final Rationale}}, written and published by the Free
  Software Foundation on 2007-06-29.
\item \hrefnofollow{http://www.softwarefreedom.org/resources/2008/compliance-guide.html}{\textit{A Practical Guide GPL Compliance}} written by Bradley M. Kuhn, Aaron
Williamson and Karen Sandler and first published on 2008-08-20.
\item \hrefnofollow{http://www.softwarefreedom.org/resources/2014/SFLC-Guide_to_GPL_Compliance_2d_ed.html}{\textit{Software Freedom Law Center Guide to GPL Compliance, 2nd
  Edition}} by Eben Moglen and Mishi Choudhary and first published on 2014-10-31.
\item \textit{Enforcement Case Studies}, written by Bradley M. Kuhn and published by the Free
  Software Foundation for its CLE courses  on 2004-01-20, 2004-08-24, and 2014-03-24.
\end{itemize}

Please note, however, that this list above does not include nor adequately
represent the substantial contributions from those who directly
contributed to this Guide using its Git (and formerly, CVS) repository.
Rather, this is a list of third-party published works from which any text was
herein included under their CC-By-SA licensing.  Thus, as the reader might
expect, the
\href{https://gitorious.org/copyleft-org/tutorial/history/master}{version
  control logs} contain the only true and accurate view available of who has
contributed which portions of this project.


\input{license-texts}


\end{document}
